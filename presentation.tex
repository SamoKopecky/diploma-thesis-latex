% Soubory musí být v kódování, které je nastaveno v příkazu \usepackage[...]{inputenc}

\documentclass[%        Základní nastavení
  %draft,    				  % Testovací překlad
  14pt,       				% Velikost základního písma je 12 bodů
	t,                  % obsah slajdů bude vždy začínat od shora (nebude vertikálně centrovaný)
	aspectratio=1610,   % poměr stran bude 16:10 (všechny projektory v učebnách na Technické 12 Brno),
	                    % další volby jsou 43, 149, 169, 54, 32.
	unicode,						% Záložky a informace budou v kódování unicode
]{beamer}				    	% Dokument třídy 'zpráva', vhodná pro sazbu závěrečných prací s kapitolami
%\usepackage{etex}

\usepackage[utf8]		  % Kódování zdrojových souborů je v UTF-8
	{inputenc}					% Balíček pro nastavení kódování zdrojových souborů

\usepackage{graphicx} % Balíček 'graphicx' pro vkládání obrázků
											% Nutné pro vložení logotypů školy a fakulty

\usepackage[          % Balíček 'acronym' pro sazby zkratek a symbolů
	nohyperlinks				% Nebudou tvořeny hypertextové odkazy do seznamu zkratek
]{acronym}
											% Nutné pro použití prostředí 'acronym' balíčku 'thesis'

%% Balíček hyperref je volán třídou beamer automaticky, proto není třeba následujícího kódu:
%\usepackage[
%	breaklinks=true,		% Hypertextové odkazy mohou obsahovat zalomení řádku
%	hypertexnames=false % Názvy hypertextových odkazů budou tvořeny
%											% nezávisle na názvech TeXu
%]{hyperref}						% Balíček 'hyperref' pro sazbu hypertextových odkazů
%											% Nutné pro použití příkazu 'nastavenipdf' balíčku 'thesis'

\usepackage{cmap} 		% Balíček cmap zajišťuje, že PDF vytvořené `pdflatexem' je
											% plně "prohledávatelné" a "kopírovatelné"

%\usepackage{upgreek}	% Balíček pro sazbu stojatých řeckých písmem
											%% např. stojaté pí: \uppi
											%% např. stojaté mí: \upmu (použitelné třeba v mikrometrech)
											%% pozor, grafická nekompatibilita s fonty typu Computer Modern!

%\usepackage{amsmath} %balíček pro sabu náročnější matematiky

\usepackage{booktabs} % Balíček, který umožňuje v tabulce používat
                      % příkazy \toprule, \midrule, \bottomrule


%%%%%%%%%%%%%%%%%%%%%%%%%%%%%%%%%%%%%%%%%%%%%%%%%%%%%%%%%%%%%%%%%
%%%%%%      Definice informací o dokumentu             %%%%%%%%%%
%%%%%%%%%%%%%%%%%%%%%%%%%%%%%%%%%%%%%%%%%%%%%%%%%%%%%%%%%%%%%%%%%

% Variable fields such as your name, title of the thesis atc. are set in this file.
% The file is SHARED between the thesis text and the presentation --- no need to set anything twice.

\usepackage[
  english,			% English study program
%%% Choose one of the thesis types
	semestral,		%	semestral project (abstracts, declaration and acknowledgements are excluded) (default)
  %bachelor,			%	Bachelor's thesis
  %master,			% Master's thesis
  %treatise,			% Treatise on doctoral thesis
  %doctoral,				% Doctoral thesis
%%% Choose one of the options:
%  left,				% Equations and captions will be aligned to the left
  center,			% Equations and captions will be centered (default)
]{thesis}   % Package for typesetting theses


%%% First and last name of the thesis author, following the scheme
% [titles in front]{FirstName}{LastName}[titles after]
% If the author does not have a title in front/after, just delete the entire string '[...]'
\author[Bc.]{Samuel}{Kopecký}

%%% Brno University of Technology identification number of author (BUT ID)
\butid{211799}

%%% First and last name of the advisor
% [titles in front]{FirstName}{LastName}[titles after]
% If the author does not have a title in front/after, just delete the entire string '[...]'
% [titles in front]{FirstName}{LastName}[titles after]
\advisor[Ing.]{David}{Smékal}[]

%%% First and last name of the opponent
% [titles in front]{FirstName}{LastName}[titles after]
% Makes use only in the presentation for defense;
% in case you do not want the opponent to be shown on the title slide, just comment out the command
% The opponent is not shown in case of the semestral thesis (no opponent exists actually at that time)
% In the case of a doctoral thesis, two to three oponents are typically involved. In such a case, if you want to have them on the title slide, please go to the definition of "VUT title page" in the thesis.sty file, uncomment and adapt their names.
\opponent[TODO]{TODO}{TODO}[TODO]

%%% Thesis title
%  In the case of a very long thesis title, it may happen that it does not fit
%  into the slot in the footbar of the slides. You may use the command
%  \def\insertshorttitle{Shortened th.\ title}
%  where the shortened version of the title appears as the parameter.
%  If you do not want to shorten the title, you will have to redefine how the slide footbar
%  is generated, see: https://bit.ly/3EJTp5A
\title{Modular communication based on post-quantum cryptohraphy}

%%% Study program/specialization
\specialization{Information Security} %Teleinformatika

%%% Department
%\department{Department of Control and Instrumentation}  %Ústav automatizace a měřicí techniky
%\department{Department of Biomedical Engineering}       %Ústav biomedicínského inženýrství
%\department{Department of Electrical Power Engineering} %Ústav elektroenergetiky
%\department{Department of Electrical and Electronic Technology}   %Ústav elektrotechnologie
%\department{Department of Physics}                      %Ústav fyziky
%\department{Department of Foreign Languages}            %Ústav jazyků
%\department{Department of Mathematics}                  %Ústav matematiky
%\department{Department of Microelectronics}             %Ústav mikroelektroniky
%\department{Department of Radio Electronics}            %Ústav radioelektroniky
%\department{Department of Theoretical and Experimental Electrical Engineering}  %Ústav teoretické a experimentální elektrotechniky
\department{Department of Telecommunications}           %Ústav telekomunikací
%\department{Department of Power Electrical and Electronic Engineering}   %Ústav výkonové elektrotechniky a elektroniky

%%% Faculty
%\faculty{Faculty of Architecture}   %Fakulta architektury
\faculty{Faculty of Electrical Engineering and~Communication}   %Fakulta elektrotechniky a~komunikačních technologií
%\faculty{Faculty of Chemistry}   %Fakulta chemická
%\faculty{Faculty of Information Technology}   %Fakulta informačních technologií
%\faculty{Faculty of Business and Management}   $Fakulta podnikatelská
%\faculty{Faculty of Civil Engineering}   %Fakulta stavební
%\faculty{Faculty of Mechanical Engineering}   %Fakulta strojního inženýrství
%\faculty{Faculty of Fine Arts}   %Fakulta výtvarných umění
%
%Logotype selection (in square brackets short logo, in curly brackets full logo):
\facultylogo[logo/FEEC_abbreviation_color_PANTONE_EN]{logo/UTKO_color_PANTONE_EN}


%%% Graduate year (typically the calendar year of the defense)
\graduateyear{2023}
%%% Academic year (typically the year of solution of the thesis in the format n/n+1)
\academicyear{2022/23}
% Date of the defense (makes use only in the presentation slides)
\date{TODO} 

%%% Place of the defense
\city{Brno}

%%% Abstract
\abstract{%
Abstract in English.
}

%%% Keywords
\keywrds{%
Keywords in English
}

%%% Thanks and acknowledgement
\acknowledgement{%
TODO
}%      % v tomto souboru doplňte údaje o sobě, o názvu práce...
                       % (tento soubor je sdílený s textem práce)


%%%%%%%%%%%%%%%%%%%%%%%%%%%%%%%%%%%%%%%%%%%%%%%%%%%%%%%%%%%%%%%%%%%%%%%%

%%%%%%%%%%%%%%%%%%%%%%%%%%%%%%%%%%%%%%%%%%%%%%%%%%%%%%%%%%%%%%%%%%%%%%%%
%%%%%%     Nastavení polí ve Vlastnostech dokumentu PDF      %%%%%%%%%%%
%%%%%%%%%%%%%%%%%%%%%%%%%%%%%%%%%%%%%%%%%%%%%%%%%%%%%%%%%%%%%%%%%%%%%%%%
%% Při vloženém balíčku 'hyperref' lze použít příkaz '\pdfsettings'
\pdfsettings
%  Nastavení polí je možné provést také ručně příkazem:
%\hypersetup{
%  pdftitle={Název studentské práce},    	% Pole 'Document Title'
%  pdfauthor={Autor studenstké práce},   	% Pole 'Author'
%  pdfsubject={Typ práce}, 						  	% Pole 'Subject'
%  pdfkeywords={Klíčová slova}           	% Pole 'Keywords'
%}
\hypersetup{pdfpagemode=FullScreen}       % otevření rovnou v režimu celé obrazovky
%%%%%%%%%%%%%%%%%%%%%%%%%%%%%%%%%%%%%%%%%%%%%%%%%%%%%%%%%%%%%%%%%%%%%%%

\usetheme{VUT} 				% barvy a rozložení prezentace odpovídající VUT FEKT
% alternativně lze použít jiná berevná témata, ale bez záruky. Například:
%\usetheme{Darmstadt} \usecolortheme{default2}
\logoheader					% vytvoření zkráceného loga VUT FEKT v hlavičce slajdu, nechte odkomentované


\begin{document}

% v případě zakomentování následujícího se zobrazí v pravém dolním rohu slajdů klikatelné navigační symboly
\disablenavigationsymbols


% titulní snímek, vysazen bez horních, dolních a postranních lišt (volba plain),
% není tak vysazen ani nadpis snímku
\maketitle
\newcommand{\npm}[2]{{#1}\,$\pm\,{#2}\%$}

\begin{frame}[c]
	\frametitle{Ciele práce}
	\large{\begin{itemize}
			\item Optimalizácia post kvantových algoritmov
			\item Protokol na výmenu dát z zabepečeným post kvantovou kryptografiou
			\item Systém modularity kryptografických algoritmov
			\item Textové terminálové rozhranie
			\item Grafické terminálové rozhranie -- TUI
		\end{itemize}
	}
\end{frame}

\begin{frame}[c]
	\frametitle{Post kvantové algoritmy}
	\begin{columns}[T]
		\begin{column}{0.8\textwidth}
			\large{\begin{itemize}
					\item Algoritmus výmenu tajného klúča -- Kyber
					\item Algoritmus digitálneho podpisu -- Dilithium
					\item NIST štandardrizované algoritmy
					\item Implementácia v jazyku Go (Golang)
					\item Výsledky implementácie porovnávané z kničnicou Circl od Cloudflare
				\end{itemize}
			}
		\end{column}
		\begin{column}{0.2\textwidth}
			\begin{figure}
				\includegraphics[width=\linewidth]{presentation_pictures/Golang_Logo.png}
			\end{figure}
		\end{column}
	\end{columns}
\end{frame}

\begin{frame}[c]
	\frametitle{Výsledky implementácie}
	\large{
		\begin{itemize}
			\item Testované na AMD 3600 s frekvenciou jadra 3.6\,Ghz
		\end{itemize}
	}
	\normalsize{
		\begin{table}[h!]
			\centering
			\begin{tabular}{|l|r|r|}
  \hline
             & PqCom [\textmu s] & Circl [\textmu s] \\
  \hline
  \hline
  Kyber512   & \npm{459.9}{1}    & \npm{109.7}{2}    \\
  Kyber768   & \npm{707.9}{1}    & \npm{171.7}{2}    \\
  Kyber1024  & \npm{1019}{1}     & \npm{267.4}{2}    \\
  \hline
  Dilithium2 & \npm{1986}{1}     & \npm{472.9}{1}    \\
  Dilithium3 & \npm{3206}{2}     & \npm{788.1}{0}    \\
  Dilithium5 & \npm{4130}{1}     & \npm{1054}{1}     \\
  \hline
\end{tabular}

		\end{table}
	}
\end{frame}

\begin{frame}[c]
	\frametitle{Protokol sieťovej komunikácie}
	\large{\begin{itemize}
			\item Založený na protokole TCP, predvolený port 4040
			\item Zabezpečený len post kvantovou kryptografiou
			\item Šifrovanie a integrita zaistená pomcou AES-256-GCM
			\item Algoritmy pre výmna klúča a digitálny podpis sú modulárne
			\item Používa fixnú velkosť hlavičky ktorá ma iba 3\,B
		\end{itemize}}
	\begin{figure}
		\includegraphics[width=.8\textwidth]{presentation_pictures/header_sk.pdf}
	\end{figure}
\end{frame}

\begin{frame}[c]
	\frametitle{Protokol sieťovej komunikácie}
	\large{\begin{itemize}
			\item Typ určuje štruktúru správy, existujú 4 typy správ
			      \begin{itemize}
				      \large{\item \texttt{ClientInitT}}
				            \large{\item \texttt{ServerInitT}}
				            \large{\item \texttt{ContentT}}
				            \large{\item \texttt{ErrorT}}
			      \end{itemize}
			\item Existuje podpora pre dalších 251 typov
		\end{itemize}}
\end{frame}

\begin{frame}[c]
	\frametitle{Protokol sieťovej komunikácie}
	\begin{figure}
		\includegraphics[width=.8\textwidth]{presentation_pictures/comm.pdf}
	\end{figure}
\end{frame}

\begin{frame}[c]
	\frametitle{Protokol sieťovej komunikácie}
	\large{\begin{itemize}
			\item Klient pošle svôj verejný klúč a podpíše správu
			\item Server overí podpis, zašifruje súkromný klúč a podpíše správu
			\item Klient overý podpis, dešifruje súkromný klúč
			\item Začne šifrovaná komunikácia pomocou súkromného klúča
		\end{itemize}
	}
\end{frame}

\begin{frame}[c]
	\frametitle{Modularita algoritmov}
	\large{\begin{itemize}
			\item Algoritmy sa volia pomocou konfiguračných súborov
			\item Konfiguračné súbory naďalej obsahujú verejný a súkromný klúč
			\item Algoritmy sa musia rovnať pri vzájomnej komunikácií
			\item Možno dodať ďalšie algoritmy z minimálnou zmenou v kóde
		\end{itemize}}
\end{frame}

\begin{frame}[c]
	% bez nadpisu snímku
	\frametitle{\mbox{ }}
	\begin{center}
		{\Huge Ukážka aplikácie v TUI}
	\end{center}
\end{frame}

\begin{frame}[c]
	\frametitle{Bežný chod aplikácie}
	\begin{figure}
		\includegraphics[width=\textwidth]{presentation_pictures/normal_comm.pdf}
	\end{figure}
\end{frame}

\begin{frame}[c]
	\frametitle{Chyba v konfigurácií}
	\begin{figure}
		\includegraphics[width=.9\textwidth]{presentation_pictures/comm_err.pdf}
	\end{figure}
\end{frame}

\begin{frame}[c]
	\frametitle{Chyba v konfigurácií}
	\begin{figure}[htbp]
		\centering
		\includegraphics[width=\textwidth]{presentation_pictures/client_err.pdf}
	\end{figure}
\end{frame}

\begin{frame}[c]
	\frametitle{Ďalšie vlastnosti aplikácie}
	\large{\begin{itemize}
			\item Ukladanie log správ do súboru počas TUI
			\item Integrácia z UNIX textovým rozhraním (presmerovanie \texttt{stdout}, \texttt{stdin})
			\item Ochrana proti útokom s opakovaním správ pomocou cookie
			\item Integrácia analýzi sieťového prenosu vo wiresharku pomocou \texttt{lua} skriptu
			\item Podpora Windows aj UNIX systémoch
		\end{itemize}}
\end{frame}

% podekovani
\begin{frame}[c]
	% bez nadpisu snímku
	\frametitle{\mbox{ }}
	\begin{center}
		{\Huge Ďakujem za pozornosť!}
	\end{center}
\end{frame}

% otázky oponenta
\frame{
	\frametitle{Otázky oponenta}
	\emph{Otázka 1}\\[2ex]
	%
	Odpoveď 1
}

\end{document}
