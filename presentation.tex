% Soubory musí být v kódování, které je nastaveno v příkazu \usepackage[...]{inputenc}

\documentclass[%        Základní nastavení
  %draft,    				  % Testovací překlad
  14pt,       				% Velikost základního písma je 12 bodů
	t,                  % obsah slajdů bude vždy začínat od shora (nebude vertikálně centrovaný)
	aspectratio=1610,   % poměr stran bude 16:10 (všechny projektory v učebnách na Technické 12 Brno),
	                    % další volby jsou 43, 149, 169, 54, 32.
	unicode,						% Záložky a informace budou v kódování unicode
]{beamer}				    	% Dokument třídy 'zpráva', vhodná pro sazbu závěrečných prací s kapitolami
%\usepackage{etex}

\usepackage[utf8]		  % Kódování zdrojových souborů je v UTF-8
	{inputenc}					% Balíček pro nastavení kódování zdrojových souborů

\usepackage{graphicx} % Balíček 'graphicx' pro vkládání obrázků
											% Nutné pro vložení logotypů školy a fakulty

\usepackage[          % Balíček 'acronym' pro sazby zkratek a symbolů
	nohyperlinks				% Nebudou tvořeny hypertextové odkazy do seznamu zkratek
]{acronym}
											% Nutné pro použití prostředí 'acronym' balíčku 'thesis'

%% Balíček hyperref je volán třídou beamer automaticky, proto není třeba následujícího kódu:
%\usepackage[
%	breaklinks=true,		% Hypertextové odkazy mohou obsahovat zalomení řádku
%	hypertexnames=false % Názvy hypertextových odkazů budou tvořeny
%											% nezávisle na názvech TeXu
%]{hyperref}						% Balíček 'hyperref' pro sazbu hypertextových odkazů
%											% Nutné pro použití příkazu 'nastavenipdf' balíčku 'thesis'

\usepackage{cmap} 		% Balíček cmap zajišťuje, že PDF vytvořené `pdflatexem' je
											% plně "prohledávatelné" a "kopírovatelné"

%\usepackage{upgreek}	% Balíček pro sazbu stojatých řeckých písmem
											%% např. stojaté pí: \uppi
											%% např. stojaté mí: \upmu (použitelné třeba v mikrometrech)
											%% pozor, grafická nekompatibilita s fonty typu Computer Modern!

%\usepackage{amsmath} %balíček pro sabu náročnější matematiky

\usepackage{booktabs} % Balíček, který umožňuje v tabulce používat
                      % příkazy \toprule, \midrule, \bottomrule


%%%%%%%%%%%%%%%%%%%%%%%%%%%%%%%%%%%%%%%%%%%%%%%%%%%%%%%%%%%%%%%%%
%%%%%%      Definice informací o dokumentu             %%%%%%%%%%
%%%%%%%%%%%%%%%%%%%%%%%%%%%%%%%%%%%%%%%%%%%%%%%%%%%%%%%%%%%%%%%%%

% Variable fields such as your name, title of the thesis atc. are set in this file.
% The file is SHARED between the thesis text and the presentation --- no need to set anything twice.

\usepackage[
  english,			% English study program
%%% Choose one of the thesis types
	semestral,		%	semestral project (abstracts, declaration and acknowledgements are excluded) (default)
  %bachelor,			%	Bachelor's thesis
  %master,			% Master's thesis
  %treatise,			% Treatise on doctoral thesis
  %doctoral,				% Doctoral thesis
%%% Choose one of the options:
%  left,				% Equations and captions will be aligned to the left
  center,			% Equations and captions will be centered (default)
]{thesis}   % Package for typesetting theses


%%% First and last name of the thesis author, following the scheme
% [titles in front]{FirstName}{LastName}[titles after]
% If the author does not have a title in front/after, just delete the entire string '[...]'
\author[Bc.]{Samuel}{Kopecký}

%%% Brno University of Technology identification number of author (BUT ID)
\butid{211799}

%%% First and last name of the advisor
% [titles in front]{FirstName}{LastName}[titles after]
% If the author does not have a title in front/after, just delete the entire string '[...]'
% [titles in front]{FirstName}{LastName}[titles after]
\advisor[Ing.]{David}{Smékal}[]

%%% First and last name of the opponent
% [titles in front]{FirstName}{LastName}[titles after]
% Makes use only in the presentation for defense;
% in case you do not want the opponent to be shown on the title slide, just comment out the command
% The opponent is not shown in case of the semestral thesis (no opponent exists actually at that time)
% In the case of a doctoral thesis, two to three oponents are typically involved. In such a case, if you want to have them on the title slide, please go to the definition of "VUT title page" in the thesis.sty file, uncomment and adapt their names.
\opponent[TODO]{TODO}{TODO}[TODO]

%%% Thesis title
%  In the case of a very long thesis title, it may happen that it does not fit
%  into the slot in the footbar of the slides. You may use the command
%  \def\insertshorttitle{Shortened th.\ title}
%  where the shortened version of the title appears as the parameter.
%  If you do not want to shorten the title, you will have to redefine how the slide footbar
%  is generated, see: https://bit.ly/3EJTp5A
\title{Modular communication based on post-quantum cryptohraphy}

%%% Study program/specialization
\specialization{Information Security} %Teleinformatika

%%% Department
%\department{Department of Control and Instrumentation}  %Ústav automatizace a měřicí techniky
%\department{Department of Biomedical Engineering}       %Ústav biomedicínského inženýrství
%\department{Department of Electrical Power Engineering} %Ústav elektroenergetiky
%\department{Department of Electrical and Electronic Technology}   %Ústav elektrotechnologie
%\department{Department of Physics}                      %Ústav fyziky
%\department{Department of Foreign Languages}            %Ústav jazyků
%\department{Department of Mathematics}                  %Ústav matematiky
%\department{Department of Microelectronics}             %Ústav mikroelektroniky
%\department{Department of Radio Electronics}            %Ústav radioelektroniky
%\department{Department of Theoretical and Experimental Electrical Engineering}  %Ústav teoretické a experimentální elektrotechniky
\department{Department of Telecommunications}           %Ústav telekomunikací
%\department{Department of Power Electrical and Electronic Engineering}   %Ústav výkonové elektrotechniky a elektroniky

%%% Faculty
%\faculty{Faculty of Architecture}   %Fakulta architektury
\faculty{Faculty of Electrical Engineering and~Communication}   %Fakulta elektrotechniky a~komunikačních technologií
%\faculty{Faculty of Chemistry}   %Fakulta chemická
%\faculty{Faculty of Information Technology}   %Fakulta informačních technologií
%\faculty{Faculty of Business and Management}   $Fakulta podnikatelská
%\faculty{Faculty of Civil Engineering}   %Fakulta stavební
%\faculty{Faculty of Mechanical Engineering}   %Fakulta strojního inženýrství
%\faculty{Faculty of Fine Arts}   %Fakulta výtvarných umění
%
%Logotype selection (in square brackets short logo, in curly brackets full logo):
\facultylogo[logo/FEEC_abbreviation_color_PANTONE_EN]{logo/UTKO_color_PANTONE_EN}


%%% Graduate year (typically the calendar year of the defense)
\graduateyear{2023}
%%% Academic year (typically the year of solution of the thesis in the format n/n+1)
\academicyear{2022/23}
% Date of the defense (makes use only in the presentation slides)
\date{TODO} 

%%% Place of the defense
\city{Brno}

%%% Abstract
\abstract{%
Abstract in English.
}

%%% Keywords
\keywrds{%
Keywords in English
}

%%% Thanks and acknowledgement
\acknowledgement{%
TODO
}%      % v tomto souboru doplňte údaje o sobě, o názvu práce...
                       % (tento soubor je sdílený s textem práce)

%%%%%%%%%%%%%%%%%%%%%%%%%%%%%%%%%%%%%%%%%%%%%%%%%%%%%%%%%%%%%%%%%%%%%%%%

%%%%%%%%%%%%%%%%%%%%%%%%%%%%%%%%%%%%%%%%%%%%%%%%%%%%%%%%%%%%%%%%%%%%%%%%
%%%%%%     Nastavení polí ve Vlastnostech dokumentu PDF      %%%%%%%%%%%
%%%%%%%%%%%%%%%%%%%%%%%%%%%%%%%%%%%%%%%%%%%%%%%%%%%%%%%%%%%%%%%%%%%%%%%%
%% Při vloženém balíčku 'hyperref' lze použít příkaz '\pdfsettings'
\pdfsettings
%  Nastavení polí je možné provést také ručně příkazem:
%\hypersetup{
%  pdftitle={Název studentské práce},    	% Pole 'Document Title'
%  pdfauthor={Autor studenstké práce},   	% Pole 'Author'
%  pdfsubject={Typ práce}, 						  	% Pole 'Subject'
%  pdfkeywords={Klíčová slova}           	% Pole 'Keywords'
%}
\hypersetup{pdfpagemode=FullScreen}       % otevření rovnou v režimu celé obrazovky
%%%%%%%%%%%%%%%%%%%%%%%%%%%%%%%%%%%%%%%%%%%%%%%%%%%%%%%%%%%%%%%%%%%%%%%

\usetheme{VUT} 				% barvy a rozložení prezentace odpovídající VUT FEKT
% alternativně lze použít jiná berevná témata, ale bez záruky. Například:
%\usetheme{Darmstadt} \usecolortheme{default2}
\logoheader					% vytvoření zkráceného loga VUT FEKT v hlavičce slajdu, nechte odkomentované


\begin{document}

% v případě zakomentování následujícího se zobrazí v pravém dolním rohu slajdů klikatelné navigační symboly
\disablenavigationsymbols

% titulní snímek, vysazen bez horních, dolních a postranních lišt (volba plain),
% není tak vysazen ani nadpis snímku
\maketitle


\begin{frame}[c]
	\frametitle{Ciele práce}
	\large{\begin{itemize}
			\item Výber post kvantových algoritmov
			\item Implementácia post kvantového algoritmu pre výmenu kľúčov/šifrovanie
			\item Implementácia post kvantového algoritmu pre digitálny podpis
			\item Výber implementačného programovacieho jazyka
		\end{itemize}}

\end{frame}

% \begin{frame}[c]
% 	\frametitle{Post kvantová kryptografia}
% 	\large{\begin{itemize}
% 			\item Shorov algoritmus prelomí IFP a DLP
% 			\item Potreba kvantového počítača z miliónmy qubitov
% 			\item Kvantový počítač od IBM\,--\,127 qubitov
% 			\item Potreba algoritmov asymetcikej kryptografie odolné voči kvantovým počítačom
% 		\end{itemize}}

% \end{frame}

\begin{frame}[c]
	\frametitle{Programovací jazyk Go}
	\large{\begin{itemize}
			\item Relatívne nový (2009) kompilovaný jazyk od spoločnosti Google
			\item Podobný jazyku C
			\item Zameraný na vytváranie rýchlych sieťových aplikácií
		\end{itemize}}

	\begin{figure}[htbp]
		\centering
		\includegraphics[width=\textwidth]{pictures/compiler_sk.pdf}
	\end{figure}
\end{frame}

\begin{frame}[c]

	\frametitle{Kyber a Dilithium}
	\large{
		\begin{itemize}
			\item Post kvantové algoritmy založené na mriežkach
			\item Zvolené v treťom kole NIST štandardizácie (Júl 2022)
			\item 3 módy bezpečnosti
			\item Formálny dôkaz o bezpečnosti problému LWE
		\end{itemize}
		\vspace{3ex}
		\begin{columns}[T]
			\begin{column}{0.5\textwidth}
				CRYSTALS-Kyber
				\begin{itemize}
					\item Výmena kľúčov
					\item Najrýchlejší algoritmus
				\end{itemize}
			\end{column}
			\begin{column}{0.5\textwidth}
				CRYSTALS-Dilithium
				\begin{itemize}
					\item Digitálny podpis
					\item Najmenší verejný kľúč
				\end{itemize}
			\end{column}
		\end{columns}
	}
\end{frame}

\begin{frame}[c]
	\frametitle{Proces implementácie algoritmu Kyber}
	\large{
		\begin{itemize}
			\item Implementácia vychádza z oficiálnej špecifikácie
			\item NTT implementácia
			\item Implementácia pomocných funkcií
			\item Zostavenie algoritmu z častí
			\item Implementácia transformácie polynómov na bajty
		\end{itemize}
	}
\end{frame}

\begin{frame}[c]
	\frametitle{Všeobecná násobenie v NTT tvare}
	\large{
		Pre polynómy $f$ a $g$
		\begin{align*}
			f & = f_0 + f_1x + f_2x^2 + \dots + f_{254}x^{254} + f_{255}x^{255}, \\
			g & = g_0 + g_1x + g_2x^2 + \dots + g_{254}x^{254} + g_{255}x^{255},
		\end{align*}
		je násobenie pre $\hat{h}=\hat{f}*\hat{g}$ v tvare NTT
		\begin{align*}
			\hat{h} & = \hat{f}_0\hat{g}_0 + \hat{f}_1\hat{g}_1x + \hat{f}_2\hat{g}_2x^2 + \dots + \hat{f}_{254}\hat{g}_{254}x^{254} + \hat{f}_{255}\hat{g}_{255}x^{255}.
		\end{align*}
	}
\end{frame}

\begin{frame}[c]
	\frametitle{Proces implementácie algoritmu Dilithium}
	\large{
		\begin{itemize}
			\item Implementácia vychádza z oficiálnej špecifikácie
			\item Využitie NTT z Kyber implementácie s malou úpravou
			\item Implementácia pomocných funkcií
			\item Zostavenie algoritmu z častí
			\item Zložitejšia implementácia transformácie polynómov na bajty
		\end{itemize}
	}
\end{frame}

\begin{frame}[c]
	\frametitle{Transformácia na bajty}
	\begin{figure}[htbp]
		\centering
		\includegraphics[width=\textwidth]{pictures/bit_packing.pdf}
	\end{figure}
\end{frame}


\begin{frame}[c]
	\frametitle{Výsledky implementácie}
	Implementácia využíva jedinú externú knižnicu pre Hash funkcie
	\begin{table}[htbp]
		\centering
		\begin{tabular}{|l|c|c|}
  \hline
  Implementácia & 5000 iterácií & Priemer         \\
  \hline
  táto práca    & 21.051\,s       & 4210\,\textmu s \\
  kyber-k2so     & 1.297\,s        & 259\,\textmu s  \\
  circl          & 0.307\,s        & 61\,\textmu s   \\
  \hline
\end{tabular}

		\caption{Kyber}
	\end{table}
	\begin{table}[htbp]
		\centering
		\begin{tabular}{|l|c|c|}
  \hline
  Implementácia & 5000 iterácií & Priemer         \\
  \hline
  táto práca   & 19.868\,s     & 3973\,\textmu s \\
  circl         & 1.693\,s      & 338\,\textmu s  \\
  \hline
\end{tabular}
		\caption{Dilithium}
	\end{table}
\end{frame}

% \begin{frame}[c]
% 	\frametitle{Navádzajúca post kvantová komunikácia}
% 	\begin{figure}[htbp]
% 		\centering
% 		\includegraphics[width=0.7\textwidth]{pictures/future.pdf}
% 	\end{figure}
% \end{frame}

\begin{frame}[c]
	\frametitle{Záver}
	\large{
		Dosiahnuté ciele
		\begin{itemize}
			\item Implementácia algoritmov pre výmenu kľúčov, šifrovanie a digitálny podpis
		\end{itemize}
		Ciele v nasledujúcej práci
		\begin{itemize}
			\item Vylepšenie výsledkov implementácie Kyber a Dilithium
			\item Implementácia všetkých bezpečnostných úrovní
			\item Implementovanie post kvantovej komunikácie
			\item Rozsiahlejšie výkonové testovanie implementácií
		\end{itemize}
	}

\end{frame}

% podekovani
\begin{frame}[c]
	% bez nadpisu snímku
	\frametitle{\mbox{ }}
	\begin{center}
		{\Huge Ďakujem za pozornosť!}
	\end{center}
\end{frame}

% otázky oponenta
% \frame{
% \frametitle{Otázky oponenta}
% 	\emph{Jaká je souvislost Vašeho vzorce (1.2) s~Maxwellovými rovnicemi v~integrálním tvaru?}\\[2ex]
% 	%
% 	Již staří Římané\,\dots
% }

\end{document}
