% V tomto souboru se nastavují téměř veškeré informace, proměnné mezi studenty:
% jméno, název práce, pohlaví atd.
% Tento soubor je SDÍLENÝ mezi textem práce a prezentací k obhajobě -- netřeba něco nastavovat na dvou místech.

\usepackage[
  %%% Z následujících voleb jazyka lze použít pouze jednu
  %czech-english,		% originální jazyk je čeština, překlad je anglicky (výchozí)
  %english-czech,	% originální jazyk je angličtina, překlad je česky
  %slovak-english,	% originální jazyk je slovenština, překlad je anglicky
  english-slovak,	% originální jazyk je angličtina, překlad je slovensky
  %
  %%% Z následujících voleb typu práce lze použít pouze jednu
  %semestral,		  % semestrální práce (nesází se abstrakty, prohlášení, poděkování) (výchozí)
  %bachelor,			%	bakalářská práce
  master,			  % diplomová práce
  %treatise,			% pojednání o disertační práci
  %doctoral,			% disertační práce
  %
  %%% Z následujících voleb zarovnání objetů lze použít pouze jednu
  %  left,				  % rovnice a popisky plovoucích objektů budou zarovnány vlevo
  center,			    % rovnice a popisky plovoucích objektů budou zarovnány na střed (vychozi)
  %
]{thesis}   % Balíček pro sazbu studentských prací


%%% Jméno a příjmení autora ve tvaru
%  [tituly před jménem]{Křestní}{Příjmení}[tituly za jménem]
% Pokud osoba nemá titul před/za jménem, smažte celý řetězec '[...]'
\author[Bc.]{Samuel}{Kopecký}

%%% Identifikační číslo autora (VUT ID)
\butid{211799}

%%% Pohlaví autora/autorky
% (nepoužije se ve variantě english-czech ani english-slovak)
% Číselná hodnota: 1...žena, 0...muž
\gender{0}

%%% Jméno a příjmení vedoucího/školitele včetně titulů
%  [tituly před jménem]{Křestní}{Příjmení}[tituly za jménem]
% Pokud osoba nemá titul před/za jménem, smažte celý řetězec '[...]'
\advisor[Ing.]{David}{Smékal}

%%% Jméno a příjmení oponenta včetně titulů
%  [tituly před jménem]{Křestní}{Příjmení}[tituly za jménem]
% Pokud osoba nemá titul před/za jménem, smažte celý řetězec '[...]'
% Nastavení oponenta se uplatní pouze v prezentaci k obhajobě;
% v případě, že nechcete, aby se na titulním snímku prezentace zobrazoval oponent, pouze příkaz zakomentujte;
% u obhajoby semestrální práce se oponent nezobrazuje (jelikož neexistuje)
% U dizertační práce jsou typicky dva až tři oponenti. Pokud je chcete mít na titulním slajdu, prosím ručně odkomentujte a upravte jejich jména v definici "VUT title page" v souboru thesis.sty.
\opponent[TODO]{TODO}{TODO}[TODO]

%%% Název práce
%  Parametr ve složených závorkách {} je název v originálním jazyce,
%  parametr v hranatých závorkách [] je překlad (podle toho jaký je originální jazyk).
%  V případě, že název Vaší práce je dlouhý a nevleze se celý do zápatí prezentace, použijte příkaz
%  \def\insertshorttitle{Zkác.\ náz.\ práce}
%  kde jako parametr vyplníte zkrácený název. Pokud nechcete zkracovat název, budete muset předefinovat,
%  jak se vytváří patička slidu. Viz odkaz: https://bit.ly/3EJTp5A
\title[Modular communication based on post-quantum cryptohraphy]{Modulární komunikace postavená na postkvantové kryptografii}

%%% Označení oboru studia
%  Parametr ve složených závorkách {} je název oboru v originálním jazyce,
%  parametr v hranatých závorkách [] je překlad
\specialization[Information security]{Informačná Bezpečnosť}

%%% Označení ústavu
%  Parametr ve složených závorkách {} je název ústavu v originálním jazyce,
%  parametr v hranatých závorkách [] je překlad
%\department[Department of Control and Instrumentation]{Ústav automatizace a měřicí techniky}
%\department[Department of Biomedical Engineering]{Ústav biomedicínského inženýrství}
%\department[Department of Electrical Power Engineering]{Ústav elektroenergetiky}
%\department[Department of Electrical and Electronic Technology]{Ústav elektrotechnologie}
%\department[Department of Physics]{Ústav fyziky}
%\department[Department of Foreign Languages]{Ústav jazyků}
%\department[Department of Mathematics]{Ústav matematiky}
%\department[Department of Microelectronics]{Ústav mikroelektroniky}
%\department[Department of Radio Electronics]{Ústav radioelektroniky}
%\department[Department of Theoretical and Experimental Electrical Engineering]{Ústav teoretické a experimentální elektrotechniky}
\department[Department of Telecommunications]{Ústav telekomunikací}
%\department[Department of Power Electrical and Electronic Engineering]{Ústav výkonové elektrotechniky a elektroniky}

%%% Označení fakulty
%  Parametr ve složených závorkách {} je název fakulty v originálním jazyce,
%  parametr v hranatých závorkách [] je překlad
%\faculty[Faculty of Architecture]{Fakulta architektury}
\faculty[Faculty of Electrical Engineering and~Communication]{Fakulta elektrotechniky a~komunikačních technologií}
%\faculty[Faculty of Chemistry]{Fakulta chemická}
%\faculty[Faculty of Information Technology]{Fakulta informačních technologií}
%\faculty[Faculty of Business and Management]{Fakulta podnikatelská}
%\faculty[Faculty of Civil Engineering]{Fakulta stavební}
%\faculty[Faculty of Mechanical Engineering]{Fakulta strojního inženýrství}
%\faculty[Faculty of Fine Arts]{Fakulta výtvarných umění}
%
%Nastavení logotypu (v hranatych zavorkach zkracene logo, ve slozenych plne):
\facultylogo[logo/FEKT_zkratka_barevne_PANTONE_CZ]{logo/UTKO_color_PANTONE_CZ}

%%% Rok odevzdání práce
\graduateyear{2023}
%%% Akademický rok odevzdání práce
\academicyear{2022/23}

%%% Datum obhajoby (uplatní se pouze v prezentaci k obhajobě)
\date{7.\,6.\,2023}

%%% Místo obhajoby
% Na titulních stránkách bude automaticky vysázeno VELKÝMI písmeny (pokud tyto stránky sází šablona)
\city{Brno}

%%% Abstrakt
\abstract[%
  Súčasné kryptografické primitíva, ktoré sú popísané na začiatku tejto práce budú prelomené budúcimi kvantovými počítačmi. Táto práca popisuje proces lámania spolu so základným popisom kvantovej mechaniky, ktorá je kľúčom k funkčným kvantovým počítačom. Táto práca tiež predstavuje dostupné riešenia, ako je postkvantová kryptografia. Konkrétnejšie je predstavená code-based, hash-based a lattice-based kryptografia. Najpodrobnejšie je opísaná lattice-based kryptografia a sú predstavené špecifické NIST štandardizované algoritmy\,--\,Kyber a Dilithium. Spolu s teoretickým popisom je poskytnutá implementácia pre obidve algoritmy a porovnanie s existujúcimi implementáciami v programovacom jazyku Go. Praktické využitie týchto algoritmov je realizované modulárnou kvantovo odolnou komunikačnou aplikáciou. Je schopná posielať ľubovoľné dáta cez kvantovo odolný zabezpečený kanál a je dobre prispôsobený univerzálnemu textovému rozhraniu UNIX systmémoch. Viac špecificky, aplikácia je schopná vymieňať súbory medzi dvoma používateľmi a tiež vytvárať terminálové používateľské rozhranie, s ktorým môžu používatelia komunikovať. Protokol, ktorý je zodpovedný za vytvorenie zabezpečeného kanála, je dobre definovaný v posledných kapitolách tejto práce. Modularita aplikácií tiež umožňuje používateľom odstrániť a/alebo pridať akýkoľvek mechanizmus výmeny kľúčov alebo digitálny podpis, ktoré sú zodpovedné za vytvorenie zabezpečeného kanála s veľmi malými zmenami kódu a dobrou integráciou do existujúcich komponentov aplikácie.
]{%
  Current cryptography primitives, which are described at the begging of this thesis will~be broken by future quantum computers. How they will be broken is described by this thesis along with a very basic description of quantum mechanics which are key to functional quantum computers. This thesis also introduces available solutions like post-quantum cryptography. More specifically code-based, hash-based and lattice-based cryptography. Lattice-based cryptography is described in most detail and specific NIST standardized algorithms are introduced\,--\,Kyber and Dilithium. Along with the theoretical description, an implementation is provided for both of the algorithms and a comparison to existing implementations in the programing language Go. Practical utilization of these algorithms is realized with a modular quantum-resistant communication application. It~can send arbitrary data through a quantum-resistant secured channel and is well adjusted to~the~UNIX universal text interface. Particularly it is able to exchange files between two users and also create a Terminal User Interface with which the users can communicate. The underlying protocol that is responsible for creating the secure channel is well defined in the latter chapters of this thesis. The modularity of the applications also allows users to remove or/and add any Key Exchange Mechanism or Digital signature which are responsible for the creation of the secure channel with very few code changes and good integration to the existing components of the application.
}

%%% Klíčová slova
\keywrds[%
  post-kvantová kryptografia, programovací jazyk Go, sieťová komunikácia, terminálové uživaťelské rozhranie, kryptografia založená na mriežkach
]{%
  post-quantum cryptography, programming language Go, network communication, terminal user interface, lattice-based cryptography
}

%%% Poděkování
\acknowledgement{%
  I would like to thank the supervisor of the thesis, Ing.~David Smékal, for his professional guidance, consultations, patience, suggestions and ideas.
}%
