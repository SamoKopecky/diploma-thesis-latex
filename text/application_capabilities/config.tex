In order to choose what post-quantum algorithms for key encapsulation and digital signature the program will use, a configuration file is used. It is in a json format and there are 4 keys that can be configured
\begin{itemize}
  \itemtt{kem\_alg}key encapsulation mechanism used for key exchange between clients,
  \itemtt{sign\_alg}digital signature algorithm for creating/verifing signatures during the initial communication,
  \itemtt{public\_key}base64 encoded string of the public key,
  \itemtt{private\_key}also a base64 encoded string of the private key.
\end{itemize}
Similarly as with the \texttt{app} command the \texttt{config} command servers for the purpose of logically separating subcommands. In this case its the \texttt{list} and \texttt{gen} commands. The latter is used for generating a new configuration file with all of the keys filled out and the former for listing available algorithm names. These algorithms can be seen in appendix \ref{ch:available_algs}.
\object[0.605]{obr}{pictures/config.pdf}{Configuartion file keys}{img:config_keys}

While generating a configuration file the first two keys for the choice of algorithms can be configured using flags \texttt{-\--kem} and \texttt{-\--sign}. By supplying a string parameters to these flags the algorithm name is chosen. The selection of algorithms is generated from the source code which defines the algorithms and their functions. To get a better understanding of how these algorithms are defined in code, refer to section \ref{sec:algs_modularity}. The other two keys for the public/private key are generated depending on the choice of algorithms. If no algorithms are selected, default algorithms are used. Finally two configuration files are generated one for the client and one for the server. The clients configuration file contains the public key of the server and his own private key. The similar can be said for the server where he has the clients public key and his own private key (illustrated by figure \ref{img:config_keys}). This is necessary for the initial communication establishment phase of the underlying protocol. See chapter \ref{ch:network_com_sec} for more information.
