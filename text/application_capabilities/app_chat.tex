An interactive mode where users can send text messages asynchronously. By running the application using this command, the TUI (\acl{TUI}) is utilized. A TUI is very similar to a GUI (\acl{GUI}) where the only difference is that it doesn't require any desktop environment and only requires a terminal interface to work. The created TUI is very easy to control, the user types in a message into the text field and can send that message to the other peer by pressing enter. The sent message will then appear in the window above together with any messages that were sent to him. To quit the application the user can press either escape or ctrl+c. The TUI is also responsive to any window size changes since it is implemented via the bubbletea\footnote{\url{https://github.com/charmbracelet/bubbletea}} TUI framework. Examples of light and dark terminal themes can be found in Appendix \ref{ch:TUI_example}.

The underlying application protocol works differently depending on whether the user is a client or a server. The role of the user can be chosen by using either the \texttt{-c} flag to act as a user or the \texttt{-l} flag to act as the server.
