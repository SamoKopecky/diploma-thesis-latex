Many CLI applications programed in Go use an framework to parse input arguments as options or commands to alter the usage of an application. This framework is called Cobra\footnote{\url{https://cobra.dev}}. Go also has a built in library for parsing input arguments but it can't compare to Cobra, which contains a lot more useful tools. The following section in this chapter will explain the differences between Cobra and the built in argument parser and how the Go application in this thesis utilizes it.

The application is mainly split into four groups of capabilities where each group gets a section characterizing it. The parts are
\begin{itemize}
  \item communication,
  \item benchmarking,
  \item configuration,
  \item other commands.
\end{itemize}
\noindent Commands usage examples for some of these capabilities can be found in appendix \ref{ch:go_instructions}. Additionally this chapter describes the process adding of new algorithms and extending it. The implementation results of Kyber and Dilithium are also present in a separate section of this chapter.
