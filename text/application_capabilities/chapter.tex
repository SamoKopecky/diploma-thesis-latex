Many CLI applications programmed in Go use a framework to parse input arguments as options or commands to alter the usage of an application. This framework is called Cobra\footnote{\url{https://cobra.dev}}. Go also has a built-in library for parsing input arguments but it isn't as future rich as Cobra, which contains a lot more useful tools. The following section in this chapter will explain how the Go application in this thesis utilizes it.

The application is mainly split into four groups of capabilities where each group gets a section characterizing it. The parts are
\begin{itemize}
  \item communication,
  \item configuration,
  \item other commands,
  \item benchmarking.
\end{itemize}
\noindent Command usage examples for some of these capabilities can be found in Appendix \ref{ch:go_instructions}. Additionally, this chapter describes the process adding of new algorithms and extending them. The implementation results of Kyber and Dilithium are also presented in a separate section of this chapter.
