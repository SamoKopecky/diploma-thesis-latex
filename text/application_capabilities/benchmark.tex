A speed performance benchmark can be run on all of the post-quantum algorithms\,--\,digital signatures and key encapsulation methods\,--\,that are added to the application using the modularity system. Benchmarking is done using the standard Go library. This approach has many advantages over building a custom benchmarking tool. It is very precise meaning it only measures tha actual time it took to run a function.
It also integrates very well with another tool called \texttt{benchstat}\footnote{\url{https://pkg.go.dev/golang.org/x/perf/cmd/benchstat}}. It is used for summarizing the resulting benchmark and providing a variance to the resulting measurements. The list of benchmarks can be found in \ref{ch:available_benchs}. A guide on how to benchmark can be found in appendix \ref{sec:how_to_bench}.

By default the go benchmark command runs a function or a piece of code as long as the complete execution time is 1 second or more. Then it divides the total amount of time it took to execute by the amount of iterations ran. The result is the time it took to run one iteration. However this setting can be changed by appending a parameter. For example in order to run a function for 2 seconds, the parameter
\begin{itemize}
  \item \texttt{-benchtime=2s}
\end{itemize}
has to be provided. One benchmark of a function can also be run multiple times by using the
\begin{itemize}
  \item \texttt{-count=x}
\end{itemize}
parameter and supplying \texttt{x} repetitions. If the number of repetitions is six or more \texttt{benchstat} can provide a variance to the resulting time of one iteration. In order to choose what benchmarks should be ran, the parameter
\begin{itemize}
  \item \texttt{-bench}
\end{itemize}
can be used provided with a regex expression. There is a possibility where the benchmarking/testing processes exists before finishing. This is because the default timeout is 10 minutes. If a benchmark will take longer then 10 seconds the timeout needs to be overwritten by supplying the
\begin{itemize}
  \item \texttt{-timeout=24h}
\end{itemize}
command. In this example the timeout is set to 24 hours.
