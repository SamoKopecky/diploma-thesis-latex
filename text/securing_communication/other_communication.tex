The other two message types are relatively simple. The first one is \texttt{ErrorT} message which is used for sending error messages. For example in a situation where the clients configured algorithms are not the same as the servers algorithms, the server sends an error message to the client, stating that there has been a misconfiguration. This message type can be seen in figure \ref{img:errort_pdf}.
\object[0.6]{obr}{pictures/errort.pdf}{Error message}{img:errort_pdf}

The last defined message type is the plain data message \texttt{ContentT} (see figure \ref{img:contentt_pdf}). This message is used for sending any data that the users want to exchange be it a file or just plain text messages. It can be deduced from the header fields that the maximum payload size is 65 535 or $2^{16}-1$. This means that in the chat mode, users can exchange a message up to 65 535 bytes long. While in the file sending mode the file that is being read is read by chunks, these chunks are then sent so the maximum file size is theoretically infinite.

\object[0.6]{obr}{pictures/contentt.pdf}{Content message}{img:contentt_pdf}
