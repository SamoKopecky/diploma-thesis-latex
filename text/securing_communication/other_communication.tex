The other two message types are relatively simple. The first one is \texttt{ErrorT} message which is used for sending error messages. For example in a situation where the client's configured algorithms are not the same as the server's algorithms, the server sends an error message to the client, stating that there has been a misconfiguration. This message type can be seen in Figure \ref{img:errort_pdf}.
\object[0.6]{obr}{pictures/errort.pdf}{Error message}{img:errort_pdf}

The last defined message type is the plain data message \texttt{ContentT} (see Figure \ref{img:contentt_pdf}). This message is used for sending any data that the users want to exchange be it a file or just plain text messages. It can be deduced from the header fields that the maximum payload size is 65~523 or $2^{16}-1-12$ bytes long. The header length is 16 bits long which dictates the maximum payload size to 65535. The nonce used in this message type is 12 bytes long so that has to be deducted from the maximum payload size. This means that in the chat mode, users can exchange a message up to 65~523 bytes long. While in the file sending mode, the file that is being read is read by chunks, these chunks are then sent, so the maximum file size is theoretically infinite. As mentioned beforehand the nonce is also a part of this type of message and is randomly generated for every new message. It is used as the initialization vector for the GCM operation mode while using AES.

\object[0.6]{obr}{pictures/contentt.pdf}{Content message}{img:contentt_pdf}
