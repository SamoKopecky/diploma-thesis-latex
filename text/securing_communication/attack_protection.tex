The most obvious approach to attacking this application would be to eavesdrop on the communication channel and just capture and read the traffic. This is not possible since as mention beforehand all traffic apart from the header and initialization messages is encrypted using symmetric cipher AES-GCM-256.

However more approaches could be used for breaking the security of this application. One of those approaches is the impersonation of an communicating entity, in this case one of the peers. The use of public key cryptography, more specifically digital signatures prevent the use of this approach. An attacker impersonating the client can't create a valid client init message since he can't create a digital signature that would be verifiable by the server. Only the real client can crete that signature since he has the private key. Similarly if the attacker would impersonate the server, he can't create a valid server init message since only the client holds the valid private key to the shared public key.

Another type of attack is the MitM (\acl{MITM}) attack. It happens when at attacker manages to create two simultaneous connections with the user and the server. The peers think they are communicating with each other, in truth the attacker is just forwarding their messages back and forth while being able to read them. In order to prevent this type of attack, the digital signature is again used together with the preconfigured keys. If the attack wants to create a separate connection with the server after receiving the client init message from the client, the only thing he can do is forward it trough to the server, which is harmless. He can't edit it or change the value for example of the public encryption key since he can't create a new signature. He is also unable to impersonate the server since he doesn't know his private key. Also even if the attack theoretically managed to create a connection during the same microsecond (in this instance the Epoch time is measured in microseconds) as the legitimate client he would also be rejected since the timestamp has to always be newer and not the same.
