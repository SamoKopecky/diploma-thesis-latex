As any modern L7 protocol, this messages in this protocol consist of a header the actual data that is being sent also referred to as a payload. The data can either be used for initializing the connection or just for sending arbitrary data. The header is very simple and consists of only two fields the length of the payload and the type of the payload. As can be seen in figure \ref{img:header}, the type is an 8\,bit integer and the length is a 16\,bit integer.

\object[0.6]{obr}{pictures/header.pdf}{Protocol header}{img:header}
\noindent From the possible 255 types only 4 types of payloads are implemented:
\begin{itemize}
  \itemtt{ClientInitT}initialization message for the client side,
  \itemtt{ServerInitT}initialization message for the server side,
  \itemtt{ContentT}generic data payload type,
  \itemtt{ErrorT}error messages.
\end{itemize}
The initialization together with the client and server init types is detailed in the following subsection \ref{subsec:init_phase}. The rest of the types are mentioned in the subsection \ref{subsec:other_phase}
