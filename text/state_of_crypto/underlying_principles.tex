One of the underlying principles used in asymmetric cryptography is the \textbf{integer factorization problem} (\acs{IFP}). This problem utilizes the idea that factorizing a big integer $n$ (in size from 2048 to 3072 bits) is impossible to compute on todays computers in polynomial time \cite{Paar2010}. But producing $n$ from two prime numbers $p$ and $q$ is trivial and fast. IFP together with modular arithmetic create the RSA (\acl{RSA}) cipher that is one of the most used ciphers used today for creating digital signatures. The private and public keys would be derived from the integer $n$.

The other principle that is used heavily in todays asymmetric cryptography is the \textbf{discrete logarithm problem} (\acs{DLP}). It heavily relies on the use of modular arithmetic and cyclic groups in which there are a finite amount of integer values. This is possible because it uses the the modulo operation together with any other operation to stay inside this cyclic group. In this group it is very easy (in polynomial time) to compute $\beta$ with
\begin{equation}
  \alpha^x\equiv\beta\,\mathrm{mod}\,p
\end{equation}
while knowing the values for $x$ and $\alpha$ but very hard (in exponential time on todays computers) to compute x using this formula
\begin{equation}
  x\equiv\mathrm{log}_\alpha\beta\,\mathrm{mod}\,p
\end{equation}
with the knowledge of only $\alpha$ and $\beta$, where $p$ is a prime number \cite{Paar2010}. DSA (\acl{DSA}) utilizes this problem for creating digital signatures. An alternative exists to this problem that uses elliptic curves instead of cyclic groups. This is because the DLP equivalent in elliptic curves is more secure while using the same size for parameters such as $x$ which is the private key \cite{Ristic2014}. This property is allows the use of smaller keys while staying on the same level of security.