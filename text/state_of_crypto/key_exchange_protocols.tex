As mentioned in section \ref{sec:symmetric_enc}, secret keys firstly need to be shared between the communicating entities before any encryption can begin. That is where a KEP (\acl{KEP}) is utilized. One category of a KEP is the KEM (\acl{KEM}). As they are a subset of asymmetric cryptography, many algorithms or ciphers used for asymmetric encryption can be converted to a KEM, for example RSA \cite{Ristic2014}. How the key exchange works is illustrated by figure \ref{img:asym_crypto}, but instead of \textit{Alice} encrypting and sending documents, she sends \textit{Bob} a randomly generated key.

Another alternative of a KEP utilizes a dedicated key exchange method, such as the Diffie-Hellman protocol. It also works on the principle of having a public, private key pair like RSA, but each entity exchanges their public key with the other entity and then they calculate the secret key from the knowledge of their own private key and of the opposite entities public key. Instead of relying on IFP it relies on the DLP (see subsection \ref{subsec:underlying_principles}). This brings an advantage, because the DH method can be then upgraded to \acl{ECDH} (ECDH), which is a faster method for exchanging keys then plain DH or RSA \cite{Ristic2014}.
