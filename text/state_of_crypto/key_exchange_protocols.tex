As mentioned in section \ref{sec:symmetric_enc}, secret keys first need to be shared between the communicating entities before any encryption can begin. That is where a KEP (\acl{KEP}) is utilized. A subcategory of a KEP is a KEM (\acl{KEM}). As they are a subset of asymmetric cryptography, many algorithms or ciphers used for asymmetric encryption can be converted to a KEM, for example, RSA \cite{Ristic2014}. How the key exchange works is illustrated by figure \ref{img:asym_crypto}, but instead of~Alice encrypting and sending documents, she sends Bob a randomly generated encrypted key.

Another alternative of a KEP utilizes a dedicated key exchange method, such as the Diffie-Hellman protocol. It also works on the principle of having a public, private key pair like RSA, but each entity exchanges its public key with the other entity and then they calculate the shared secret key from the knowledge of their private key and the opposite entity's public key. Instead of relying on IFP, it relies on the~DLP (see subsection \ref{subsec:underlying_principles}). This brings an advantage because the DH method can be then upgraded to \acl{ECDH} (ECDH), which is a faster method for exchanging keys than plain DH \cite{Ristic2014}.
