The other important type of cryptography is asymmetric cryptography also called public key cryptography. Compared to symmetric (see section \ref{sec:symmetric_enc}), asymmetric algorithms are most often slower, require bigger-sized keys and use two keys instead of one key. One of the keys that can be shared is the public key, the~second key that has to be kept secret is called the private key. Depending on the use of these keys, asymmetric cryptography can be used in two ways -- as an \textbf{encryption} cipher or as a \textbf{digital signature} scheme.

The principle of an encryption algorithm can be seen in figure \ref{img:asym_crypto}. Alice can encrypt a document using Bob's public key since the public key is shared with everyone and because Bob wants anyone to be able to send him encrypted messages. After receiving the encrypted document Bob can decrypt it with his private key since he is the only one that owns it. \cite{Smart2004}

\object[0.9]{obr}{pictures/asym_crypto.pdf}{Symplified asymmetric encryption cipher}{img:asym_crypto}

Digital signature schemes serve as a tool to verify the origin of data. The following process is illustrated by figure \ref{img:asym_sign}. If Bob wants anyone who receives his document to be able to verify that he was the one who created it, he signs the~document with his private key. Everyone else including Bob can check whether the~document came from Bob by verifying the signature with his pubic key. If the verification succeeds the verifier can be sure that Bob generated the signature because he is the only that posses the private key that generated the signature. \cite{Paar2010}

In practice, Bob would be signing a~hash of the document instead of the document itself, and would also send an unsigned document. Bob would be comparing a~hash of the document with the verified signature. This is because as mentioned before asymmetric algorithms are slow relative to symmetric algorithms and signing all of the data is unnecessary when signing the hash of some data servers the same purpose. Hash functions are described in the previous section of this chapter (section \ref{sec:hash_functions}).

\object[0.9]{obr}{pictures/asym_sign.pdf}{Simplified digital signature scheme}{img:asym_sign}
