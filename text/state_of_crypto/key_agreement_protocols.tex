As mentioned in section \ref{sec:symmetric_enc}, secret keys firstly need to be shared between the communicating entities before any encryption can begin. That is where Key encapsulating mechanisms are utilized. As they are a subset of asymmetric cryptography, many algorithms or ciphers used for asymmetric encryption can be converted to a KEM, for example RSA \cite{Ristic2014}. How the key exchange works is illustrated by figure \ref{img:asym_crypto}, but instead of \textit{Alice} encrypting and sending documents, she sends \textit{Bob} a randomly generated key.

Another alternative of crating KEMs is with the use of dedicated key exchange methods, such as the Diffie-Hellman method. It also works on the principle of having a public, private key pair like RSA but instead of relaying on the IFP it relies on the DLP (see subsection \ref{subsec:underlying_principles}). This brings an advantage, because the DH method can be then upgraded to \acl{ECDH} (ECDH), which is a faster method for exchanging keys \cite{Ristic2014}.