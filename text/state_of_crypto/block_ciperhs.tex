Block ciphers operate on blocks of data and use padding to handle situations when a message can't be perfectly split into blocks. The same key used for each block. Symmetric block ciphers also use different modes of operation to add additional context to individual blocks from previous blocks. This process is important for the security of symmetric block ciphers, because a block cipher without any mode of operation or a ECB (\acl{ECB}) mode generates the same output from the same input. This means an attacker could delete or add any block in an encrypted message without the receivers knowledge. Some examples of secure mode of operations for block ciphers are:
\begin{itemize}
  \item \textbf{OFB} (\acl{OFB}),
  \item \textbf{CFB} (\acl{CFB}),
  \item \textbf{GCM} (\acl{GCM}).
\end{itemize}

These ciphers are based on a substitution-permutation network (figure \ref{img:sbox_pbox}), which consists of two layers, a substitution layer and a permutation layer as the name implies. 

The substitution layer introduces \textit{confusion} to the data. Confusion creates a relation between the key and the ciphertext, where one changed bit in the key will generate a change for many bits int the ciphertext. In practice a substitution layer just substitutes a byte with the help of a substitution table which is predefined and used for every operation (see figure \ref{img:sbox_pbox}). On the other hand the permutation layer introduces \textit{diffusion}, which means that a changed bit in the plaintext will dissipate into more changed bits in the ciphertext. As can be seen in the figure \ref{img:sbox_pbox} an example of a permutation layer functions by scrambling the order of bytes randomly. \cite{Paar2010}\cite{Shannon1949}

Of course in practice a cipher needs a lot more then just a simple substitution-permutation network. Good examples of block ciphers that use this principe are AES (\acl{AES}) and DES (\acl{DES}). DES is no longer deemed secure and should not be used \cite{Barker2017}. AES on the other hand is still considered secure even to attacks from quantum computers if longer keys are used\cite{Chen2016}.

\object[0.8]{obr}{pictures/sbox-pbox.pdf}{Substitution-permutation network}{img:sbox_pbox}
