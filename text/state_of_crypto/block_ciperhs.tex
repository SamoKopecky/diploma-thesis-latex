Block ciphers operate on blocks of data and use padding to handle situations when a message can't be perfectly split into blocks. The same key is used for each block. Symmetric block ciphers can also use different modes of operation to add additional context to individual blocks from previous blocks. This process is important for the security of symmetric block ciphers because a block cipher without any mode of operation or an~ECB (\acl{ECB}) mode generates the same output from the same input. This means an attacker could delete or add any block in an encrypted message without the receiver's knowledge. Some examples of a secure mode of operations for block ciphers are \cite{Paar2010}
\begin{itemize}
  \item \textbf{OFB} (\acl{OFB}),
  \item \textbf{CFB} (\acl{CFB}),
  \item \textbf{GCM} (\acl{GCM}).
\end{itemize}

These ciphers are based on a substitution-permutation network (figure \ref{img:sbox_pbox}), which consists of two layers, a substitution layer and a permutation layer as the~name implies.

The substitution layer introduces \textit{confusion} to the data. Confusion creates a correlation between the key and the ciphertext, where one changed bit in the key will generate a change for many bits in the ciphertext. In practice, a substitution layer just substitutes one byte with the help of a substitution table. This table is predefined and used for every operation (see figure \ref{img:sbox_pbox}). On the other hand, the permutation layer introduces \textit{diffusion}, which means that a changed bit in the plaintext will dissipate into more changed bits in the ciphertext. In other words, it functions by scrambling the order of bytes randomly. An example can be seen in figure \ref{img:sbox_pbox} of a permutation layer. \cite{Paar2010}\cite{Shannon1949}

Of course, in practice, a cipher needs a lot more than just a simple substitution-permutation network. Good examples of block ciphers that use this principle are AES (\acl{AES}) and DES (\acl{DES}). DES is no longer deemed secure and should not be used \cite{Barker2017}. AES on the other hand is still considered secure even to attacks from quantum computers if longer keys are used \cite{Chen2016}.

\object[0.6]{obr}{pictures/sbox-pbox.pdf}{Substitution-permutation network}{img:sbox_pbox}
