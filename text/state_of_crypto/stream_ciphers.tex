Unlike block ciphers, stream ciphers encrypt one bit at a time instead of blocks. The main principle behind stream ciphers is the bit operation XOR and a PRNG (\acl{PRNG}). The key is randomly generated by the PRNG function. Then the message is XORed with the generated key. The XOR operation can also be rewritten as mod $2$ and thus the encryption process can be described as
\begin{equation}
  c = E(p)\equiv p + k\ \mathrm{mod}\,2
\end{equation}
\noindent and the decryption process
\begin{equation}
  p = D(c)\equiv c + k\ \mathrm{mod}\,2
\end{equation}
\noindent for $c$ as the ciphertext, $p$ as the plaintext, $k$ as the secret key $E$ and $D$ as the encryption and decryption functions respectively \cite{Paar2010}.

Examples of stream ciphers include RC4, Salsa20 or ChaCha20. It is no longer recommended to use the RC4 cipher. Salsa20 is a newer stream cipher and is~considered to be resident even against quantum computers. \cite{Bernstein149}\cite{Ristic2014}
