Another algorithm from the group of lattice-based cryptography is the CRYSTALS-Dilithium signature scheme. It was also standardized during the 3rd round of the NIST standardization process on post-quantum cryptography. It is based on the Fiat-Shamir paradigm which means a prover can convince a verifier of the fact that they hold a private key without actually revealing it. Similarly, Kyber is also based on the MLWE problem. Dilithium also has a binding property that allows a signature to be linked with a unique public key and a message. When it comes to the security of Dilithium, it is proven that a signature is unforgeable by classical and quantum computers. NIST mentioned in their report on the 3rd round of standardization that Dilithium has a strong security basis and along with Falcon is one of the most efficient signature algorithms. \cite{Alagic2022}
