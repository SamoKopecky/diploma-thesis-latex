The verification process for Dilithium starts by unpacking the required variables from the public key and the signature. \rmhat{A} is generated the same way as in the signature generation algorithm, it is then multiplied by the vector $z$ from the public key and called $Az$. The public parameter $t_1$ is at first scaled by $2^d$ where $d$ is defined by Table \ref{tab:dil_sec_levels}. This is making up for the lost bits during the key generation. It is then scaled again by $c$ which is parsed from the signature. The result of subtracting $ct_1$ from $Az$ is used as the first input for the \texttt{UseHint} function. The second input parameter is $h$ from the signature. Lastly, the hash of the signed message and $t_1$ is hashed together with the result of \texttt{UseHint}. If the result of this operation is equal to $c$ the process succeeded and the signature is verified, if they don't equal the verification failed. See Figure \ref{img:dil_verify} for the summary of the process.
\object[0.5]{obr}{pictures/dil_enc.pdf}{Dilithium signature creation}{img:dil_sign}
\clearpage
\object[0.5]{obr}{pictures/dil_dec.pdf}{Dilithium signature verification}{img:dil_verify}
