As with Kyber, Dilithium is implemented using only the standard go libraries and one external library \texttt{crypto}\footnote{\url{https://pkg.go.dev/golang.org/x/crypto}} that contains implementations for SHAKE-128 and SHAKE-256 hash functions. Dilithium can be implemented in two ways, the first one is by using a bigger public key. This implementation of Dilithium is also simpler overall. The other option is implementing a more complex algorithm that has a~smaller public key by a factor of more than half. For this thesis, a more complex implementation was chosen. How this alternative differs from the simpler one will be explained in subsection \ref{subsec:dil_reducing_pub_key}. The algorithms as a whole is described in figure \ref{img:dil_all}. Analogous to the Kyber algorithm figure each square represents a mathematical structure or a program variable, where the structures are mostly represent a vector of polynomials. The small letters at the beginning of arrows denote the number of polynomials that the resulting structure consists of.

The process of signing in Dilithium follows a well defined order as with many other digital signatures. Firstly the public/private keys are generated and the private key is used in~the signing process. This is key is not shared and kept secret by the signer. The result of a signing process is a signature which can be verified by anyone who owns the~related public key. Since the public key is shared, it is not kept secret by the signer. The following sections will explain all of these steps in~more detail. For an even more detailed description of Dilithium, check the algorithm implementation in practical part of this thesis.

\object{tab}{tables/dil_security_levels}{Dilithium security levels \cite{y0VQZiTmHEg2xvPn}}{tab:dil_sec_levels}

Table \ref{tab:dil_sec_levels} displays the individual parameters for each of the Dilithium parameter sets. The implementation in this thesis contains all Dilithium security modes. When a parameter is relevant to the process being explained it will be mentioned and explained in that scenario instead of all the parameters explained in this section.
