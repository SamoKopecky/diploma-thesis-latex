As mentioned the functions described in Section \ref{sec:dil_bit_man} are used in the complex implementation of dilithium to reduce the size of the public while providing the same security. Some of the bits from the public key are transferred to the private key. This makes the private key bigger but also makes the public key smaller. However now when verifying a signature, the calculations are not precise enough when only using the cut of the public key. However, during the signing process, hints are made using the \texttt{MakeHints} function. This is possible because some bits of the public key are stored in the private key. This is what allows the verification calculation to be just precise enough to correctly decide whether the verification is correct. \cite{y0VQZiTmHEg2xvPn}
