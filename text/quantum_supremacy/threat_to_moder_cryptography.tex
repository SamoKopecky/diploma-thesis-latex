The future impact of quantum computers on classical cryptography can be seen in~table \ref{tab:impact_of_quantum}. ECC (\acl{ECC}) algorithms and RSA aren't safe from quantum computers with a sufficient amount of qubits using Shor's algorithm. Currently, it is estimated that the required amount of qubits for Shor's algorithm to be efficient enough is in the tens of millions \cite{Bernstein149}\cite{Mosca2015} physical qubits. IBM managed to create a 433 physical qubit quantum processor in 2022 so humanity is not yet at the point where everyday internet communication using public key cryptography can be broken using quantum computers \cite{0MBNdFRCTLK35MFY}. However, the threat is still there since traffic encrypted today using modern cryptography can still be broken later using quantum computers.

IBM has projected in their new roadmap to a practical quantum computer, that by 2025 they expect to have working quantum computers that contains between 4~158 physical qubits \cite{Gambetta2021}. If this grows exponentially, a~replacement for the current public key algorithms needs to be found. Each of the new candidates will be discussed in detail in chapter \ref{ch:pq_crypto}.

Symmetric cryptography and hash functions on the other hand are much more resistant to quantum computers. For the current ciphers and algorithm to be quantum-resistant only the symmetric key size and digest size for hash functions needs to increase. For example, in the case of AES-128, it is sufficient enough to switch to AES-256 where the performance hit is negligible \cite{Bernstein149}.

\object{tab}{tables/impact_of_quantum_computers}{Impact of quantum computers on classical cryptography\cite{Bernstein149}\cite{Chen2016}}{tab:impact_of_quantum}
