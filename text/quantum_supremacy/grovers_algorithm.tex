Symmetric cryptography and hash functions can also be broken by another algorithm named Grover's algorithm. It is categorized as a search algorithm so instead of solving any any mathematical problem, it just searches through all the possible options. Given a set of bits $\{0, 1\}^n$ where $n$ is the size of the set a classical computer will search for a specific binary string of length $n$ in $O(2^n)$ time. Grover's algorithm can search for the same binary string in $O(2^{n/2})$ time. \cite{Yanofsky2008}

Symmetric cryptography keys are also a binary string created from a set of bits size $n$, where Grover's algorithm can be used to find a key by trying all possible values. Similarly hash functions output a binary string also from a set of bits. Grover's algorithm can be used to try to generate all the possible hash values inside a quantum computer and when the the it finds a match it can retrospectively find the output that generated the hash value \cite{Pretson2022}. Since the Grover's algorithm is not as efficient as Shor's algorithm in find solutions that break ciphers or algorithms, key/hash value sizes can be increased to prevent theses kind of attacks \cite{Chen2016}.
