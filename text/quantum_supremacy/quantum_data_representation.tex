Quantum computers as the name implies, are based on the special properties of quantum mechanics. One of these many properties that quantum computers work with is the superposition of states. At very small sizes (sizes of individual particles) objects can be in such a state. Unlike ordinary objects of ordinary sizes, they can exist in more than one location at the same time. This phenomenon only occurs if the object is not being seen (is not being measured). However, this means whenever an object is measured in such a state the position of the object collapses into a single point in space. \cite{Yanofsky2008}

\object[0.4]{obr}{pictures/qubit.pdf}{Representation of a qubit}{img:qubit}

This unique property is what allows data to be represented in a quantum computer. In a classical computer, data is represented using bits. These only have 2 distinct values 0 or 1. Quantum computers don't work with bits but quantum bits or qubits in short. A qubit is represented by two pairs of complex numbers $c_0$~and~$c_1$. Complex numbers can be converted into real numbers $p_0$ and $p_1$
\begin{equation}
  \begin{aligned}
    p_0 &= \lvert c_0 \rvert^2, \\
    p_1 &= \lvert c_1 \rvert^2,
  \end{aligned}
\end{equation}
in this form they represent the probability of a~qubit collapsing (after a~measurement) into discrete values 0 or 1 and becoming a~classical bit \cite{Yanofsky2008}. This concept is also illustrated in figure \ref{img:qubit} where the pointing arrows illustrate the qubit being measured.

Using complex numbers qubits can also be represented using the bra-ket notation
\begin{equation}
  \lvert\psi\rangle=c_0|0\rangle + c_1|1\rangle
\end{equation}
where $\psi$ represents the particle in a~superposition of all possible states (0 and 1). A quantum computer can hold more than one qubit in a state of superposition. Unlike classical computers which always have one state, quantum computers can use the property of superposition and be in many states at the same time. This means it can evaluate a function for many values at the same time, which leads to great parallelism of quantum algorithms. A quantum algorithm doesn't work like a~classical algorithm. It starts with a single position for all the qubits in~the input. During the algorithm, the qubits are manipulated in their superposition state. When the algorithm finishes the state is then measured. At no point during the algorithm, the state can be measured, because then the superposition would be lost due to the qubits collapsing into a single state. \cite{Yanofsky2008}
