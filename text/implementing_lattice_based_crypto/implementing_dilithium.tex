As with Kyber, Dilithium was implemented using only the standard go libraries and one external library \cite{00fV2cvg7Z6H2tS3} that contains implementations for \texttt{SHAKE-128} and \texttt{SHAKE-256}. Dilithium can be implemented in two ways, the first is with using a bigger public key and more complexity. The other option is implementing a more complex algorithm which has a smaller public key by a factor of more then half. For this thesis the more complex implementation was chosen. The algorithm is explained at a very basic level in figure\footnote{This figure illustrates the simpler version where the whole public key is used for the sake of clarity in the diagram.} \ref{img:dil_all}. The small letter at the beginning of arrows denoted the size of the object. At first the public/secret keys are generated then the secret key is used in the signing process. A signature is generated which then can be verified by anyone who owns the related public key. Following sections explain all of these steps in more detail with the help code snippets. As before with Kyber these code snippets can't be actually compiled and is only a go styled pseudocode. For the compilable implementation check the practical part of this thesis.

Table \ref{tab:dil_sec_levels} shows the individual parameters for each of the Dilithium parameter sets. The implementation in this thesis contains only Dilithium 2. When a parameter is relevant for the process being explained it will be mentioned and explained in that scenario instead of all the parameters being explained in this section,

\object{tab}{tables/dil_security_levels}{Dilithium security levels \cite{y0VQZiTmHEg2xvPn}}{tab:dil_sec_levels}
