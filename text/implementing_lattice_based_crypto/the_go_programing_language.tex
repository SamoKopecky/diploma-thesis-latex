The first iteration of the Go programming language was created at Google. It is an open-source programing language and has many similarities with C which means it is a compiled language and a statically typed language \cite{Donovan2016}. In a statically typed language a variable has to have a type assigned to it before the compilation process. As can be seen in figure \ref{img:compiler} a compiler translates the source program into an executable, which can be ran multiple times without the need to compile again \cite{Aho2006}. This makes Go faster then most of interpreted languages like Python, since an interpreter needs to translate the source code every time it has to run.

\object[0.9]{obr}{pictures/compiler.pdf}{Compiler}{img:compiler}

However unlike C it has a garbage collector, that means it has an automatic memory management \cite{Donovan2016}. In C a programmer has to manage memory on its own, allocated and free it by using functions. Go and its garbage collector take care of allocating and freeing memory which makes it a lot less error prone when it comes to memory management. Together with good overall performance Go was also designed to make high performance network applications thats why it was chosen for this thesis.