This appendix contains the necessary information for building the go program, running it and then instructions on how to run the provided tests.

\section{How to build}
Go supports most of the well-known operating systems such as Linux, Windows and Mac. This application supports the Linux and Windows operating systems. It was not tested on a Mac system but theoretically, it should also work on it. First of all download the latest version of the go binary from this link
\begin{itemize}
  \item \url{https://go.dev/doc/install}.
\end{itemize}
Once go is installed the binary executable file for this thesis can be built by running
\begin{itemize}
  \item \texttt{go build -v -o pqcom}
\end{itemize}
or for Windows
\begin{itemize}
  \item \texttt{go build -v -o pqcom.exe}
\end{itemize}
inside a command line interface. The command also has to be run inside the root directory of the project. The output of this command should yield a~file named \texttt{pqcom} or \texttt{pqcom.exe} respectively. From now on only the Linux binary will be used for examples but the same examples also apply to the Windows executable.
\section{How to run}
Once the binary is built it can be run just like any other binary. Run with
\begin{itemize}
  \item \texttt{./pqcom}
\end{itemize}
in the command line interface. To run the Windows executable run
\begin{itemize}
  \item \texttt{pqcom.exe}
\end{itemize}
Refer to chapter \ref{ch:app_capab} for application capabilities or run
\begin{itemize}
  \item \texttt{./pqcom -\--help}
\end{itemize}
to see what commands are available. Additionally every command has has a short description of what it does on top of available commands/flags. For example to see information about the app command run
\begin{itemize}
  \item \texttt{./pqcom app -\--help}
\end{itemize}

\section{Examples}
Before commands that create quantum-resistant connections can be used a configuration file is needed. To generate configuration files for both of the peers run
\begin{itemize}
  \item \texttt{./pqcom config gen}
\end{itemize}
To use different algorithms for the configuration check the section \ref{sec:cmd_config}. To use a created configuration file use one of the three options defined in section \ref{sec:cmd_app}.

In order to receive all log messages the option \texttt{-\--log debug} can be appended to the commands to enable logging at the debug level. By default, the logging will be output to the \texttt{stderr} channel of the terminal. However, while the application uses the \texttt{chat} mode, log messages are instead saved to a file located in the log directory listed in appendix \ref{ch:directories_app}.

\subsection{Chat mode}
Now that configuration files are generated, the application can be used to create connections, for example using the \texttt{chat} command. Firstly the server has to start listening. By default command
\begin{itemize}
  \item \texttt{./pqcom app chat -l -\--config pqcom\_server.json}
\end{itemize}
will start listening on port \texttt{4040} and on address \texttt{127.0.0.1}. Now a client has to connect by running
\begin{itemize}
  \item \texttt{./pqcom app chat -c -\--config pqcom\_client.json}
\end{itemize}
where the default remote port is again \texttt{4040} with the IP address \texttt{127.0.0.1}. If everything was done correctly a TUI should open where users can send messages to each other.

\subsection{File exchange mode}
To run the application in send/receive mode for file exchange one of the peers first has to run
\begin{itemize}
  \item \texttt{./pqcom app receive -\--config pqcom\_server.json > output.txt}
\end{itemize}
and the other peer has to run
\begin{itemize}
  \item \texttt{cat input.txt | ./pqcom app send -\--config pqcom\_client.json}
\end{itemize}
When the commands are run in the correct order, the client will send the contents of the \texttt{input.txt} file to the server who will save the file in a file named \texttt{output.txt}. However these two command will only work on Unix-like systems. To create a file exchange between two Windows systems run the following commands. The server side has to run
\begin{itemize}
  \item \texttt{./pqcom app receive -\--config pqcom\_server.json -\--dir .}
\end{itemize}
and the client needs to run
\begin{itemize}
  \item \texttt{./pqcom app send -\--config pqcom\_server.json -\--file input.txt}
\end{itemize}
The received file will be written in the given directory and named \texttt{pqcom\_temp\_XXXX} where the 5 X characaters are replaced by random characters.

\section{How to test}
Tests that check whether the implementations are working correctly can be run by entering
\begin{itemize}
  \item \texttt{go test -v ./...}
\end{itemize}
into the command line interface in the root directory of the project. These tests check whether the implementations of the Kyber and Dilithium are functional.

\section{How to benchmark}
\label{sec:how_to_bench}
In order to launch benchmarks for all available post-quantum algorithms, meaning all KEM and digital signature algorithms provided via the modularity system, run this command
\begin{itemize}
  \item \texttt{go test -bench="Kem\$|Signature\$" -run=\^{}\# ./... | tee out.txt}
\end{itemize}
In order to use \texttt{benchstat} for a better analysis of the results, first it has to be installed inside the go binary folder by running
\begin{itemize}
  \item \texttt{go install golang.org/x/perf/cmd/benchstat@latest}
\end{itemize}
Then a new benchmark file has to be generated so that \texttt{benchstat} can produce valid statistical data. It can be generated by running
\begin{itemize}
  \item \texttt{go test -count=8 -bench="Kem\$|Signature\$" -run=\^{}\# ./... | \textbackslash\\tee out.txt}
\end{itemize}
After installing \texttt{benchstat} it can be used on the generated file by running
\begin{itemize}
  \item \texttt{benchstat out.txt}
\end{itemize}
In order to only test the algorithms implemented in this thesis run
\begin{itemize}
  \item \texttt{go test -bench="PqCom\textbackslash w+All" -run=\^{}\# ./... | tee out.txt}
\end{itemize}
Similarly as before the count be increased so that the results can be analyzed using \texttt{benchstat}.
