% For the list of references, use one of the two options below

%%%%%%%%%%%%%%%%%%%%%%%%%%%%%%%%%%%%%%%%%%%%%%%%%%%%%%%%%%%%%%%%%%%%%%%%%
%1) References created directly, by hand, using the 'thebibliography' environment

\begin{thebibliography}{99}
  \bibitem{Bernstein149}
  BERNSTEIN, Daniel J. a Tanja LANGE. Post-quantum cryptography. \textit{Nature} [online]. 2017, 14.9, \textbf{2017}(549), 188-194 [cit. 2022-10-09]. Dostupné z: doi:\url{https://doi.org/10.1038/nature23461}
  \bibitem{Smart2004}
  SMART, Nigel. \textit{Cryptography: An Introduction} [online]. 3rd. ed. McGraw-Hill College, 2004 [cit. 2020-10-18]. ISBN 978-0077099879. Dostupné z: \url{https://www.cs.umd.edu/~waa/414-F11/IntroToCrypto.pdf}
  \bibitem{Ristic2014}
  RISTIĆ, Ivan. \textit{Bulletproof SSL and TLS: Understanding and Deploying SSL/TLS and PKI to Secure Servers and Web Applications Ivan Ristic}. 6 Acantha Court, Montpelier Road, London W5 2QP, United Kingdom: Feisty Duck, 2014. ISBN 978-1-907117-04-6.
  \bibitem{Paar2010}
  PAAR, Christof a Jan PELZL. \textit{Understanding Cryptography: A Textbook for Students and Practitioners}. 2nd edition. London New York: Springer Heidelberg Dordrecht, 2010, 382~s. ISBN 978-3-642-44649-8.
  \bibitem{Shannon1949}
  SHANNON, Claude E. Communication Theory of Secrecy Systems. \textit{Bell System Technical Journal}. 1949, \textbf{4}(28), 656-715.
  \bibitem{Barker2017}
  BARKER, Elaine a Nicky MOUHA. \textit{Recommendation for the Triple Data Encryption Algorithm (TDEA) Block Cipher}. 2nd ed. NIST Pubs, 2017, 32~s. Dostupné také z: \url{https://nvlpubs.nist.gov/nistpubs/SpecialPublications/NIST.SP.800-67r2.pdf}
  \bibitem{Chen2016}
  CHEN, Lily, Stephen JORDAN, Yi-Kai LIU, Dustin MOODY, Rene PERALTA, Ray PERLNER a Daniel SMITH-TONE. NISTIR 8105. \textit{Report on Post-Quantum Cryptography}. NIST, 2016, 15~s. Dostupné také z: \url{http://dx.doi.org/10.6028/NIST.IR.8105}
  % \bibitem{Nir2015}
  % NIR, Y. a A. LANGLEY. \textit{ChaCha20 and Poly1305 for IETF Protocols}. Internet Engineering Task Force, 2015, 45~s. Dostupné také z: \url{https://tools.ietf.org/html/rfc7539}
  \bibitem{rd1wUlxEgliEynii}
  FIPS PUB 180-4. \textit{Secure Hash Standard}. Gaithersburg, USA: NIST, 2015, 36~s. Dostupné také z: \url{http://dx.doi.org/10.6028/NIST.FIPS.180-4}
  \bibitem{1Od8f4TuMxetfmHu}
  FIPS PUB 202. \textit{SHA-3 standard: permutation-based hash and extendable output functions}. Gaithersburg, USA: NIST, 2015, 37~s. Dostupné také z: \url{http://dx.doi.org/10.6028/NIST.FIPS.202}
  \bibitem{Bernstein2009}
  BERNSTEIN, Daniel J., Johannes BUCHMANN a Erik DAHMEN. \textit{Post-Quantum Cryptography}. Berlin: Springer-Verlag, 2009, 248~s. ISBN 978-3-540-88701-0.
  \bibitem{Yanofsky2008}
  YANOFSKY, Noson S. a Mirco A. MANNUCCI. \textit{Qunatum computing for cumputer scientists}. New York: Cambridge university press, 2008, 402~s. ISBN 978-0-521-87996-5.
  \bibitem{McMahon2008}
  MCMAHON, David. \textit{Quantum computing explained}. New Jersey: John Wiley \& Sons, 2008, 351~s. ISBN 978-0-470-09699-4. 
  \bibitem{Pittenger2000}
  PITTENGER, Arthur O. \textit{An Introduction to Quantum Computing Algorithms}. Boston: Birkhäuser, 2000, 150~s. ISBN ISBN 0-8176-4127-0.
  \bibitem{Pretson2022}
  PRETSON, Richard. \textit{Applying Grover-s Algorithm to Hash Functions: A Software Perspective}. Bedford: The MITRE Corporation, 2022. Dostupné také z: \url{https://arxiv.org/pdf/2202.10982.pdf}
  \bibitem{Mosca2015}
  MOSCA, Michele. \textit{Cybersecurity in an era with quantum computers: will we be ready?}. Ontario: Cryptology ePrint Archive, 2015, 4~s. Dostupné také z: \url{https://eprint.iacr.org/2015/1075}
  \bibitem{0MBNdFRCTLK35MFY}
  IBM Unveils Breakthrough 127-Qubit Quantum Processor. IBM. \textit{IBM Newsroom} [online]. 2021 [cit. 2022-10-26]. Dostupné z: \url{https://newsroom.ibm.com/2021-11-16-IBM-Unveils-Breakthrough-127-Qubit-Quantum-Processor}
  \bibitem{Gambetta2021}
  GAMBETTA, Jay. Expanding the IBM Quantum roadmap to anticipate the future of quantum-centric supercomputing. IBM. \textit{IBM research} [online]. 2021 [cit. 2022-10-26]. Dostupné z: \url{https://research.ibm.com/blog/ibm-quantum-roadmap-2025}
  \bibitem{Alagic2022}
  ALAGIC, Gorjan, Daniel APON, David COOPER, et al. NIST IR 8413-UPD1. \textit{Status Report on the Third Round of the NIST Post-Quantum Cryptography Standardization Process}. NIST, 2022, 102~s. Dostupné také z: \url{https://doi.org/10.6028/NIST.IR.8413-upd1}
  \bibitem{Ajati1996}
  AJATI, Miklós. Generating hard instances of lattice problems. \textit{Proceedings of the twenty-eighth annual ACM symposium on Theory of Computing} [online]. 1996, 99-108 [cit. 2022-11-01]. Dostupné z: doi:\url{https://doi.org/10.1145/237814.237838}
  \bibitem{Goldreich1997}
  GOLDREICH, Obed, Shafi GOLDWASSER a Shai HALEVI. Public-key cryptosystems from lattice reduction problems. \textit{Advances in Cryptology --- CRYPTO '97} [online]. Heidelberg: Springer Berlin Heidelberg, 1997, 112--131 [cit. 2022-11-02]. Dostupné z: doi:10.1007/BFb0052231
  \bibitem{Regev2005}
  REGEV, Oded. On lattices, learning with errors, random linear codes, and cryptography. \textit{Proceedings of the thirty-seventh annual ACM symposium on Theory of computing} [online]. 2005, \textbf{5}, 84-93 [cit. 2022-11-03]. Dostupné z: doi:10.1145/1060590.1060603
  \bibitem{Grimes2020}
  GRIMES, Roger A. \textit{Cryptography Apocalypse: Preparing for the Day When Quantum Computing Breaks Today-s Crypto}. Canada: John Wiley \& Sons, 2020, 263~s. ISBN 978-1-119-61819-5.
  \bibitem{Forouzan2010}
  FOROUZAN, Behrouz. \textit{TCP/IP Protocol Suite}. 4th edition. Raghothaman Srinivasan: McGraw-Hill, 2010, 1029~s. ISBN 978-0-07-337604-2.
  \bibitem{Donovan2016}
  DONOVAN, Alan A. A. a Brian W. KERNIGHAN. \textit{The Go Programming Language}. 2nd edition. Crawfordsville, Indiana: Addison-Wesley, 2016, 399~s. ISBN 978-0-13-419044-0.
  \bibitem{Aho2006}
  AHO, Alfred V., Monica S. LAM, Ravi SETHI a Jeffrey D. ULLMAN. \textit{Compilers Principles, Techniques, \& Tools}. 2nd edition. Addison-Wesley, 2006, 1035~s. ISBN 0-321-48681-1.
  \bibitem{00fV2cvg7Z6H2tS3}
  Crypto module. \textit{Go Packages} [online]. [cit. 2022-11-17]. Dostupné z: \url{https://pkg.go.dev/golang.org/x/crypto}
  \bibitem{YbbuGxVPF0GGTxfN}
  AVANZI, Roberto, Joppe BOS, Léo DUCAS, et al. \textit{CRYSTALS-Kyber: Algorithm Specifications And Supporting Documentation}. 3rd ed. 43~s. Dostupné také z: \url{https://pq-crystals.org/kyber/data/kyber-specification-round3-20210804.pdf}
  \bibitem{Liang2021}
  LIANG, Zhichuang, Shiyu SHEN, Yuantao SHI, Dongni SUN, Chongxuan ZHANG, Guoyun ZHANG, Yunlei ZHAO a Zhixiang ZHAO. Number Theoretic Transform: Generalization, Optimization, Concrete Analysis and Applications. \textit{Information Security and Cryptology} [online]. Springer, Cham, 2021, \textbf{12612}, 415-432 [cit. 2022-11-17]. Dostupné z: doi:10.1007/978-3-030-71852-7\_28
  \bibitem{y0VQZiTmHEg2xvPn}
  BAI, Shi, Léo DUCAS, Eike KILTZ, Tancrede LEPOINT, Vaidm LYUBASHEVKSY, Peter SCHWABE, Gregor SEILER a Damien STEHLÉ. \textit{CRYSTALS-Dilithium: Algorithm Specifications and Supporting Documentation}. 3rd ed. 38~s. Dostupné také z: \url{https://pq-crystals.org/dilithium/data/dilithium-specification-round3-20210208.pdf}

  
\end{thebibliography}

%%%%%%%%%%%%%%%%%%%%%%%%%%%%%%%%%%%%%%%%%%%%%%%%%%%%%%%%%%%%%%%%%%%%%%%%%
%%2) References generated using BibTeX (automatically from a database of sources)
%% Selection of the citation 'style'
% \bibliographystyle{unsrt}
%% Selection of the database file containing the sources
% \bibliography{text/literatura}
%
%% The following command is only to show a list of references using BibTeX.
%% It makes listed all the items from literatura.bib, although they are not cited in the text.
% \nocite{*}