To achieve a small traffic overhead, this application uses the smallest header possible. The header only contains the data size and type. For a more detailed description of the header see chapter \ref{ch:network_com_sec}. This makes the traffic overhead very low. Another mechanism that reduces traffic overhead is the number of initialization messages at the start of the connection. For example, the protocol TLSv1.2 uses handshake messages to negotiate on a cipher suite and then create a shared key using some key exchange protocol \cite{Ristic2014}. The protocol created in this application uses a pre-configured cryptography suite and doesn't have to spend time negotiating it like TLS and only sends necessary messages to establish a secret key. This results in the protocol needing 1 round trip to initialize a connection. Again more information on the protocol flow can be found in chapter \ref{ch:network_com_sec}.

However, one flaw is that this application uses TCP. To minimize traffic overhead UDP would be a better fit since it doesn't require connection initialization via the three-way handshake. It would also mean that things like flow control, error handling, synchronization and more would have to be solved in the application layer. TCP was chosen over UDP because it has been designed to do these things well and efficiently. As a result, the application needs 2 round trips to initialize a connection.
