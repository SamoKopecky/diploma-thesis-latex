One of the goals of this thesis is to create an application capable of exchanging messages or files between two users. Secured only using quantum-resistant cryptography while relying only on the console environment for controlling the application. To exchange messages the application uses the peer-to-peer paradigm. However, there is still one peer called the user and one peer called the server to differentiate between the two peers. That is why in some places like the implementation code, client and server are used instead of peers. It is also important to mention that the application uses TCP as its transport layer protocol. It is implemented in the Go programming language. As the public key algorithms, CRYSTALS-Kyber and CRYSTALS-Dilithium are chosen. AES is chosen as the symmetric key encryption algorithm. Additional sections in this chapter explain why these choices were made and shortly introduce the chosen tools. The keyword modular in the thesis name signifies algorithm modularity in its implementation (explained further in Subsection \ref{subsec:modular_app}).
