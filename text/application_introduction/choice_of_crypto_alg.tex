As mentioned in \ref{sec:nist} a group of three lattice-based algorithms has been standardized by NIST, of which Kyber and Dilithium are a part of. That is the main reason why they are chosen as the public key algorithms for this thesis. Since the application is modular any security level of Kyber and Dilithium can be used but the default choice is Kyber1024 and Dilithium5. Other possible choices for these algorithms can be seen in appendix \ref{ch:available_algs}. Each of them has its respective chapter (\ref{ch:kyber} and \ref{ch:dilithium}) which describes some implementation aspects and also introduces some theoretical background that makes them secure.

\object{tab}{tables/choose_algs}{Choosen algorithms}{tab:choosen_algs}

As for symmetric cryptography, AES is chosen with a key size of 256\,bits. AES is used in modern cryptography as the main symmetric cipher and is also usable in post-quantum cryptography but requires bigger-sized keys, due to the threat of Grover's algorithm. Used hash functions in this application are SHAKE-128/256 for the implementation of Kyber and Dilithium following the author's recommendations. SHA3-512 is used by the networking part of the application. A summary of the chosen cryptographic algorithms can be found in the table \ref{tab:choosen_algs}.
