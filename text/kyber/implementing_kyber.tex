The implementation of Kyber in this thesis is done using only the standard Go library except for one external library called crypto\footnote{\url{https://pkg.go.dev/golang.org/x/crypto}} which is required for the implementations of the hash functions SHAKE-128/256 (refer to section \ref{sec:hash_functions} for hash functions). Figure \ref{img:kyber_all} illustrates a very simplified block diagram of how Kyber works. Individual blocks in this figure represent a mathematical structure or a variable in a program. This structure is almost always composed of polynomials which represent a ring and is described in section \ref{sec:kyber_theroteical}. A collection of polynomials can also be called a vector of polynomials. The small letter or number at the start of an arrow coming from the structures denotes the size of the structure. In the case of a vector, it denotes the number of polynomials the vector contains. So for example the letter $k$ denotes that a vector consists of $k$ polynomials.

As to the process of how Kyber works, firstly the public and private keys need to be generated. A~random message $m$ is then encrypted by one communicating entity using the~public key. The encrypted message is then decrypted by the other communicating entity and $m$ becomes the shared key $K$. The following subsections will explain each sub-algorithm of the block diagram in more detail. The Go code in the practical part is also a great reference to understand how Kyber works. These functions are then wrapped by another set of three functions which are also key generation, encapsulation and decapsulation. This is done to provide additional security for the Kyber scheme.

\object{tab}{tables/kyber_security_levels}{Kyber security levels \cite{YbbuGxVPF0GGTxfN}}{tab:kyber_sec_levels}

Kyber uses a set of parameters to define its security level, of which it has three as seen in table \ref{tab:kyber_sec_levels}. Kyber in this thesis is implemented for all the parameter levels. What individual parameters mean will be explained in further subchapters.
