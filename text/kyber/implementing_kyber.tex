The implementation of Kyber in this thesis is done using only the standard Go library except for one external library \cite{00fV2cvg7Z6H2tS3} which is required for the implementations of the hash functions \texttt{SHAKE-128/256} (refer to section \ref{sec:hash_functions} for hash functions). Figure \ref{img:kyber_all} illustrates a very simplified block diagram of how Kyber works. The small letter or number at the start of an arrow denotes the size of the object in polynomials. So for example the letter $k$ denotes that an object consists of $k$ polynomials. At first the public and private keys need to be generated. A~random message $m$ is then encrypted by one communicating entity using the~public key. The encrypted message is then decrypted by the other communicating entity and $m$ becomes the shared key. The following subsections will explain each sub-algorithm of the block diagram in more detail. The Go code in the practical part is also a great reference to understand how Kyber works.

\object{tab}{tables/kyber_security_levels}{Kyber security levels \cite{YbbuGxVPF0GGTxfN}}{tab:kyber_sec_levels}

Kyber uses a set of parameters to define its security level, of which it has three as seen in table \ref{tab:kyber_sec_levels}. Kyber in this thesis is implemented for all the parameter levels. What individual parameters mean will be explained in further subchapters.
