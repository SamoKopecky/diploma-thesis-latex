Kyber uses a structure called rings, more specifically the ring $R_q$ denoted as
\begin{equation}
  \mathbb{Z}_q[X]/(X^n+1).
\end{equation}
A ring contains a polynomial of $n$ elements, where the coefficients of this polynomial are integers reduced modulo $q$ and the powers of the polynomial are reduced $(X^n+1)$. The parameters $n$ and $q$ are defined in Table \ref{tab:kyber_sec_levels}. An example of a~polynomial with elements from the ring $R_q$ is
\begin{equation}
  t_1 = 1564 + 2189x + 258x^2 + \dots + 655x^{n-2} + 2587x^{n-1}.
\end{equation}
A vector of size $k$ consists of $k$ polynomials with coefficients from the ring $R_q$. The~parameter $k$ can be found in Table \ref{tab:kyber_sec_levels}. The the polynomial $t_1$ from the~previous example together with a new polynomial $t_2$
\begin{equation}
  t_2 = 2408 + 1932x + 420x^2 + \dots + 3256^{n-2} + 2399^{n-1},
\end{equation}
form a vector of polynomials $\mathrm{T}=(t_1, t_2)$. A matrix of size $k\times k$ consists of $k^2$ polynomials from the ring $R_q$ aligned as a square 2-dimensional matrix. \cite{YbbuGxVPF0GGTxfN}

The addition of elements from a ring is just adding the individual polynomials and is relatively fast. Multiplication of vectors or matrices the usual way (multiplying each element by each element of the other polynomial) is computationally much more demanding with big $n$. In this case where $n=256$ the number of computations would be $n^2=262144$. A more efficient way to calculate the multiple of two polynomials is using an NTT (\acl{NTT}) where the number of operations is only $n\,\mathrm{log}(n)=1387$. This transformation is a more specific version of the FFT (\acl{FFT}). However, before doing the NTT multiplication it is first required to transform the polynomial into NTT form. Do the calculation with some other polynomial in NTT form and then do the inverse NTT transformation on the result. In this thesis structures that are converted to NTT are denoted with a hat, for example, \rmhat{A}. \cite{Liang2021}
