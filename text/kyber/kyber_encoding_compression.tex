To transfer polynomials over the network they need to be serialized into bytes. Kyber defines two functions for this purpose:
\begin{itemize}
  \item \texttt{encode}($p$,\,$l$)\,--\,convert a polynomial into $32*l$ bytes,
  \item \texttt{decode}(B,\,$l$)\,--\,convert $32*l$ bytes into a polynomial.
\end{itemize}
Since these functions only work on a polynomial, if they are to be applied to a vector of polynomials, every polynomial is processed separately.

Another functionality of Kyber is the compression of polynomials that are encoded. Because Kyber is based on LWE, the calculations don't have to be precise just close enough. This is why a compression mechanism that discards some low-order bits from encoded polynomials can be utilized. Two more functions are defined by Kyber for compressing and decompressing bytes
\begin{itemize}
  \item \texttt{compress}($x$,\,$d$)\,--\,compress a number into the range of $\{0,\dots,2^d-1\}$,
  \item \texttt{decompress}($x$,\,$d$)\,--\,decompress a number while loosing some low-order bits.
\end{itemize}
In order to apply this transformation to a polynomial vector each polynomial is transformed separately. The functions are used on each coefficient of the polynomial. The parameters $d_u$ and $d_v$ are used as the inputs for these functions.

Polynomials need to be randomly generated in Kyber. A CBD (\acl{CBD}) function is defined which takes as an input in a byte array of some parameter $\eta$ multiplied by 64. It then generates a fixed length polynomial of size $n$. This function is also used in generating random vectors where each polynomial is again generated separately. The input $\eta$ value can either be $\eta_1$ or $\eta_2$ defined in Table \ref{tab:kyber_sec_levels}.
