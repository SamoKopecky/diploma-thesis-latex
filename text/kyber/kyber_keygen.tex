The key generation functions starts by generating random parameters (illustrated by Figure \ref{img:cpapke_keygen}). A random seed $\rho$ is used to generate the matrix \rmhat{A}. It is publicly known to everyone and needs to be shared. However, since the function that generates it is deterministic only $\rho$ needs to be shared instead of the whole matrix. This mechanism saves a lot of network traffic because the matrix \rmhat{A} would consume a lot more network traffic than just sending $\rho$. Two vectors $s$ and $e$ are generated from a different random seed. In this case, the seed is not shared since $s$ and $e$ need to remain secret. After transforming the generated vectors into the NTT domain, \rmhat{A} and \rmhat{s} are multiplied. The vector \rmhat{e} is then added to the result and creates the public key. The encoded vector $s$ is then used as the private key.

\object[0.5]{obr}{pictures/kyber_keygen.pdf}{Kyber key generation}{img:cpapke_keygen}
