Two of the most used communication paradigms to provide services to users are client-server and peer-to-peer, where the the former is more commonly seen in the internet. As the name implies a service is hosted by a server, a client or more clients can connect to this server to consume the hosted service. Most of the time the server is a more powerful computer system, so that it can handle more requests at the same time. Services can be provided on individual basis, for example a simple website that provides HTML content to the user. It can also provide a connection between two users, so that they can communicate. For this to work the server has to create multiple connections, one with each of the users. If the communication is encrypted, the server has to first decrypt an encrypted message, read it, encrypt it again and send it to the second user. The content of the exchanged messages were also seen by the server, which implies the users have to trust the server to not store or log their exchanged messages anywhere.

The second communication paradigms, peer-to-peer relies on the fact that if two entities want to communicate they will create a connection between them only which eliminates the problem with the client-server paradigm connecting two users. Each entity consists of a server and a client also since each entity needs to able to listen for incoming connections and also accept create connections to other listening entities. 

TODO: finish