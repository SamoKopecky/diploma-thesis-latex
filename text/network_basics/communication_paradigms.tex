Two of the most used communication paradigms to provide services to users are client-server and peer-to-peer, where the former is more commonly used in~the~Internet \cite{Forouzan2010}. As the name implies a service is hosted on a server, and a client or more clients can connect to this server to consume the hosted service. Most of the time the server is a more powerful computer system so that it can handle more requests at the same time. Services may consist of providing some content to one user, for example, a simple website that provides HTML content. It can also provide a connection between two users so that they can communicate. For this, to work the server has to create multiple connections, one with each of the users. If the communication is encrypted, the server has to first decrypt an encrypted message, read it, encrypt it again and send it to the second user. The content of the exchanged messages was also seen by the server, which implies the users have to trust the server to not store or log their exchanged messages anywhere. However, if the client-server paradigm is enhanced with end-to-end encryption, the server doesn't have to decrypt/encrypt anything and just forwards the messages. For more information on end-to-end encryption see section \ref{sec:e2e}.

The second communication paradigm, peer-to-peer relies on the fact that if two entities want to communicate they will create a connection between them only, which eliminates the problem with the client-server paradigm of connecting two users. Each entity consists of a server and a client, since both of them need to~be~able to listen for incoming connections and also accept connections from~other listening entities. This makes it also a derivative of the client-server paradigm just without the middle entity. It also makes it easier to implement end-to-end encryption.

The issue with using peer-to-peer for communicating with users is that each user needs to have an open port. Since most users on the internet are behind a NAT (\acl{NAT}), it is not always easily solvable. This issue is mitigated by using the client-server paradigm since the clients can initialize the~connections with the server which then transfers the messages between the initialized connections. However, as mentioned before this leaves the messages open and readable by the server if end-to-end encryption is not used. A comprise has to be made between the ease of use and the amount of trust one is willing to give.
