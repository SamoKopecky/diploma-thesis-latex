In order to fully understand how any application that creates a communication channel between two entities works, it is important to firstly look at the concepts of network basics. The contents of this chapter will focus on the topis such as the~TCP/IP protocol suite, TCP (\acl{TCP}), UDP (\acl{UDP}) protocols and differences between client-server and peer-to-peer communications.

The TCP/IP protocol suit consists of 5 layers as can be seed in table \ref{tab:tcp_ip_table}. Each layer is defined by one or more protocols. A protocol defines strict rules for what, how and when should an entity communicate \cite{Forouzan2010}. Each layer in the TCP/IP suite contains at least one protocol, which again dictates how the communication should proceed on that layer. A protocol layer communicates horizontally with other protocol layers using PDUs (\acl{PDU}), each PDU is either encapsulated or de-encapsulated into another PDU, depending on way the data is flowing through the TCP/IP layers. During this process a new header is added or removed. A~header contains important information fot that specific layer. For example the IP address is contained in the header for the network layer. The layers also communicate vertically using either physical channels (physical and data link layers) or virtual channels (all other layers). Physical channels are created between a physical medium by which the bits travels through, virtual channels on the other hand are created between applications on devices. Channels on each layer use a different identifier to differentiate between them. Aforementioned information for each layer can be found in table \ref{tab:tcp_ip_table}.

\object{tab}{tables/tcp_ip_details}{TCP/IP protocol suite \cite{Forouzan2010}}{tab:tcp_ip_table}
