TCP and UDP work on the transport layer in the TCP/IP protocol suite. The transport layer is responsible for creating connections between applications, where each applications is identified with a port number. A port number can be any number in the range of 0-65535 \cite{Forouzan2010}. For an application to be available it must listen on a port number so a client knows where to send his data. Similarly if a client connects to an application he is also given a port number so that the server knows where to send this data. Booth TCP and UDP work with port numbers but an application is able to listen on the same port for booth TCP and UDP protocols.

The UDP protocol was designed to be fast and unreliable \cite{Forouzan2010}. It posses these properties because it is a connection-less protocol. That means that there is no guarantee that the data that is being transferred will arrive as intended and without errors. It also means that the data can be sent faster and has less overhead communication compared to TCP. This model fits very well while sending very small amounts of data very quickly, like in the case of DNS translation. Before any data can be transferred, it has to be split into datagrams of smaller size. These are then sent one by one to the targeted entity.

The TCP protocol on the other hand is a connection-oriented protocol. Before data can transferred between two entities first a connection has to be established using the three-way handshake (figure \ref{img:three_way}) \cite{Forouzan2010}. It works by setting one bit flags in the TCP header, in this case the SYN and ACK flags. After a connection has been established, the data transfer can begin. Unlike UDP, TCP is also numbering its segments which means it is able to detect if a segment was lost while being transferred, and then it can try to transfer it again. Another feature of TCP is flow control, which can be used for controlling how much data the communication entities can exchange at one time \cite{Forouzan2010}. All of these features bring a much bigger overhead to each segment since more information needs to be tracked. This takes a toll on how fast segments can be transferred and also increases the size of the segments, which results in a slower and more reliable protocol then UDP. An example of good usage for the TCP is the HTTP protocol where a website needs to transferred exactly as intended without any errors.


\object[0.5]{obr}{pictures/3way_handshake.pdf}{three-way handshake}{img:three_way}
