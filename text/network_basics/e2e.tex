E2EE (\acl{E2EE}) is a concept that allows data to be transmitted from one end user to another end user without being revealed or being tampered with along the way. It is mostly used in messaging apps. It is very easy to implement using the peer-to-peer parading. Before communication starts, each of the end users converges on some private key using some key exchange protocol. One user uses this key to encrypt the data and the other user can then decrypt it. Of course in a real-life scenario where third parties try to attack this communication a lot of other things have to be considered like the integrity of the data and authenticity of the users.

However, in the client-server paradigm, it's a bit trickier to implement E2EE, since there is always some other entity between two end users. If the middle entity is a malicious one, it could easily use a man-in-the-middle attack on the key exchange protocol. Despite this, there are some protocols that provide E2EE that ensure no third entity can utilize such an attack. One such example is the Signal protocol which uses the Double Ratchet Algorithm \cite{Marlinspike2016}.
