This family of post-quantum cryptography utilizes error correction codes. These are codes that can either detect or correct an error in some binary string by adding additional bits. However they can correct/detect an error up to a threshold. If an big enough error is introduced, the code may no longer be able to detect or correct it.

The first ever code-based scheme was introduced by Robert J. McEliece in 1978 and so the scheme got the name from its inventor McEliece. The private key is defined as a random \textit{Goppa code} which is able to correct errors in a coded sequence of bits. The public key is a matrix $G$ and the plaintext $m$ is a bit string. Additionally another bit string is randomly created called $e$. The ciphertext $c$ is then calculated with
\begin{equation}
  c=mG+e.
\end{equation}
Only the owner of the aforementioned \textit{Goppa code} can extract $m$ and $e$ from $c$ since the code was designed to efficiently correct errors added by the bit string $e$. \cite{Bernstein149}

Since this cryptosystem was introduced in 1978, it is well understood and has never been successfully broken. However for the system to be secure the private/public keys have to be relatively large compared to the keys of modern cryptography like ECDSA. On the other hand they are very fast compared to the other algorithms submitted to the NIST standardization process. That is why 3 code-based algorithms are still being considered in the 4th round. \cite{Chen2016}
