Replacing modern cryptography like RSA with post-quantum algorithms is not as easy as it might seem. Quantum-resistant algorithms are certainly needed to prepare for the thread for quantum computers but one big disadvantage of these algorithms compared to modern cryptography is their computational requirements. Since they use more complicated structures and principles they also require more memory and processing power to compute. Some embedded devices might even take too long to compute some post-quantum algorithm to be useful or might just fail since they don't have enough memory. Another problem is the key sizes of these algorithms. They are a lot bigger as can be seen in table \ref{tab:key_sizes} compared to modern cryptography. The chosen security parameters for the algorithms mentioned in the table below correspond to the NIST security level of 3. Level 3 is defined as a security level that is only breakable by an attack that can break the AES algorithm with a key size of 192 bits or less \cite{8lV5dQrQyshiCp3i}. Some embedded devices also have a very limited network bandwidth because they are battery-powered. These are the reasons why adapting post-quantum cryptography might not be as seamless as it might seem.

\object{tab}{tables/key_sizes}{Key size comparisons \cite{Barker2020}\cite{YbbuGxVPF0GGTxfN}\cite{y0VQZiTmHEg2xvPn}}{tab:key_sizes}
