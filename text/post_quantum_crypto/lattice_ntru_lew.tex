Another post-quantum KEM scheme is NTRU or \acl{NTRU}. It is one of the most efficient public key encryption schemes since instead of~using a basis for  its public key, it uses a $p$-coefficient polynomial:
\begin{equation}
  h=h_0+h_1x+h_2x^2+\dots+h_{p-1}x^{p-1}.
\end{equation}
Like GGH this scheme is based on the CVP. The message consists of two polynomials $d, e$ which are not a lattice points and the ciphertext $c$ which is calculated from $d, e$ where $h$ points to a lattice point. An attacker cannot easily compute back $d, e$ only with the knowledge of $h, c$. The use of polynomials makes NTRU much faster then the GGH cryptosystem and was considered heavily for post-quantum standardization by NIST \cite{Bernstein149}.

LWE is not a cryptosystem by itself, but many cryptosystem are based on it. The problem is based in modular linear equations for example
\begin{align}
  3s_1+6s_2+7s_3+2s_4&\equiv 10\,\mathrm{mod}\,11,\\
  10s_1+8s_2+3s_3+5s_4&\equiv 1\,\mathrm{mod}\,11,\\
  5s_1+s_2+7s_3+10s_4&\equiv 8\,\mathrm{mod}\,11,\\
  6s_1+8s_2+3s_3+4s_4&\equiv 7\,\mathrm{mod}\,11,
\end{align}
where the goal is to find $s_1, s_2, s_3, s_4$. This is easily solvable even for big $n$ amount of equations with the Gaussian elimination, but if an error is added to the right side of each equation (-1 or +1)
\begin{align}
  3s_1+6s_2+7s_3+2s_4&\equiv 9\,\mathrm{mod}\,11,\\
  10s_1+8s_2+3s_3+5s_4&\equiv 2\,\mathrm{mod}\,11,\\
  5s_1+s_2+7s_3+10s_4&\equiv 9\,\mathrm{mod}\,11,\\
  6s_1+8s_2+3s_3+4s_4&\equiv 6\,\mathrm{mod}\,11,
\end{align}
for big $n$ it becomes a significantly harder problem. \cite{Regev2005}

Unlike other mentioned lattice-based post-quantum cryptosystems, LWE-based cryptosystems are supported by a theoretical proof of security \cite{Bernstein2009}. This makes them a very good candidate for standardization by NIST, more specifically the algorithm CRYSTALS-Kyber (see section \ref{sec:nist}).
