Lattice-based cryptography is said to be the most promising replacement for public key cryptography, since two of them have already been standardized by the \acl{NIST} or NIST (see section \ref{sec:currently_standardized}) \cite{Alagic2022}. A lattice can be described as an infinite set of points in an $n$ dimensional space. The space generated by these points is a periodic structure, an example can be seen in figure \ref{img:lattice}. A lattice\,--\,the points in it\,--\,is generated by $n$ linearly independent vectors which can also be called a bases for the lattice \cite{Ajati1996}. Linearly independent vectors have the special property of not being a combination of any other vectors from the set of linearly independent vectors. An example of these vectors is also illustrated in figure \ref{img:lattice}. Vectors that generate a lattice can also be written in mathematical notation as
\begin{equation}
  \mathcal{L}(B)=(b_1,\dots,b_n)
\end{equation}
where $\mathcal{L}(B)$ denotes a lattice created by a basis $B$. The basis is created from vectors ($b_1,\dots,b_n$).

\object[0.5]{obr}{pictures/lattice.pdf}{2-dimensional lattice}{img:lattice}

To use lattices for cryptographic constructions, a vector $v$ in a lattice has to be defined with coordinates from the set of all integers $\mathbb{Z}$. If every coordinate is then reduced with the operation defined as
\begin{equation}
  v\equiv v\,\mathrm{mod}\,q
\end{equation}
where $q$ is also an integer from $\mathbb{Z}$, the lattice is then called a q-ary lattice \cite{Bernstein2009}.

A cryptographic construction in lattices additionally also needs a mathematical problem to be defined that can easily be calculated given in input but difficult to invert and calculate back the input that was given, it other words a one way function has to exists. One-way functions may also be described as a computational problems. In lattice-based cryptography there exist many computational problems, some of the are
\begin{itemize}
  \item \textbf{SVP}\,--\,\acl{SVP},
  \item \textbf{CVP}\,--\,\acl{CVP},
  \item \textbf{LWE}\,--\,\acl{LWE}\cite{Bernstein2009}.
\end{itemize}
How these computational problems are used and in which cryptographic algorithms or cryptosystems will be described in the following sections.

% Describe in detail what it is, what kind of problems it is dependent on, what algorithms exist, where it is used currently, what kind of attacks might be possible, speed, key sizes adoption etc.
