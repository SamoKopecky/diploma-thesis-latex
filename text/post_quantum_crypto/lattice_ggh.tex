

As mentioned in \ref{sec:key_agreement} KEM (\acl{KEM}) is one way of creating a KEP. In the case of lattice cryptography, an algorithm for creating a dedicated key exchange method like DH hasn't been found in lattice based cryptography so the only choice is to use a KEM. A KEM needs a public key encryption scheme to work, fortunately many of them that use lattices have been discovered, so they can be used as key encapsulating mechanisms. 

One of the first public key encryption schemes was the GGH cryptosystem which was named after its inventors Goldreich, Goldwasser and Halevi. Booth the private and public keys are vector basis $B$ and $H$ respectively. A basis can also be written as a matrix where the columns of the matrix consist of the basis vectors. Additionally they form the same lattice $\mathcal{L}(B)=\mathcal{L}(H)$. The basis $B$ is a good lattice and generates orthogonal or nearly orthogonal vectors. Basis $H$ is the called the bad basis which is derived from basis $B$ using a matrix $T$ where
\begin{equation}
  \begin{aligned}
    BT=H, \\
    HT^{-1}=B.
  \end{aligned}
\end{equation}
This transformation of $B$ into $H$ creates an orthogonality defect, which means the generated vectors by the basis are not longer orthogonal or close to orthogonal, this fact will be important later. The message to be encrypted is encoded into a vector $v$ which is a lattice point in $\mathcal{L}$. Next a small noise vector $e$ is chosen that is not a lattice point in $\mathcal{L}$. Given these values the ciphertext $c$ can be computed with $c = Hv + e$. The vector $v$ or the plaintext can be extracted from $c$ given $v=T \round{B^{-1}c}$. The rounding operation is very important here since it removes the error that was added by vector $e$. \cite{Bernstein2009}\cite{Goldreich1997}

\object[0.5]{obr}{pictures/lattice_ggh.pdf}{\acl{CVP}}{img:lattice_ggh}

Finding the original vector from the cipertext is called the Closest Vector Problem (CVP) and is illustrated by figure \ref{img:lattice_ggh}. The goal of this problem is to find the closest vector that is on a lattice point using a vector that isn't on a lattice point. The security of GGH relies on fact that the CVP is easily computed while using the good basis $B$, but hard in the bad basis $H$. As mention earlier, a good basis is orthogonal and finding the closest vector is easily done using Babai's algorithm. However this algorithm is inefficient in a basis that is not orthogonal, which in this case is the basis $H$. Based on this fact it can be assumed that only the owner of the good basis (private key) can decrypt a message. \cite{Goldreich1997} 

The only problem with GGH is that for it to be secure enough, it needs to have very big keys and a result the computations are too slow. That is why this algorithm can't be used in practice \cite{Bernstein2009}.
