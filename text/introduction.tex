\chapter*{Introduction}
\phantomsection
\addcontentsline{toc}{chapter}{Introduction}

Until now the ever increasing amount of computer power available was met with increasing key sizes or parameters for existing cryptographic algorithms. For example RSA (\acl{RSA}) where a few years ago 2048 bits was providing sufficient security but now 3072 bits of security is needed Development of quantum computers is the next big step in technology evolution and brings many new possibilities of improvement but also many dangers to modern cryptography algorithms. For example Shor's algorithm which is capable of braking modern asymmetric cryptography, which includes popular algorithms like RSA, ECDH (\acl{ECDH}), ECDSA(\acl{DSA}).

A new approach to competing with the increasing computational power and new technologies had to introduced. To combat the problem of a~possible quantum supremacy happening this thesis introduces the concept of post quantum cryptography which consists of cryptographic algorithms that are resistant to attacks using quantum algorithms or classical algorithms. One such group of post quantum algorithms are lattice-based algorithms. They are the most promising group of post quantum algorithms for standardization by NIST (\acl{NIST}). NIST has so far led three rounds of the standardization process. During the writing of this thesis the third round of NIST standardization has ended and the two of the winning algorithms are Kyber and Dilithium. Kyber is a KEM (\acl{KEM}) and Dilithium servers as a digital signature algorithm. These algorithms are implemented in this thesis. There is also a 4th round where algorithms from the families of hash-based cryptography and code-based cryptography are competing.

The implementation language used for Kyber and Dilithium is Go. Go is chosen because it creates a good balance between performance and simplicity. The performance is owned due to the~fact that is a compiled language like C and shares many of its features. However it also frees the programer of many difficult and error prone features like memory management. In Go its solved using a garbage collector. To introduce the basic idea of these implemented algorithms, simplified block diagrams explaining the processes of~Kyber and Dilithium are located in appendix \ref{ch:block_diagrams}.

Since these post-quantum algorithms have just been standardized recently, there aren't that many useful applications and programs that utilize them. A good first step would be to create a simple chatting application or file sharing application to showcase the security of these algorithms. One such application is introduced in this thesis.
