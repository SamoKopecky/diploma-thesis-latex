\chapter*{Conclusion}
\phantomsection
\addcontentsline{toc}{chapter}{Conclusion}

Post-quantum algorithms Kyber and Dilithium are successfully implemented in this thesis using the Go programming language. The measured performance of these implementations was compared to a cryptographic library created by Cloudflare. The conclusion is that Cloudflare's implementation of the algorithms is on average about 4.2 times faster than the implementation done in this thesis. Algorithms are described in this thesis along with other families of post-quantum cryptography.

Along with the implementations of Kyber and Dilithium, this thesis also introduces a quantum-safe modular communication application. The modularity is achieved by allowing any post-quantum algorithms implemented in Go to be added with very minimal code changes and good integration into the application. For example, if an algorithm is added it is automatically included in the benchmark suite of tests, where its performance can be compared to other implementations. The final state of the applications uses Kyber which is used to create a session key and Dilithium used for digital signatures. The user of this application can easily change which algorithm he desires simply by editing the configuration file or by generating a new one. Configuration files can be generated by the application and similarly, as with the benchmarking test suites, any added algorithm implementation is seamlessly included as a configuration option.

The underlying protocol which the application uses is very well defined in this thesis and also supports a custom dissector script. This script can be used together with Wireshark to observe the protocol in real traffic scenarios. Methods for preventing common network attacks like Man in the Middle or repeat attacks are also included in the protocol. The application that uses this protocol was designed to be used only in a terminal/console environment. It contains two basic functionalities first it allows the application to send or receive arbitrary data. This data can take a form of a file which can be transmitted between two users. All of this data transfer is of course done over a secured channel. This channel is established using post-quantum cryptography, implied by the aforementioned algorithms. The encryption used during the channel is also quantum-safe since it uses AES-256. The application can also be used as a chatting application. To create a compromise between maintaining a universal text interface and ease of use, the application in this mode runs in a terminal user interface instead of a graphical user interface.

One of the main goals moving forward would be to improve the performance of the affirmation algorithms. To improve the security of the underlying protocol, a~formal verification would be required to theoretically prove its security. Additionally implementing Go unit and integration tests would also help strengthen the protocol security.
