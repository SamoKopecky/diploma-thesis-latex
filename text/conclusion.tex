\chapter*{Conclusion}
\phantomsection
\addcontentsline{toc}{chapter}{Conclusion}

The aim of this this was to introduce the reader with the possibility of a quantum supremacy where powerful enough quantum computers are capable of breaking modern cryptography. As was shown this might be a real possibility in the near future since IBM and other companies have started researching quantum computers and even building them. However the best quantum computer at the time of writing this is a 127 qubit one. Although that doesn't change the fact that post quantum algorithms which are resistent to quantum algorithm attack for public key cryptography are needed.

So far this thesis has introduced Kyber and Dilithium which are lattice-based post quantum algorithms standardized by NIST (\acl{NIST}). The continuation of this thesis will include implementations for additional post quantum algorithms such as SPHINCS+ which is based on hashes and McEliece based on codes.

The implemented algorithms in this thesis are working correctly but are much less performant then other implementations such as the cloudflares circl library. One of the goals for the continuation of this thesis will be improving the performance of Kyber and Dilithium. Another goal will be creating a client-server communication protocol for exchanging data. The protocol will use the implementation of post quantum public key algorithms described in this thesis.
