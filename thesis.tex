% Soubory musí být v kódování, které je nastaveno v příkazu \usepackage[...]{inputenc}

\documentclass[%        Základní nastavení
%  draft,    				  % Testovací překlad
  12pt,       				% Velikost základního písma je 12 bodů
  a4paper,    				% Formát papíru je A4
  oneside,      			% Jednostranný tisk
	%twoside,      			% Dvoustranný tisk (kapitoly a další důležité části tedy začínají na lichých stranách)
	unicode,						% Záložky a metainformace ve výsledném  PDF budou v kódování unicode
]{report}				    	% Dokument třídy 'zpráva', vhodná pro sazbu závěrečných prací s kapitolami

\usepackage[utf8]		  %	Kódování zdrojových souborů je UTF-8
	{inputenc}					% Balíček pro nastavení kódování zdrojových souborů

\usepackage[				% Nastavení geometrie stránky
	bindingoffset=10mm,		% Hřbet pro vazbu
	hmargin={25mm,25mm},	% Vnitřní a vnější okraj
	vmargin={25mm,34mm},	% Horní a dolní okraj
	footskip=17mm,			  % Velikost zápatí
	nohead,					      % Bez záhlaví
	marginparsep=2mm,		  % Vzdálenost marginálií
	marginparwidth=18mm,	% Šířka marginálií
]{geometry}

\usepackage{sectsty}
	%přetypuje nadpisy všech úrovní na bezpatkové, kromě \chapter, která je přenastavena zvlášť v thesis.sty
	\allsectionsfont{\sffamily}

\usepackage{graphicx} % Balíček 'graphicx' pro vkládání obrázků
											% Nutné pro vložení logotypů školy a fakulty

\usepackage[          % Balíček 'acronym' pro sazby zkratek a symbolů
	nohyperlinks				% Nebudou tvořeny hypertextové odkazy do seznamu zkratek
]{acronym}
											% Nutné pro použití prostředí 'acronym' balíčku 'thesis'

\usepackage[
	breaklinks=true,		% Hypertextové odkazy mohou obsahovat zalomení řádku
	hypertexnames=false % Názvy hypertext. odkazů budou tvořeny nezávisle na názvech TeXu
]{hyperref}						% Balíček 'hyperref' pro sazbu hypertextových odkazů
											% Nutné pro použití příkazu 'pdfsettings' balíčku 'thesis'

\usepackage{pdfpages} % Balíček umožňující vkládat stránky z PDF souborů
                      % Nutné při vkládání titulních listů a zadání přímo
                      % ve formátu PDF z informačního systému

\usepackage{enumitem} % Balíček pro nastavení mezerování v odrážkách
  \setlist{topsep=0pt,partopsep=0pt,noitemsep} % konkrétní nastavení

\usepackage{cmap} 		% Balíček cmap zajišťuje, že PDF vytvořené `pdflatexem' je
											% plně "prohledávatelné" a "kopírovatelné"

%\usepackage{upgreek}	% Balíček pro sazbu stojatých řeckých písmem
											%% např. stojaté pí: \uppi
											%% např. stojaté mí: \upmu (použitelné třeba v mikrometrech)
											%% pozor, grafická nekompatibilita s fonty typu Computer Modern!

\usepackage{amsmath} %balíček pro sabu náročnější matematiky

\usepackage{dirtree}	% sazba adresářové struktury
                      % vhodné pro prezentaci obsahu elektronické přílohy (např. CD)

\usepackage[formats]{listings}	% Balíček pro sazbu zdrojových textů
\lstset{              % nastavení
%	Definice jazyka použitého ve výpisech
%    language=[LaTeX]{TeX},	% LaTeX
%	language={Matlab},		% Matlab
	language={Go},           % jazyk C
    basicstyle=\ttfamily\small,	% definice základního stylu písma
    tabsize=2,			% definice velikosti tabulátoru
    inputencoding=utf8,         % pro soubory uložené v kódování UTF-8
		columns=fixed,  %fixed nebo flexible,
		fontadjust=true %licovani sloupcu
    extendedchars=true,
    keywordstyle=\color{orange}\bfseries,
    commentstyle=\itshape\color{purple!40!black},
    numbersep=8pt,
    literate=%  definice symbolů s diakritikou
    {á}{{\'a}}1
    {č}{{\v{c}}}1
    {ď}{{\v{d}}}1
    {é}{{\'e}}1
    {ě}{{\v{e}}}1
    {í}{{\'i}}1
    {ň}{{\v{n}}}1
    {ó}{{\'o}}1
    {ř}{{\v{r}}}1
    {š}{{\v{s}}}1
    {ť}{{\v{t}}}1
    {ú}{{\'u}}1
    {ů}{{\r{u}}}1
    {ý}{{\'y}}1
    {ž}{{\v{z}}}1
    {Á}{{\'A}}1
    {Č}{{\v{C}}}1
    {Ď}{{\v{D}}}1
    {É}{{\'E}}1
    {Ě}{{\v{E}}}1
    {Í}{{\'I}}1
    {Ň}{{\v{N}}}1
    {Ó}{{\'O}}1
    {Ř}{{\v{R}}}1
    {Š}{{\v{S}}}1
    {Ť}{{\v{T}}}1
    {Ú}{{\'U}}1
    {Ů}{{\r{U}}}1
    {Ý}{{\'Y}}1
    {Ž}{{\v{Z}}}1
}

\definecolor{mygreen}{RGB}{0,204,102}
\usepackage{amsfonts} % Required for \mathbb
\usepackage{mathtools} % Used for \lfloor and \rceil
\usepackage{multirow} % Multi row/column tables
\usepackage{fancyvrb} % Used for BVerbatin
\makeatletter
\g@addto@macro{\UrlBreaks}{\UrlOrds}
\makeatother
%%%%%%%%%%%%%%%%%%%%%%%%%%%%%%%%%%%%%%%%%%%%%%%%%%%%%%%%%%%%%%%%%
%%%%%%      Definice informací o dokumentu             %%%%%%%%%%
%%%%%%%%%%%%%%%%%%%%%%%%%%%%%%%%%%%%%%%%%%%%%%%%%%%%%%%%%%%%%%%%%

% V tomto souboru se nastavují téměř veškeré informace, proměnné mezi studenty:
% jméno, název práce, pohlaví atd.
% Tento soubor je SDÍLENÝ mezi textem práce a prezentací k obhajobě -- netřeba něco nastavovat na dvou místech.

\usepackage[
  %%% Z následujících voleb jazyka lze použít pouze jednu
  %czech-english,		% originální jazyk je čeština, překlad je anglicky (výchozí)
  %english-czech,	% originální jazyk je angličtina, překlad je česky
  %slovak-english,	% originální jazyk je slovenština, překlad je anglicky
  english-slovak,	% originální jazyk je angličtina, překlad je slovensky
  %
  %%% Z následujících voleb typu práce lze použít pouze jednu
  semestral,		  % semestrální práce (nesází se abstrakty, prohlášení, poděkování) (výchozí)
  %bachelor,			%	bakalářská práce
  %master,			  % diplomová práce
  %treatise,			% pojednání o disertační práci
  %doctoral,			% disertační práce
  %
  %%% Z následujících voleb zarovnání objetů lze použít pouze jednu
  %  left,				  % rovnice a popisky plovoucích objektů budou zarovnány vlevo
  center,			    % rovnice a popisky plovoucích objektů budou zarovnány na střed (vychozi)
  %
]{thesis}   % Balíček pro sazbu studentských prací


%%% Jméno a příjmení autora ve tvaru
%  [tituly před jménem]{Křestní}{Příjmení}[tituly za jménem]
% Pokud osoba nemá titul před/za jménem, smažte celý řetězec '[...]'
\author[Bc.]{Samuel}{Kopecký}

%%% Identifikační číslo autora (VUT ID)
\butid{211799}

%%% Pohlaví autora/autorky
% (nepoužije se ve variantě english-czech ani english-slovak)
% Číselná hodnota: 1...žena, 0...muž
\gender{0}

%%% Jméno a příjmení vedoucího/školitele včetně titulů
%  [tituly před jménem]{Křestní}{Příjmení}[tituly za jménem]
% Pokud osoba nemá titul před/za jménem, smažte celý řetězec '[...]'
\advisor[Ing.]{David}{Smékal}

%%% Jméno a příjmení oponenta včetně titulů
%  [tituly před jménem]{Křestní}{Příjmení}[tituly za jménem]
% Pokud osoba nemá titul před/za jménem, smažte celý řetězec '[...]'
% Nastavení oponenta se uplatní pouze v prezentaci k obhajobě;
% v případě, že nechcete, aby se na titulním snímku prezentace zobrazoval oponent, pouze příkaz zakomentujte;
% u obhajoby semestrální práce se oponent nezobrazuje (jelikož neexistuje)
% U dizertační práce jsou typicky dva až tři oponenti. Pokud je chcete mít na titulním slajdu, prosím ručně odkomentujte a upravte jejich jména v definici "VUT title page" v souboru thesis.sty.
\opponent[TODO]{TODO}{TODO}[TODO]

%%% Název práce
%  Parametr ve složených závorkách {} je název v originálním jazyce,
%  parametr v hranatých závorkách [] je překlad (podle toho jaký je originální jazyk).
%  V případě, že název Vaší práce je dlouhý a nevleze se celý do zápatí prezentace, použijte příkaz
%  \def\insertshorttitle{Zkác.\ náz.\ práce}
%  kde jako parametr vyplníte zkrácený název. Pokud nechcete zkracovat název, budete muset předefinovat,
%  jak se vytváří patička slidu. Viz odkaz: https://bit.ly/3EJTp5A
\title[Modular communication based on post-quantum cryptohraphy]{Modulární komunikace postavená na postkvantové kryptografii}

%%% Označení oboru studia
%  Parametr ve složených závorkách {} je název oboru v originálním jazyce,
%  parametr v hranatých závorkách [] je překlad
\specialization[Information security]{Informačná Bezpečnosť}

%%% Označení ústavu
%  Parametr ve složených závorkách {} je název ústavu v originálním jazyce,
%  parametr v hranatých závorkách [] je překlad
%\department[Department of Control and Instrumentation]{Ústav automatizace a měřicí techniky}
%\department[Department of Biomedical Engineering]{Ústav biomedicínského inženýrství}
%\department[Department of Electrical Power Engineering]{Ústav elektroenergetiky}
%\department[Department of Electrical and Electronic Technology]{Ústav elektrotechnologie}
%\department[Department of Physics]{Ústav fyziky}
%\department[Department of Foreign Languages]{Ústav jazyků}
%\department[Department of Mathematics]{Ústav matematiky}
%\department[Department of Microelectronics]{Ústav mikroelektroniky}
%\department[Department of Radio Electronics]{Ústav radioelektroniky}
%\department[Department of Theoretical and Experimental Electrical Engineering]{Ústav teoretické a experimentální elektrotechniky}
\department[Department of Telecommunications]{Ústav telekomunikací}
%\department[Department of Power Electrical and Electronic Engineering]{Ústav výkonové elektrotechniky a elektroniky}

%%% Označení fakulty
%  Parametr ve složených závorkách {} je název fakulty v originálním jazyce,
%  parametr v hranatých závorkách [] je překlad
%\faculty[Faculty of Architecture]{Fakulta architektury}
\faculty[Faculty of Electrical Engineering and~Communication]{Fakulta elektrotechniky a~komunikačních technologií}
%\faculty[Faculty of Chemistry]{Fakulta chemická}
%\faculty[Faculty of Information Technology]{Fakulta informačních technologií}
%\faculty[Faculty of Business and Management]{Fakulta podnikatelská}
%\faculty[Faculty of Civil Engineering]{Fakulta stavební}
%\faculty[Faculty of Mechanical Engineering]{Fakulta strojního inženýrství}
%\faculty[Faculty of Fine Arts]{Fakulta výtvarných umění}
%
%Nastavení logotypu (v hranatych zavorkach zkracene logo, ve slozenych plne):
\facultylogo[logo/FEKT_zkratka_barevne_PANTONE_CZ]{logo/UTKO_color_PANTONE_CZ}

%%% Rok odevzdání práce
\graduateyear{2023}
%%% Akademický rok odevzdání práce
\academicyear{2022/23}

%%% Datum obhajoby (uplatní se pouze v prezentaci k obhajobě)
\date{14.\,11.\,2022}

%%% Místo obhajoby
% Na titulních stránkách bude automaticky vysázeno VELKÝMI písmeny (pokud tyto stránky sází šablona)
\city{Brno}

%%% Abstrakt
\abstract[%
  Překlad abstraktu
  (v~angličtině, pokud je originálním jazykem čeština či slovenština; v~češtině či slovenštině, pokud je originálním jazykem angličtina)
]{%
  Abstrakt práce v~originálním jazyce
}

%%% Klíčová slova
\keywrds[%
  Překlad klíčových slov
  (v~angličtině, pokud je originálním jazykem čeština či slovenština; v~češtině či slovenštině, pokud je originálním jazykem angličtina)
]{%
  Klíčová slova v~originálním jazyce
}

%%% Poděkování
\acknowledgement{%
  Rád bych poděkoval vedoucímu bakalářské/diplomové/disertační práce
  panu Ing.~XXX YYY, Ph.D.\ za odborné vedení,
  konzultace, trpělivost a~podnětné návrhy k~práci.
}%  % do tohoto souboru doplňte údaje o sobě, druhu práce, názvu...

%%%%%%%%%%%%%%%%%%%%%%%%%%%%%%%%%%%%%%%%%%%%%%%%%%%%%%%%%%%%%%%%%%%%%%%%

%%%%%%%%%%%%%%%%%%%%%%%%%%%%%%%%%%%%%%%%%%%%%%%%%%%%%%%%%%%%%%%%%%%%%%%%
%%%%%%     Nastavení polí ve Vlastnostech dokumentu PDF      %%%%%%%%%%%
%%%%%%%%%%%%%%%%%%%%%%%%%%%%%%%%%%%%%%%%%%%%%%%%%%%%%%%%%%%%%%%%%%%%%%%%
%% Při načteném balíčku 'hyperref' lze použít příkaz '\pdfsettings':
\pdfsettings
%  Nastavení polí je možné provést také ručně příkazem:
%\hypersetup{
%  pdftitle={Název studentské práce},    	% Pole 'Document Title'
%  pdfauthor={Autor studenstké práce},   	% Pole 'Author'
%  pdfsubject={Typ práce}, 						  	% Pole 'Subject'
%  pdfkeywords={Klíčová slova}           	% Pole 'Keywords'
%}
%%%%%%%%%%%%%%%%%%%%%%%%%%%%%%%%%%%%%%%%%%%%%%%%%%%%%%%%%%%%%%%%%%%%%%%

\pdfmapfile{=vafle.map}

%%%%%%%%%%%%%%%%%%%%%%%%%%%%%%%%%%%%%%%%%%%%%%%%%%%%%%%%%%%%%%%%%%%%%%%
%%%%%%%%%%%       Začátek dokumentu               %%%%%%%%%%%%%%%%%%%%%
%%%%%%%%%%%%%%%%%%%%%%%%%%%%%%%%%%%%%%%%%%%%%%%%%%%%%%%%%%%%%%%%%%%%%%%
\begin{document}
\pagestyle{empty} %vypnutí číslování stránek

%%% Vložení desek -- od září 2021 na žádost fakulty nepoužíváno
\includepdf[pages=1]%  buďto generovaných informačním systémem
{pdf/student-desky}% název souboru nesmí obsahovat mezery!
%%% NEBO vytvoření desek z balíčku
%%\makecover
%%%
%\oddpage % při dvojstranném tisku přidá prázdnou stránku
%% kazdopádně ale:
%\setcounter{page}{1} %resetovaní čítače stránek -- desky do číslování nezahrnujeme

%% Vložení titulního listu
\includepdf[pages=1]%    buďto generovaného informačním systémem
{pdf/student-titulka}% název souboru nesmí obsahovat mezery!
%% NEBO vytvoření titulní stránky z balíčku
%\maketitle
%%
\oddpage  % při dvojstranném tisku se přidá prázdná stránka

%% Vložení zadání
\includepdf[pages=1]%   buďto generovaného informačním systémem
{pdf/student-zadani}% název souboru nesmí obsahovat mezery!
%% NEBO lze vytvořit prázdný list příkazem ze šablony
%\patternpage{}%
%	{\sffamily\Huge\centering ZDE VLOŽIT LIST ZADÁNÍ}%
%	{\sffamily\centering Z~důvodu správného číslování stránek}
%%
\oddpage% při dvojstranném tisku se přidá prázdná stránka

%% Vysázení stránky s abstraktem
\makeabstract

% Vysázení stránky s rozšířeným abstraktem
% (pokud píšete práci v češtině či slovenštině, vložení rozšířeného abstraktu zrušte;
%  pro semestrální projekt také není potřeba rozšířený abstrakt uvádět)
% \input{text/rozsireny_abstrakt}

%%% Vysázení citace práce
\makecitation

%%% Vysázení prohlášení o samostatnosti
\makedeclaration

%%% Vysázení poděkování
\makeacknowledgement

%%% Vysázení obsahu
\tableofcontents

%%% Vysázení seznamu obrázků
% (vynechejte, pokud máte dva nebo méně obrázků)
\listoffigures

%%% Vysázení seznamu tabulek
% (vynechejte, pokud máte dvě nebo méně tabulek)
\listoftables

%%% Vysázení seznamu výpisů kódu
% (vynechejte, pokud máte dva nebo méně výpisů)
\lstlistoflistings

\cleardoublepage\pagestyle{plain}   % zapnutí číslování stránek

% Macro súbor
\newcommand{\object}[5][1]{
    \IfStrEq{#2}{obr}{
        \begin{figure}[htbp]
            \centering
            \includegraphics[width=#1\textwidth]{#3}
            \caption{#4}
            \label{#5}
        \end{figure}
    }{}
    \IfStrEq{#2}{tab}{
        \begin{table}[htbp]
            \centering
            \caption{#4}
            \input{#3}
            \label{#5}
        \end{table}
    }{}
}

\newcommand{\listing}[5]{
    \lstinputlisting[frame=single,numbers=left,caption={#2},label=#3,firstline=#4,lastline=#5,firstnumber=#4]{#1}
}

\newcommand{\code}[1]{\texttt{#1}}

\newcommand{\codequote}[1]{\textquotedblright#1\textquotedblright}

\newcommand{\minisection}[1]{\noindent{}\textbf{#1}}

\newcommand{\question}[3]{
    \begin{frame}[t]
        \frametitle{#1}
        \emph{#2}\\[2ex]
        %
        #3
    \end{frame}
}

\DeclarePairedDelimiter{\round}\lfloor\rceil

\newcommand{\rmhat}[1]{${\hat{#1}}$}

\newcommand{\itemtt}[1]{\item \texttt{#1}\,--\,}

\newcommand{\npm}[2]{{#1}\,$\pm\,{#2}\%$}


%Pro vkládání kapitol i příloh používejte raději \include než \input
%%% Vložení souboru 'text/introduction.tex' s úvodem
\chapter*{Introduction}
\phantomsection
\addcontentsline{toc}{chapter}{Introduction}

Here comes the introduction of the thesis, for example\,\dots

This thesis is devoted to \acs{DSP} (\acl{DSP}),
especially it analyses the effect happening when the Nyquist condition for \ac{symfs} is not satisfied.%
\footnote{This sentence is only to demonstrate how abbreviations can be used and typeset.}

The template is set to twoside printing by default.
Do not be surprised that you find empty pages in your PDF.
They are there to make the chapters and other important stuff begin on the right side when the document is printed.
Having a~serious reason to print one-sided, please switch the option \texttt{twoside} to \texttt{oneside}!

test

daw
da
w
test
test
test
test
test
test


% Všetky ostatatné kapitoly
\chapter{Current state of cryptography}
\label{ch:state_of_crypto}
Cryptography is an essential part of internet communication. It makes sure an~established connection has three required properties \cite{Bernstein149}
\begin{itemize}
  \item \textbf{confidentiality}\,--\,data can't be read by 3rd parties,
  \item \textbf{integrity}\,--\,data can't be edited by 3rd parties,
  \item \textbf{authenticity}\,--\,communicating parties can't be impersonated.
\end{itemize}
Many cryptographic primitives exist in cryptography to ensure the aforementioned properties. The most commonly used protocol that utilizes these algorithms and~specifications is TLS (\acl{TLS}).

Building blocks for cryptographic algorithms are cryptographic primitives. These are mathematical problems that can be solved in polynomial time ($O(n^x)$) with the~knowledge of some secret. Without the knowledge of this secret the problem can only be solved in exponential time ($O(x^n)$). This means if a new algorithm is found that is able to solve the problem without the knowledge the secret in polynomial time, the underling cryptographic primitive is broken and can no longer be safely used in any cryptographic algorithms or specifications.\cite{Smart2004}

Cryptography can be split into symmetric cryptography and asymmetric cryptography. These groups and their underling cryptographic primitives will be described in more detail in the following sections (sections \ref{sec:symmetric_enc} and \ref{sec:asymmetric_enc}).


\section{Symmetric cryptography}
\label{sec:symmetric_enc}
Symmetric cryptography is used for maintaining the confidentiality of data that is~being transferred over a communication medium. The general idea of symmetric ciphers is that they are fast ciphers (compared to the asymmetric ones) that only use one secret (the secret key) to encrypt data. This key needs to be either pre-shared before the communication starts or a KEP ({\acl{KEP}}) has to be used (see Section \ref{sec:key_agreement}). \cite{Ristic2014}

How symmetric ciphers work is illustrated in Figure \ref{img:sym_crypto}. In a situation where Alice wants to send Bob an obfuscated document (plaintext), Alice first needs to encrypt the document with the shared secret key. She then sends Bob the encrypted document (ciphertext) and Bob can decrypt it again with a shared key. Symmetric ciphers can be split into block and stream ciphers.

\object[0.9]{obr}{pictures/sym_crypto.pdf}{Simplified symmetric cipher}{img:sym_crypto}


\subsection{Block ciphers}
\label{subsec:block_ciphers}
Block ciphers operate on blocks of data and use padding to handle situations when a message can't be perfectly split into blocks. The same key is used for each block. Symmetric block ciphers can also use different modes of operation to add additional context to individual blocks from previous blocks. This process is important for the security of symmetric block ciphers because a block cipher without any mode of operation or an~ECB (\acl{ECB}) mode generates the same output from the same input. This means an attacker could delete or add any block in an encrypted message without the receiver's knowledge. Some examples of a secure mode of operations for block ciphers are \cite{Paar2010}
\begin{itemize}
  \item \textbf{OFB} (\acl{OFB}),
  \item \textbf{CFB} (\acl{CFB}),
  \item \textbf{GCM} (\acl{GCM}).
\end{itemize}

Block ciphers are based on a substitution-permutation network (see Figure \ref{img:sbox_pbox}), which consists of two layers, a substitution layer and a permutation layer as the~name implies. The substitution layer introduces confusion to the data. Confusion creates a correlation between the key and the ciphertext, where one changed bit in the key will generate a change for many bits in the ciphertext. In practice, a substitution layer just substitutes one byte with the help of a substitution table. This table is predefined and used for every operation. On the other hand, the permutation layer introduces diffusion, which means that a changed bit in the plaintext will dissipate into more changed bits in the ciphertext. In other words, it functions by scrambling the order of bytes randomly. An example can be seen in Figure \ref{img:sbox_pbox} of a permutation layer. \cite{Paar2010}\cite{Shannon1949}

Of course, in practice, a cipher needs a lot more than just a simple substitution-permutation network. Good examples of block ciphers that use this principle are AES (\acl{AES}) and DES (\acl{DES}). DES is no longer deemed secure and should not be used \cite{Barker2017}. AES on the other hand is still considered secure even to attacks from quantum computers if longer keys are used \cite{Chen2016}.

\object[0.6]{obr}{pictures/sbox-pbox.pdf}{Substitution-permutation network}{img:sbox_pbox}


\subsection{Stream ciphers}
\label{subsec:stream_ciphers}
Unlike block ciphers, stream ciphers encrypt one bit at a time instead of blocks. The main principle behind stream ciphers is the bit operation XOR and a PRNG (\acl{PRNG}). The key is randomly generated by the PRNG function. Then the message is XORed with the generated key. The XOR operation can also be rewritten as mod $2$ and thus the encryption process can be described with equation
\begin{equation}
  c = E(p)\equiv p + k\ \mathrm{mod}\,2
\end{equation}
\noindent and the decryption process
\begin{equation}
  p = D(c)\equiv c + k\ \mathrm{mod}\,2
\end{equation}
\noindent for $c$ as the ciphertext, $p$ as the plaintext, $k$ as the secret key $E$ and $D$ as the encryption and decryption functions respectively \cite{Paar2010}.

Examples of stream ciphers include RC4, Salsa20 or ChaCha20. It is no longer recommended to use the RC4 cipher. Salsa20 is a newer stream cipher and is~considered to be resident even against quantum computers. \cite{Bernstein149}\cite{Ristic2014}


\section{Hash functions}
\label{sec:hash_functions}
Hash functions work by digesting a~message of arbitrary size into a fixed-sized output or a variable-sized output (SHAKE family hash function) called the hash value. The digest process can also be described as a transformation of bits into another set of bits
\begin{equation}
  H(k, n): \{0,1\}^k \rightarrow \{0,1\}^d,
\end{equation}
where $k$ stands for the size of the input message and $d$ stands for the output size.
\newpage
\noindent For a hash function to be secure it also must possess these three properties \cite{Paar2010}:
\begin{itemize}
  \item \textbf{preimage resistance}\,--\,it is computationally infeasible to find the input of an~already generated hash value,
  \item \textbf{second preimage resistance}\,--\,for a given hash value, it is computationally infeasible to generate two inputs that map to the same hash value,
  \item \textbf{collision resistance}\,--\,there mustn't exist two different inputs that generate the same hash value.
\end{itemize}

How the digest process works internally depends on the specific hash function being used, it doesn't have a single definition. For example, a hash function can be based on a Merkle-Damag\aa rd construction. This construction and many more use compression functions, which take in the input of some size and reduce it into an output of a smaller size. In the Merkle-Damg\aa rd construction, the message is firstly split into blocks. With the help of a compression function, the blocks are then consumed one by one. The output of one compressed block is then fed back to the input of~another round of compression until all the blocks are consumed \cite{Smart2004}.

Other types of constructions are also used such as hash functions based on the KECCEK construction also called the sponge construction. The main idea behind the sponge construction is that after each round of compression, several bits are firstly absorbed by the compression function and then some bits are taken out of each compression iteration. These bits then make up the final hash value. How many bits are absorbed or taken out is dictated by the hash function parameters. \cite{1Od8f4TuMxetfmHu}

Examples of specific hash functions are listed in table \ref{tab:hash_func}. Since SHAKE can generate any sized output, its hash value size is dictated by the parameter $d$.

\object{tab}{tables/hash_functions}{Modern hash functions \cite{1Od8f4TuMxetfmHu}\cite{rd1wUlxEgliEynii}}{tab:hash_func}

Hash functions are used in many areas of cryptography. As an example, they are used in digital signature schemes (section \ref{sec:asymmetric_enc}), message authentication codes (\acs{MAC}), pseudo-random number generators and even public-key quantum-resistant cryptography \cite{Chen2016}.


\section{Asymmetric cryptography}
\label{sec:asymmetric_enc}
The other important type of cryptography is asymmetric cryptography. Compared to symmetric (see section \ref{sec:symmetric_enc}), asymmetric algorithm are most often slower, require bigger sized keys and use two keys instead of one key. One of the keys that is available to the public is called a public key, the second one that only ty the entity that generated it should now is called a secret key. Depending on the use of these keys, asymmetric cryptography can be used in two ways -- as an \textbf{encryption} cipher or as a \textbf{digital signature} scheme.

An encryption algorithm can be seen in figure \ref{img:asym_crypto}. \textit{Alice} can encrypt a document using \textit{Bobs} public key since the public key is shared with everyone and \textit{Bob} wants anyone to send him encrypted messages. After receiving the encrypted document \textit{Bob} can decrypt it with his private key since he is the only one that owns it. \cite{Smart2004}

\object[0.9]{obr}{pictures/asym_crypto.pdf}{Asymmetric encryption cipher}{img:asym_crypto}

Digital signature schemes serve as a tool to verify the origin of data. The following process is illustrated by figure \ref{img:asym_sign}. If \textit{Bob} wants anyone who receives his document to be able to verify he was the one who created it, he signs the document with his private key. Everyone else including \textit{Alice} can check whether the document came from \textit{Bob} by verifying the signature with his pubic key. If the verification succeeds the verifier can be sure that \textit{Bob} generated the signature because he is the only that posses it. \cite{Paar2010}

In practice \textit{Bob} would be signing a hash of the document instead of the document itself, and would also send a the unsigned document. \textit{Alice} would be comparing hashes hash of the document with the verified signature. This is because as mentioned before asymmetric algorithms are slow relative to symmetric algorithms and signing all of the data is unnecessary when signing the hash of a some data servers the same purpose. Hash functions are described in the previous section of this chapter (section \ref{sec:hash_functions}).

\object[0.9]{obr}{pictures/asym_sign.pdf}{Digital signature scheme}{img:asym_sign}


\subsection{Underlying principles}
\label{subsec:underlying_principles}
One of the underlying principles used in asymmetric cryptography is the integer factorization problem (\acs{IFP}). This problem utilizes the idea that factorizing a~big integer $n$ composing of two prime numbers (more than 3072 bits) is impossible to~compute on today's computers in polynomial time \cite{Paar2010}. But producing $n$ from two prime numbers $p$ and $q$ is trivial and fast
\begin{equation}
  n=p*q.
\end{equation}
where $p$ and $q$ are the unique prime numbers. IFP together with modular arithmetic create the RSA (\acl{RSA}) cipher that is one of the most used ciphers used today for creating digital signatures. The private and public keys are derived from the integer $n$.

The other principle that is used often in today's asymmetric cryptography is the discrete logarithm problem (\acs{DLP}). It heavily relies on the use of modular arithmetic and cyclic groups in which there are a finite amount of integer values. This is possible because it uses the modulo operation together with other operations to stay inside this cyclic group. In this group, it is very easy (in polynomial time) to compute $\beta$ with
\begin{equation}
  \alpha^x\equiv\beta\,\mathrm{mod}\,p
\end{equation}
while knowing the values for $x$ and $\alpha$, but very hard (in exponential time on present-day computers) to compute $x$ using this formula
\begin{equation}
  x\equiv\mathrm{log}_\alpha\beta\,\mathrm{mod}\,p
\end{equation}
with the knowledge of only $\alpha$ and $\beta$, where $p$ is a prime number with a bit size of at least 3072 \cite{Paar2010}. DSA (\acl{DSA}) utilizes this problem for creating digital signatures. An alternative algorithm exists that uses elliptic curves instead of cyclic groups called ECDSA (\acl{ECDSA}). This is because the DLP equivalent in elliptic curves is more secure while using the same size for parameters such as $x$ which is the private key \cite{Ristic2014}. This property allows the use of smaller keys while staying on the same level of security.


\section{Key exchange protocols}
\label{sec:key_agreement}
As mentioned in section \ref{sec:symmetric_enc}, secret keys firstly need to be shared between the communicating entities before any encryption can begin. That is where a KEP (\acl{KEP}) is utilized. One category of a KEP is the KEM (\acl{KEM}). As they are a subset of asymmetric cryptography, many algorithms or ciphers used for asymmetric encryption can be converted to a KEM, for example RSA \cite{Ristic2014}. How the key exchange works is illustrated by figure \ref{img:asym_crypto}, but instead of \textit{Alice} encrypting and sending documents, she sends \textit{Bob} a randomly generated key.

Another alternative of a KEP utilizes a dedicated key exchange method, such as the Diffie-Hellman protocol. It also works on the principle of having a public, private key pair like RSA, but each entity exchanges their public key with the other entity and then they calculate the secret key from the knowledge of their own private key and of the opposite entities public key. Instead of relying on IFP it relies on the DLP (see subsection \ref{subsec:underlying_principles}). This brings an advantage, because the DH method can be then upgraded to \acl{ECDH} (ECDH), which is a faster method for exchanging keys then plain DH or RSA \cite{Ristic2014}.


\chapter{Quantum supremacy}
\label{ch:quantum_supr}
Modern cryptography described in chapter \ref{ch:state_of_crypto} was designed with the assumption that the adversary would only have access to a classical computer. It turns out many of the algorithms and schemes used in modern cryptography are extremely vulnerable to quantum computers given the quantum computers has enough computational power~\cite{Bernstein2009}. The following sections will explain what are quantum computers, how can they break classical cryptography and exactly which parts are vulnerable to~quantum computers.

% Mention which groups of modern cryptography qc can break (symmetric, hash, asymetric). Describe how quantum computers can break modern cryptography (Shors algorithm.). Hint the next chapter for solution to this threat. Describe the current state of quantum computers, explain how they work (smplified). What are the best quantum computers to date. A historical view of quantum computers. 

% The table compares the elliptic curve and cyclic groups algorithms, 
% \ref{tab:public_params} 

% use table from \url{https://nvlpubs.nist.gov/nistpubs/SpecialPublications/NIST.SP.800-57pt1r5.pdf}

\section{Quantum data representation}
\label{sec:quantum_data_repr}
Quantum computers as the name implies, are based on the special properties of quantum mechanics. One of these many properties that quantum computers work with is the superposition of states. At very small sizes (sizes of individual particles) objects can be in such a state. Unlike ordinary objects of ordinary sizes, they can exist in more than one location at the same time. This phenomenon only occurs if the object is not being seen (is not being measured). However, this means whenever an object is measured in such a state the position of the object collapses into a single point in space. \cite{Yanofsky2008}

\object[0.4]{obr}{pictures/qubit.pdf}{Representation of a qubit}{img:qubit}

This unique property is what allows data to be represented in a quantum computer. In a classical computer, data is represented using bits. These only have 2 distinct values 0 or 1. Quantum computers don't work with bits but quantum bits or qubits in short. A qubit is represented by two pairs of complex numbers $c_0$~and~$c_1$. Complex numbers can be converted into real numbers $p_0$ and $p_1$
\begin{equation}
  \begin{aligned}
    p_0 & = \lvert c_0 \rvert^2, \\
    p_1 & = \lvert c_1 \rvert^2,
  \end{aligned}
\end{equation}
in this form they represent the probability of a~qubit collapsing (after a~measurement) into discrete values 0 or 1 and becoming a~classical bit \cite{Yanofsky2008}. This concept is also illustrated in figure \ref{img:qubit} where the pointing arrows illustrate the qubit being measured.

By using complex numbers to represent qubits, they can be represented using the bra-ket notation
\begin{equation}
  \lvert\psi\rangle=c_0|0\rangle + c_1|1\rangle
\end{equation}
where $\psi$ represents the particle in a~superposition of all possible states. A quantum computer can hold more than one qubit in a state of superposition. Unlike classical computers which always have one state, quantum computers can use the property of superposition and be in many states at the same time. This means it can evaluate a function for many values at the same time, which leads to great parallelism of quantum algorithms. However, a quantum algorithm doesn't work like a~classical algorithm. It starts with a single position for all the qubits in~the input. During the algorithm, the qubits are manipulated in their superposition state. When the algorithm finishes the state is then measured. At no point during the algorithm, the state can be measured, because then the superposition would be lost due to the qubits collapsing into a single state. \cite{Yanofsky2008}


\section{Shor's algorithm}
\label{sec:shors_algorithm}
The biggest thread to modern cryptography is Shor's algorithm. It can be used for factoring prime numbers ($N$) in time complexity of $O(n^2\ log\,n\ log\ log\,n)$ where $n$ is the number of bits required to represent $N$ \cite{Yanofsky2008}. One of the fundamental problems used in modern cryptography is the IFP (subsection \ref{subsec:underlying_principles}) used in RSA, which can now be broken by Shor’s algorithm in polynomial time. Shor's algorithm can be split into two parts. The first part can easily be computed on a classical computer and the second part can also theoretically be done on a classical computer but it would take much longer then on a quantum computer.

The first part of the Shor's algorithms is as follows. Generate a random number $a$ in the range of $a\in\{2,\dots,N-1\}$ which is co-prime to $N$
\begin{equation}
  \mathrm{GCD}(a, N) = 1,
\end{equation}
fortunately we can use the Euclid's algorithm to compute the GCD (\acl{GCD}) very fast even on a classical computer. From there the order of $a$ has to be found. The order is the smallest number such that
\begin{equation}
  \label{eq:order}
  a^r \equiv 1\,\mathrm{mod}\,N.
\end{equation}
Finding order of $a$ is computationally infeasible in polynomial time using a classical computer, thats why a quantum computer is needed to find $r$ and will be explained later. If $r$ is odd or $a^r\equiv-1\,\mathrm{mod}\,N$ it is discarded and and a new $r$ is found. Equation \ref{eq:order} can now be altered by subtracting 1 from booth sides
\begin{equation}
  a^r -1\equiv 0\,\mathrm{mod}\,N,
\end{equation}
and now can be rewritten as
\begin{equation}
  a^r -1=kN,
\end{equation}
where $k$ is some integer. With the help of $x^2 - y^2=(x+y)(x-y)$ the previous equation can be written as
\begin{equation}
  (\sqrt{a^r}+1)(\sqrt{a^r}-1)= kN,
\end{equation}
or even a more readable version as
\begin{equation}
  \label{eq:half}
  (a^{r/2}+1)(a^{r/2}-1)= kN.
\end{equation}
Equation \ref{eq:half} can now be used to find at least one nontrivial factor of $N$ by calculating
\begin{equation}
  \begin{aligned}
    \mathrm{GCD}((a^{r/2}+1), N), \\
    \mathrm{GCD}((a^{r/2}-1), N),
  \end{aligned}
\end{equation}
and by dividing $N$ with the first factor the second factor can be calculated and thus break any algorithm or cipher that depends on the IFP. After some modifications, it also be used for breaking the DLP. \cite{Yanofsky2008}\cite{Pittenger2000}

The second part of the algorithm as mention is used to find the order of $a$. An~order of a number can also be represented as a period of function  $f_{a,N}(x)$
\begin{equation}
  f_{a,N}(x)\equiv a^x\,\mathrm{mod}\,N,
\end{equation}
where its output values are repeated at regular intervals of size $r$ \cite{Yanofsky2008}. The table \ref{tab:func_period} shows an example for $N=15$, $a=2$ and $x\in\{0,1,2,3,4,5\}$, where it is shown that the period (order) of $a$ is $r=4$. The repeating outputs can also be seen for $x\in\{5,6\}$. As mention earlier, computing this on a classical computer for large $N$ is infeasible but a quantum computer can evaluate a function for many values at the same time (see section \ref{sec:quantum_data_repr}). This property is used in finding the order of $a$. It firstly calculates the repeating sequence of outputs for the function $f_{a,N}$ all at the same time. Using the QFT (\acl{QFT}) the period is found which is the number $r$, then it can be used in the rest of the algorithm. \cite{McMahon2008}

\object{tab}{tables/order_example}{Example of a function period}{tab:func_period}


\section{Grover's algorithm}
\label{sec:grovers_alg}
Symmetric cryptography and hash functions can also be broken by another algorithm named Grover's algorithm. It is categorized as a search algorithm so instead of solving any any mathematical problem, it just searches through all the possible options. Given a set of bits $\{0, 1\}^n$ where $n$ is the size of the set a classical computer will search for a specific binary string of length $n$ in $O(2^n)$ time. Grover's algorithm can search for the same binary string in $O(2^{n/2})$ time. \cite{Yanofsky2008}

Symmetric cryptography keys are also a binary string created from a set of bits size $n$, where Grover's algorithm can be used to find a key by trying all possible values. Similarly hash functions output a binary string also from a set of bits. Grover's algorithm can be used to try to generate all the possible hash values inside a quantum computer and when the the it finds a match it can retrospectively find the output that generated the hash value \cite{Pretson2022}. Since the Grover's algorithm is not as efficient as Shor's algorithm in find solutions that break ciphers or algorithms, key/hash value sizes can be increased to prevent theses kind of attacks \cite{Chen2016}.


\section{Threat to modern cryptography}
\label{sec:threat_to_modern}
The future impact of quantum computers on classical cryptography can be seen in table \ref{tab:impact_of_quantum}. ECC (\acl{ECC}) algorithms and RSA aren't secure from quantum computers with sufficient amount of qubits using the the Shor's algorithm. Currently it is estimated that the required amount of qubits for Shor's algorithm to be efficient enough is in the tens of millions \cite{Bernstein149}\cite{Mosca2015}. IBM managed to create a 127-qubit quantum processor in 2021 so humanity is not yet at the point where everyday internet communication using public key cryptography can be broken using quantum computers \cite{0MBNdFRCTLK35MFY}. 

Although that doesn't change the fact that future quantum computers might be able to. It fact IBM has projected in their new roadmap to a practical quantum computer, that by 2026 they expect to have working quantum computers that contain between 10 000 to 100 000 qubits \cite{Gambetta2021}. However research shows that in order to use quantum algorithms such as Shor's algorithm to break modern cryptography, millions of qubits are required \cite{Bernstein149}. Still a~replacement for the current public key algorithms needs to be found. Each of the new candidates will be discuses in detail in chapter \ref{ch:pq_crypto}.

Symmetric cryptography and hash functions on the other hand are much more resistant to quantum computers. For the current ciphers and algorithm to be quantum-resistant only the symmetric key size and digest size for hash functions needs to increase. For example in the case of AES-128 it is sufficient enough to switch to AES-256 where the performance hit is negligible \cite{Bernstein149}.

\object{tab}{tables/impact_of_quantum_computers}{Impact of quantum computers on classical cryptography\cite{Bernstein149}\cite{Chen2016}}{tab:impact_of_quantum}


\chapter{Post-quantum cryptography}
\label{ch:pq_crypto}
Quantum supremacy may not be happening right now but may happened in the future. New public key algorithms need to be standardized so they can be used as replacements for the quantum vulnerable algorithms such as RSA and ECDH. NIST (\acl{NIST}) has began the first standardization process for post-quantum algorithms, which are algorithms that are resistant to the future thread of quantum computers \cite{Chen2016}. The main candidates that will be described  in individual subsections are
\begin{itemize}
  \item Lattice-based cryptography,
  \item Code-based Cryptography,
  \item Multivariate Public Key Cryptography,
  \item Hash-based digital signature schemes.
\end{itemize}


\section{Lattice-based cryptography}
\label{sec:lattice_based_crypto}
Lattice-based cryptography is said to be the most promising replacement for public key cryptography since two of them have already been standardized by the NIST (see section \ref{sec:nist}) \cite{Alagic2022}. A lattice can be described as an infinite set of points in an $n$ dimensional space. The space generated by these points is a periodic structure, an example can be seen in figure \ref{img:lattice}. A lattice\,--\,the points in it\,--\,is generated by $n$ linearly independent vectors which can also be called a bases for the lattice \cite{Ajati1996}. Linearly independent vectors have the special property of not being a combination of any other vectors from the set of linearly independent vectors. An example of~these vectors is illustrated in figure \ref{img:lattice}. Vectors that generate a lattice can also be written in mathematical notation as
\begin{equation}
  \mathcal{L}(B)=(b_1,\dots,b_n)
\end{equation}
where $\mathcal{L}(B)$ denotes a lattice created by a basis $B$. The basis is created from vectors ($b_1,\dots,b_n$).

\object[0.5]{obr}{pictures/lattice.pdf}{2-dimensional lattice}{img:lattice}

To use lattices for cryptographic constructions, a vector $v$ in a lattice has to be defined with coordinates from the set of all integers $\mathbb{Z}$. If every coordinate is then reduced with the operation defined as
\begin{equation}
  v\equiv v\,\mathrm{mod}\,q
\end{equation}
where $q$ is also an integer from $\mathbb{Z}$, the lattice is then called a q-ary lattice \cite{Bernstein2009}.

A cryptographic construction in lattices additionally needs a mathematical problem to be defined that can easily be calculated given in input but difficult to invert and calculate back the input that was given, in other words, a one-way function has to exist. One-way functions may also be described as a computational problem. In~lattice-based cryptography there exist many computational problems, some of~them are \cite{Bernstein2009}
\begin{itemize}
  \item \textbf{SVP}\,--\,\acl{SVP},
  \item \textbf{CVP}\,--\,\acl{CVP},
  \item \textbf{LWE}\,--\,\acl{LWE}.
\end{itemize}
How these computational problems are used and in which cryptographic algorithms or cryptosystems will be described in the following sections.


\subsection{GGH public key cryptosystem}
\label{subsec:ggh}
As mentioned in Figure \ref{sec:key_agreement} KEM (\acl{KEM}) is one way of creating a KEP. In the case of lattice cryptography, an algorithm for creating a dedicated key exchange method like DH hasn't been found, so the only choice is to use a KEM. A KEM needs a public key encryption scheme to work, fortunately, many of them that use lattices have been discovered, so they can be used as key encapsulating mechanisms.

One of the first public key encryption schemes was the GGH cryptosystem which was named after its inventors Goldreich, Goldwasser and Halevi. Booth, the private and public keys are vector basis $B$ and $H$ respectively. A basis can also be written as a matrix where the columns of the matrix consist of the basis vectors. Additionally they form the same lattice $\mathcal{L}(B)=\mathcal{L}(H)$. The basis $B$ is a good lattice and generates orthogonal or nearly orthogonal vectors. Basis $H$ is called the bad basis and is derived from basis $B$ using a matrix $T$ where
\begin{equation}
  \begin{aligned}
    BT&=H, \\
    HT^{-1}&=B.
  \end{aligned}
\end{equation}
This transformation of $B$ into $H$ creates an orthogonality defect, which means the generated vectors by the basis are no longer orthogonal or close to orthogonal, this fact will be important later. The message to be encrypted is encoded into a vector $v$ which is a lattice point in. Next a small noise vector $e$ is chosen that is not a lattice point. Given these values the ciphertext $c$ can be computed with $c = Hv + e$. The vector $v$ or the plaintext can be extracted from $c$ given $v=T \round{B^{-1}c}$. The~rounding operation is very important here since it removes the~error that was added by vector~$e$. \cite{Bernstein2009}\cite{Goldreich1997}

\object[0.5]{obr}{pictures/lattice_ggh.pdf}{\acl{CVP}}{img:lattice_ggh}

Finding the original vector from the ciphertext is called the Closest Vector Problem (CVP) and is illustrated by Figure \ref{img:lattice_ggh}. The goal of this problem is to find the closest vector that is on a lattice point using a vector that isn't on a lattice point. The security of GGH relies on the fact that the CVP is easily computed while using the good basis $B$, but hard in the bad basis $H$. As mentioned earlier, a good basis is orthogonal and finding the closest vector is easily done using Babai's algorithm. However, this algorithm is inefficient in a basis that is not orthogonal, which in this case is the basis $H$. Based on this fact it can be assumed that only the owner of the good basis (private key) can decrypt a message. \cite{Goldreich1997}

The only problem with GGH is that for it to be secure enough, it needs to have very big keys and as a result the computations are too slow. That is why this algorithm can't be used in practice \cite{Bernstein2009}.


\subsection{NTRU and LWE public key cryptosystems}
\label{subsec:ntru_lwe}
Another post-quantum KEM scheme is NTRU or \acl{NTRU}. It is one of the most efficient public key encryption schemes since instead of~using a basis for its public key, it uses a polynomial consisting of $p$ coefficients
\begin{equation}
  h=h_0+h_1x+h_2x^2+\dots+h_{p-1}x^{p-1}.
\end{equation}
Like GGH this scheme is based on the CVP, so it uses a similar concept for the key exchange. Additionally, it generates the public key $h$ where it is very efficient when preforming required operations. The use of polynomials makes NTRU much faster than the GGH cryptosystem and was considered heavily for post-quantum standardization by NIST \cite{Bernstein149}.

LWE is not a cryptosystem by itself, but many cryptosystems are based on it. The problem is based on modular linear equations for example
\begin{align}
  3s_1+6s_2+7s_3+2s_4  & \equiv 10\,\mathrm{mod}\,11, \\
  10s_1+8s_2+3s_3+5s_4 & \equiv 1\,\mathrm{mod}\,11,  \\
  5s_1+s_2+7s_3+10s_4  & \equiv 8\,\mathrm{mod}\,11,  \\
  6s_1+8s_2+3s_3+4s_4  & \equiv 7\,\mathrm{mod}\,11,
\end{align}
where the goal is to find $s_1, s_2, s_3, s_4$. This is easily solvable even for big $n$ amount of equations with the Gaussian elimination, but if an error is added to the right side of each equation (-1 or +1)
\begin{align}
  3s_1+6s_2+7s_3+2s_4  & \equiv 9\,\mathrm{mod}\,11, \\
  10s_1+8s_2+3s_3+5s_4 & \equiv 2\,\mathrm{mod}\,11, \\
  5s_1+s_2+7s_3+10s_4  & \equiv 9\,\mathrm{mod}\,11, \\
  6s_1+8s_2+3s_3+4s_4  & \equiv 6\,\mathrm{mod}\,11,
\end{align}
for big $n$ it becomes a significantly harder problem. \cite{Regev2005}

Unlike other mentioned lattice-based post-quantum cryptosystems, LWE-based cryptosystems are supported by a theoretical proof of security \cite{Bernstein2009}. This makes them a very good candidate for standardization by NIST, more specifically the algorithm CRYSTALS-Kyber (see section \ref{sec:nist}).


\subsection{Digital signature schemes}
\label{subsec:lattice_digital_schemes}
As described in \ref{sec:asymmetric_enc}, digital signatures are another branch of public key cryptography. Many of the same cryptosystems used for public key encryption can be converted into digital signature schemes, for example booth GGH and NTRU. However the basic versions of these signature schemes have some security flaws that cause them to unusable in practice. \cite{Bernstein2009}

The situation with LWE-based cryptosystem is different. Digital signature schemes that use the LWE problem or a modified version of it called MLWE (\acl{MLWE}) are also very good candidates for standardization by NIST, for example the digital signature scheme CRYSTALS-Dilithium \cite{Grimes2020}. More information on the topic of standardization can be found in section \ref{sec:nist}.


\section{Code-based cryptography}
\label{sec:code_based_crypto}
This family of post-quantum cryptography utilizes error correction codes. These are codes that can either detect or correct an error in some binary string by adding additional bits. However they can correct/detect an error up to a threshold. If an big enough error is introduced, the code may no longer be able to detect or correct it.  

The first ever code-based scheme was introduced by Robert J. McEliece in 1978 and so the scheme got the name from its inventor McEliece. The private key is defined as a random \textit{Goppa code} which is able to correct errors in a coded sequence of bits. The public key is a matrix $G$ and the plaintext $m$ is a bit string. Additionally another bit string is randomly created called $e$. The ciphertext $c$ is then calculated with
\begin{equation}
  c=mG+e.
\end{equation}
Only the owner of the aforementioned \textit{Goppa code} can extract $m$ and $e$ from $c$ since the code was designed to efficiently correct errors added by the bit string $e$. \cite{Bernstein149}

Since this cryptosystem was introduced in 1978, it is well understood and has never been successfully broken. However for the system to be secure the private/public keys have to be relatively large compared to the keys of modern cryptography like ECDSA. On the other hand they are very fast compared to the other algorithms submitted to the NIST standardization process. That is why 3 code-based algorithms are still being considered in the 4th round. \cite{Chen2016}


\section{Hash-based cryptography}
\label{sec:hash_based_crypto}
Hash-based cryptography is mainly used for post-quantum digital signatures. Any hash function can be used inside a hash-based cipher as long as they are collision resistant. That means they don't rely on any hard mathematical problems, which makes their security requirement very low. Also because of this fact, every hash-based cipher can have many alternatives using many different hash functions. \cite{Bernstein2009}

The first hash-based scheme was proposed by Leslie Lamport in 1975. If the output of the chosen hash function $h$ is 256 bits, the private key $x$ consists of 256 pairs of random bit strings, where each string is 256 bits long. The public key $y$ is then generated by hashing every random bit string. At this point the public and private keys are
\begin{align}
  x & =(x_{0,0}, x_{0,1}, x_{1,0}, x_{1,1},\dots,x_{256, 0},x_{256, 1}),                   \\
  y & =(h(x_{0,0}), h(x_{0,1}), h(x_{1,0}), h(x_{1,1}),\dots,h(x_{256, 0}),h(x_{256, 1})).
\end{align}
Given a message $m$ which at first is hashed, the signature $\sigma$ consists of either $x_{0,0}$ if the first bit of the message hashes its 0 or $x_{0,1}$ if its 1. This repeats for every bit of the hashed message. The resulting signature for a given message is
\begin{align}
  h(m)   & =(01\dots0)^{256},                           \\
  \sigma & =(x_{0,0}, x_{1,1}, \dots, x_{256,0})^{256}.
\end{align}
The verifier also hashes the message, generating the same hash. He then chooses hash values from the public key depending on the bit string of the message hash and creates $y_p$. Then he hashes each hash of the signature and generates $h(\sigma)$, which are
\begin{align}
  h(\sigma) & =(h(x_{0,0}), h(x_{1,1}), \dots, h(x_{256,0}))^{256}, \\
  y_p       & =(h(x_{0,0}), h(x_{1,1}), \dots, h(x_{256,0}))^{256}.
\end{align}
If $h(\sigma)=y_p$ then the signature is verified. However, the signer cannot reuse the same private key since it was already used. That is why this algorithm is called Lamport's one-time signature. \cite{Bernstein149}

The solution to one-time hash-based signatures was introduced by Ralph Merkle in 1979 and is called Merkle's tree signature scheme. Key generation starts with generating a binary tree which always has $2^n$ leaves, this is the master private key. Each leaf corresponds to a hash of Lamport's public key. Every two neighboring nodes are hashed together to create their parent node. In figure \ref{img:merkel_tree} an example can be given for $N_1$ and $N_2$ where
\begin{equation}
  N_3=h(N_1, N_2).
\end{equation}
\object[1]{obr}{pictures/merkel-tree.pdf}{Merkel tree}{img:merkel_tree}
\noindent After the tree is generated the master public key is the root of the tree, in this case, $N_{15}$. To sign a message, the signer chooses some random leaf node and signs a message with Lamport's one-time signature using the corresponding private key. In this case $N_5$. To verify this signature, at first, the one-time signature is verified using Lamport's public key. The signature is verified but the public key also has to be verified. That is why the signer also sent the least amount of hashes needed to compute the root. In this case he sent $N_4$, $N_3$, $N_{14}$ as seen in figure \ref{img:merkel_tree_path}. To calculate $N_{15}$, the verifier has to calculate
\begin{align}
  N_6    & =h(N_4,N_5)     \\
  N_7    & =h(N_6,N_3)     \\
  N_{15} & =h(N_{14},N_7).
\end{align}
If the calculated $N_{15}$ equals the master public key, the signature is verified. \cite{Bernstein2009}
\object[1]{obr}{pictures/merkel-tree-path.pdf}{Merkel tree -- signature verification}{img:merkel_tree_path}

As mentioned earlier, the security requirements for hash-based ciphers are very low and the principles that they are based on are very well understood. This makes them excellent candidates in the NIST standardization process. However, one flaw of these ciphers is that the signer has to keep a record of previously signed messages because they can produce only a limited amount of signatures. Although one signature has been standardized by NIST in the 3rd round and that is SPHINCS+, which is based on the aforementioned Merkle's tree signature.


\section{NIST Standardization}
\label{sec:nist}
NIST (\acl{NIST}) started a standardization process in 2017 for the field of post-quantum public-key cryptography. The first call for submissions was initiated in December of 2016, where 69 post-quantum algorithms were accepted into the first round of standardization. As of writing this thesis the latest round -- the third round -- ended on July 2022. The result was a standardization of 1 KEM protocol and 3 Digital signature protocols, for more details refer to~table \ref{tab:pqc_standardized}. Additionally 4 more post-quantum algorithms will be advancing to the fourth round of standardization, more information can be found in table \ref{tab:pqc_forth_round}. \cite{Alagic2022}

\object{tab}{tables/pq_standardized}{Standardized post-quantum algorithms \cite{Alagic2022}}{tab:pqc_standardized}
\object{tab}{tables/pq_fourth_round}{Forth round submissions \cite{Alagic2022}}{tab:pqc_forth_round}


\section{Disadvantages of post-quantum cryptography}
\label{sec:disadvantages}
Replacing modern cryptography like RSA with post-quantum algorithms is not as easy as it might seems. Quantum resistant algorithms are certainly needed to prepare for the thread for quantum computers but one big disadvantages these algorithms compared to modern cryptography is their computational requirements. Since they use more complicated structures and principles they also require more memory and processing power to compute. Some embedded devices might even take too long to compute some post-quantum algorithm to be useful or might just fail since they don't have enough memory. Another problem is the key sizes of these algorithms. They are a lot bigger as can be seen table \ref{tab:key_sizes} compared to modern cryptography. The chosen security parameters for the algorithms mention in the table bellow correspond to the NIST security level of 3. Level 3 is defined as a security level which is only breakable by an attack that is able to break the AES algorithm with a key size of 192 bits or less \cite{8lV5dQrQyshiCp3i}.  Again some embedded devices have a very limited network bandwidth  because they are battery powered. These are the reasons why adapting post-quantum cryptography might not be as seamless as it might seem.

\object{tab}{tables/key_sizes}{Key size comparisons \cite{Barker2020}\cite{YbbuGxVPF0GGTxfN}\cite{y0VQZiTmHEg2xvPn}}{tab:key_sizes}


\chapter{Network basics}
\label{ch:network_baics}
In order to fully understand how any application that creates a communication channel between two entites works, it is important to firtly look at the concepts of network basics. The contents of this chapter will focus on the topis such as the TCP/IP protocol suite, TCP (\acl{TCP}), UDP (\acl{UDP}) protocols and differenes between client-server and peer-to-peer communications.

The TCP/IP protocol suit consists of 5 layers as can be seed in table \ref{tab:tcp_ip_table}. Each layer is defined by one or more protocols. A protocol defines strict rules for what, how and when should an entity communicate. Each row in the TCP/IP suite contains atleast one protocol, which again dictates how the communication should proceed on that layer. A protocol layer communicates horizontally with other protocol layers using PDUs (\acl{PDU}), each PDU is either encapsulated or deencapsulated into another PDU, depending on way the data is flowing through the TCP/IP layers. During this process a new header is added or removed. A header contains important information fot that specific layer. For example the IP address is contained in the header for the network layer. The layers also communicate vertically using either physical channels (physical and data link layers) or virtual channels (all other layers). Psysical channels are created between a physical medium by whitch the bits travels through, virtual channels on the other hand are created between applications or more complex diveces like routers. Channels on each layer use a different identifier to differencieate between them. Aformentioned information for each layer can be found in table \ref{tab:tcp_ip_table}.

\object{tab}{tables/tcp_ip_details}{TCP/IP protocol suite}{tab:tcp_ip_table}


% Describe network basics such as TCP 3 way handshake, UDP and so on.

TODO: cite


\section{TCP and UDP protocols}
\label{sec:tcp_and_udp}
TCP and UDP work on the transport layer in the TCP/IP protocol suite. The transport layer is responisble for creating connections between applications, where each applications is identified with a port number. A port number can be any number in the range of 0-65535. For an application to be avaliable it must listen on a port number so a client knows where to send his data. Similiary if a client connects to an application he is also given a port number so that the server knows where to send this data. Booth TCP and UDP work with port numbers but an application is able to listen on the same port for booth TCP and UDP protocols.

The UDP protocol was designed to be fast and unreliable. It posses these properties because it is a connectionless protocol. That means that there is no guarantee that the data that is being transfered will arive as intended and without erros. It also means that the data can be sent faster and has less overhead communication compoared to TCP. This model fits very well while sending very small amounts of data very quickly, like in the case of DNS translation. Before any data can be transfered, it has to be split into datagrams of smaller size. Usuall a UDP communication is very simple and might look something like this:

TODO UDP COMMUNICATION

The TCP protocol on the other hand is a connection-oriented protocol. Before data can transfered between two entities first a connection has to be established using the three-way handshake (TODO: ref). It works by setting one bit flags in the TCP header, in this case the SYN and ACK flags. After a connection has been established, the data transfer can begin. Unlike UDP, TCP is also numbering its segments which means it is able to detect if a segment was lost while being transfered, and then it can try to transfer it again. Another feature of TCP is flow control, which can be used for controlling how much data the communication entites can exchange at one time. All of these features bring a much bigger overhead to each segment since more information needs to be tracked. This takes a toll on how fast segments can be trasfered and also increases the size of the segements, which results in a slower and more reliable protocol then UDP. An example of good usage for the TCP is the HTTP protocol where a website needs to transfered exactly as intended without any errors.

TODO TCP COMMUNICATION

TODO CITE


\section{Communication paradigms}
\label{sec:comm_paradigms}
Two of the most used communication paradigms to provide services to users are client-server and peer-to-peer, where the the former is more commonly seen in the internet. As the name implies a service is hosted by a server, a client or more clients can connect to this server to consume the hosted service. Most of the time the server is a more powerful computer system, so that it can handle more requests at the same time. Services can be provided on individual basis, for example a simple website that provides HTML content to the user. It can also provide a connection between two users, so that they can communicate. For this to work the server has to create multiple connections, one with each of the users. If the communication is encrypted, the server has to first decrypt an encrypted message, read it, encrypt it again and send it to the second user. The content of the exchanged messages were also seen by the server, which implies the users have to trust the server to not store or log their exchanged messages anywhere.

The second communication paradigms, peer-to-peer relies on the fact that if two entities want to communicate they will create a connection between them only which eliminates the problem with the client-server paradigm connecting two users. Each entity consists of a server and a client also since each entity needs to able to listen for incoming connections and also accept create connections to other listening entities. 

TODO: finish


\section{End-to-End Encryption}
\label{sec:e2e}
E2EE (\acl{E2EE}) is a concept that allows data to be transmitted from one end user to another end user without being revealed or being tampered with along the way. It is mostly used in messaging apps. It is very easy to implement using the peer-to-peer parading. Before communication starts, each of the end users converges on some secret key using some key exchange protocol. One user uses this key to encrypt the data and the other user can then decrypt it. Of course in a real-life scenario where third parties try to attack this communication a lot of other things have to be considered like the integrity of the data and authenticity of the users.

However, in the client-server paradigm, it's a bit trickier to implement E2EE, since there is always some other entity between two end users. If the middle entity is a malicious one, it could easily use a man-in-the-middle attack on the key exchange protocol. Despite this, there are some protocols that provide E2EE that ensure no third entity can utilize such an attack. One such example is the Signal protocol which uses the Double Ratchet Algorithm \cite{Marlinspike2016}.


\chapter{Application introduction}
\label{ch:app_intro}
One of the goals of this thesis is to create an application capable of exchanging messages or files between two users, secured only using quantum-resistant cryptography. The graphical user interface is only implemented in the console environment. To exchange messages the application uses the peer-to-peer paradigm. However, there is still one peer called the user and one peer called the server to differentiate between the two peers. That is why in some places like the implementation code, client and server are used instead of peers. It is also important to mention that the application uses TCP as its transport layer protocol. It is implemented in the Go programming language. As the public key algorithms, CRYSTALS-Kyber and CRYSTALS-Dilithium are chosen. AES is chosen as the symmetric key encryption algorithm. Additional sections in this chapter explain why these choices were made and shortly introduce the chosen tools. The keyword \textit{modular} in the thesis name signifies algorithm modularity in its implementation (explained further in subsection \ref{subsec:modular_app}).


\section{Philosophy behind the application}
\label{sec:philo_behind_app}
The main philosophy of this application was inspired by a network protocol called Wireguard \cite{Donenfeld2020} and a book The Art of UNIX Programming \cite{Raymond2003}. It is meant to be a simple and small application for sending arbitrary amounts of data using only post-quantum cryptography. The four main ideas that are incorporated into the application are
\begin{itemize}
  \item small traffic overhead,
  \item text-based interface,
  \item modularity,
  \item extensibility.
\end{itemize}
These ideas are introduced in the following sections.


\subsection{Small traffic overhead}
\label{subsec:network_footprint}
To achieve a small traffic overhead, this application uses the smallest header possible. The header only contains the data size and type. For a more detailed description of the header see chapter \ref{ch:network_com_sec}. This makes the traffic overhead very low. Another mechanism that reduces traffic overhead is the number of initialization messages at the start of the connection. For example, the protocol TLSv1.2 uses handshake messages to negotiate on a cipher suite and then create a shared key using some key exchange protocol \cite{Ristic2014}. The protocol created in this application uses a pre-configured cryptography suite and doesn't have to spend time negotiating it like TLS and only sends necessary messages to establish a secret key. This results in the protocol needing 1 round trip to initialize a connection. Again more information on the protocol flow can be found in chapter \ref{ch:network_com_sec}.

However, one flaw is that this application uses TCP. To minimize traffic overhead UDP would be a better fit since it doesn't require connection initialization via the three-way handshake. It would also mean that things like flow control, error handling, synchronization and more would have to be solved in the application layer. TCP was chosen over UDP because it has been designed to do these things well and efficiently. As a result, the application needs 2 round trips to initialize a connection including the TCP handshake.


\subsection{Text based interface}
\label{subsec:text_based_interface}
From the start this application was designed to be mostly used in UNIX-like systems (but also supports other platforms like Windows). Most of UNIX programs and applications use a text based interface, this way its very easy to redirect an output of one program to the input of another program and vice versa. That is why this application is also designed to use the universal text interface so that it can easily interact with other UNIX programs. The data to send can be redirected via a pipe into the application. Received data can be saved to a file via redirecting its output. For more information about the capabilities of this application refer to chapter \ref{ch:app_capab}


\subsection{Modularity}
\label{subsec:modular_app}
Modularity in this case means that any other post-quantum algorithm can be easily plugged in to the application without any massive code code changes. This allows anyone to just add their preferred algorithm to the application in a relatively short time and without the knowledge of the whole code base. How the an actual algorithm can be added is detailed in section \ref{sec:algs_modularity}.

\subsection{Extensibility}
\label{subsec:extensible_app}
This application is made to be extensible, which means it is very easy to use the underlying communication protocol for sending arbitrary amounts of data and extending it with additional functionality. For example, if another programmer wants to implement a new GUI (\acl{GUI}), the application was programmed in a way where it's easy to use the underlying secure communication protocol. The programmer doesn't have to understand how the protocol works, he just needs to know how to use it. More specifically it uses a Go feature called channels for sending and receiving data. When data is sent to the channel it is automatically encrypted and similarly, if some data is received from a channel it is already verified and decrypted.


\section{The Go programming language}
\label{sec:go_lang}
The first iteration of the Go programming language was created at Google. It~is an~open-source programming language and has many similarities with C. That means it is a compiled language and a statically typed language \cite{Donovan2016}. In a statically typed language, a variable has to have a type assigned to it before the compilation process. As can be seen in Figure \ref{img:compiler} a compiler translates the source program into an executable, which can run multiple times without the need to compile again \cite{Aho2006}. This makes Go faster than most of the interpreted languages like Python since an~interpreter needs to translate the source code every time it has to run.

\object[0.89]{obr}{pictures/compiler.pdf}{Compiler}{img:compiler}

However unlike C, it has a garbage collector, which means it is capable of automatic memory management \cite{Donovan2016}. In C a programmer has to manage memory on his own, allocate and free it by using functions. Go and its garbage collector takes care of allocating and freeing memory which makes it a lot less error-prone when it comes to memory management. Together with good overall performance, Go was also designed to make high-performance network applications that's why it was chosen for this thesis.


\section{Choice of cryptography algorithms}
\label{sec:implementations}
As mentioned in \ref{sec:nist} a group of three lattice-based algorithms has been standardized, of which Kyber and Dilithium are a part. That is the main reason why they are chosen as the public key algorithms for this thesis. Each of them has its respective chapter (\ref{ch:kyber} and \ref{ch:dilithium}) which describes some implementation aspects and also introduces some theoretical background that makes them secure.
\object{tab}{tables/choose_algs}{Choosen algorithms}{tab:choosen_algs}

As for symmetric cryptography, AES is chosen with a key size of 256\,bits. AES is used in modern cryptography as the main symmetric cipher and is also usable in post-quantum cryptography but requires a bigger sized keys, due to the threat of Grover's algorithm. Used hash functions in this application are SHAKE-128/256 for the implementation of Kyber and Dilithium following the author's recommendations. SHA3-512 is used by the networking part of the application. A summary of the chosen cryptographic algorithms can be found in the table below (table \ref{tab:choosen_algs}).


\chapter{CRYSTALS-Kyber}
\label{ch:kyber}
CRYSTALS (\acl{CRYSTALS}) Kyber is a~quantum resistant KEM (\acl{KEM}) standardized in the 3rd round of the NIST standardization process. It is based on modified version the LWE problem (section \ref{subsec:ntru_lwe}) called MLWE (\acl{MLWE}). Lattices have a solid theoretical security foundation because they have been researched for a long time and are not a new invention. NIST has concluded in their report of in their 3rd round of standardization that Kyber has sufficient security against quantum computer attacks. Even in the worst case scenario where the development of quantum computers is underestimated. As to performance it has been shown that Kyber is the fastest algorithm amongst the other lattice KEM NIST finalists when it comes to key generation, encapsulation and decapsulation in software and hardware. \cite{Grimes2020}\cite{Alagic2022}



\section{Implementing Kyber}
\label{sec:implementing_kyber}
The implementation of Kyber in this thesis is done using only the standard Go library except for one external library \cite{00fV2cvg7Z6H2tS3} which is required for the implementations of the hash functions \texttt{SHAKE-128/256} (refer to section \ref{sec:hash_functions} for hash functions). Figure \ref{img:kyber_all} illustrates a very simplified block diagram of how Kyber works. The small letter or number at the start of an arrow denotes the size of the object in polynomials. So for example the letter $k$ denotes that an object consists of $k$ polynomials. At first the public and secret keys need to be generated. A~random message $m$ is then encrypted by one communicating entity using the~public key. The encrypted message is then decrypted by the other communicating entity and $m$ becomes the shared key. The following subsections will explain each sub-algorithm of the block diagram in more detail. The Go code in the practical part is also a great reference to understand how Kyber works.

\object{tab}{tables/kyber_security_levels}{Kyber security levels \cite{YbbuGxVPF0GGTxfN}}{tab:kyber_sec_levels}

Kyber uses a set of parameters to define its security level, of which it has three as seen in table \ref{tab:kyber_sec_levels}. Kyber in this thesis is implemented for all the parameter levels. What individual parameters mean will be explained in further subchapters.


\section{Theoretical background}
\label{sec:kyber_theroteical}
Kyber uses a structure called rings, more specifically the ring $R_q$ denoted as
\begin{equation}
  \mathbb{Z}_q[X]/(X^n+1).
\end{equation}
A ring contains a polynomial of $n$ elements, where the coefficients of this polynomial are integers reduced modulo $q$ and the powers of the polynomial are reduced $(X^n+1)$. The parameters $n$ and $q$ are defined in table \ref{tab:kyber_sec_levels}. An example of a~polynomial with elements from the ring $R_q$
\begin{equation}
  t_1 = 1564 + 2189x + 258x^2 + \dots + 655x^{n-2} + 2587x^{n-1}.
\end{equation}
A vector of size $k$ consists of $k$ polynomials with coefficients from the ring $R_q$. The~parameter $k$ can again be found in table \ref{tab:kyber_sec_levels}. The the polynomial $t_1$ from the~previous example together with a new polynomial $t_2$
\begin{equation}
  t_2 = 2408 + 1932x + 420x^2 + \dots + 3256^{n-2} + 2399^{n-1},
\end{equation}
form a vector of polynomials $\mathrm{T}=(t_1, t_2)$. A matrix of size $k\times k$ consists of $k^2$ polynomials from the ring $R_q$ aligned as a square 2-dimensional matrix. \cite{YbbuGxVPF0GGTxfN}

The addition of elements from a ring is just adding the individual polynomials and is relatively fast. Multiplication of vectors or matrices the usual way (multiplying each element by each element of the other polynomial) is computationally much more demanding with big $n$. In this case where $n=256$ the number of computations would be $n^2=262144$. A more efficient way to calculate the multiple of two polynomials is using an NTT (\acl{NTT}) where the number of operations is only $n\,\mathrm{log}(n)=1387$. This transformation is a more specific version of the FFT (\acl{FFT}). However, before doing the NTT multiplication it is first required to transform the polynomial into NTT form. Do the calculation with some other polynomial in NTT form and then do the inverse NTT transformation on the result. In this thesis structures that are converted to NTT are denoted with a hat, for example, \rmhat{A}. \cite{Liang2021}


\section{Encoding, Compression and randomness}
\label{sec:kyber_enc_compr}
In order to transfer polynomials over the network they need to be serialized into bytes. Kyber defines two functions for this purpose:
\begin{itemize}
  \item \texttt{encode}($p$,\,$l$)\,--\,convert a polynomial \texttt{poly} into $32*l$ bytes,
  \item \texttt{decode}(B,\,$l$)\,--\,convert $32*l$ bytes into a polynomial.
\end{itemize}
In the following code listings functions like \texttt{encodePolyVec()} exist. It encodes every polynomial in the vector. The same applies for the decode functionality.

Another feature of Kyber is the compression of polynomials that are encoded. Due to the fact that Kyber is based on LWE, the calculations don't need exact numbers to be correct. This is why a compression mechanism that discards some low-order bits from encoded polynomials can be introduced. Two more functions are defined by kyber for compressing and decompressing bytes:
\begin{itemize}
  \item \texttt{compress}($x$,\,$d$)\,--\,compress a number into the range of $\{0,\dots,2^d-1\}$,
  \item \texttt{decompress}($x$,\,$d$)\,--\,decompress a number while loosing some low-order bits.
\end{itemize}
Similarly to the encoding/decoding function theses functions can be applied for every coefficient of a polynomial.

Whenever it mentioned that a random polynomial or a polynomial vector has been generated it is implied that a CBD (\acl{CBD}) is used with a parameter either $\eta_1$ or $\eta_2$.

\section{Key generation}
\label{sec:kyber_keygen}
The key generation functions starts with generating random parameters (illustrated by \ref{img:cpapke_keygen}). A random seed $\rho$ is used to generate the matrix \rmhat{A}. It is publicly known to everyone and needs to be shared, however since the function that generates it is deterministic only $\rho$ needs to be shared instead of the whole matrix. This is mechanism saves a lot of network traffic because the matrix \rmhat{A} would consume a lot more network traffic then just sending $\rho$. Two vectors $s$ and $e$ are generated from a different random seed. However in this case the seed is not shared since $s$ and $e$ need to remain secret. After transforming the generated vectors into the NTT domain, \rmhat{A} and \rmhat{s} are multiplied. The vector \rmhat{e} is then added to the result and creates the public key. The encoded vector $s$ is then used as the private key.

\object[0.5]{obr}{pictures/kyber_keygen.pdf}{Kyber key generation}{img:cpapke_keygen}


\section{Encapsulation}
\label{sec:kyber_enc}
The encapsulation process relies on the encryption function (figure \ref{img:cpapke_enc}) and will be explained further. Firstly the parameters have to be set up. The public key is decoded into \rmhat{t}. The matrix \rmhat{A} is generated from $\rho$ which is also a part of the public key. A random polynomial vector \rmhat{r} is created and transformed into the NTT domain. Parameters $e_1$ and $e_2$ are also randomly generated where the~first one is another polynomial vector and $e_2$ is just a single polynomial. The last required parameter is the message $m$.

\rmhat{A} and \rmhat{r} are multiplied and $e_1$ is added to the result. Afterward, it is transformed from the NTT domain since booth factors are in the NTT domain. The result of these operations is the $u$ which forms a part of the ciphertext. The message $m$ is decoded to create a~polynomial from it and decompressed. A polynomial is calculated using the factor of \rmhat{t} and \rmhat{r} and is again similarly transformed from the NTT domain. Then the polynomial $e_2$ is added to it together with the decoded message. To create the ciphertext booth $u$ and $v$ are compressed and encoded to get them ready for network transfer. The parameters used in the compression and encoding processes are $d_u$ and $d_v$ reference in table \ref{tab:kyber_sec_levels}.

\object[0.5]{obr}{pictures/kyber_enc.pdf}{Kyber encryption function}{img:cpapke_enc}

As a final step, a random key is generated and encrypted using the above-defined encryption function. The resulting ciphertext is sent over to be decrypted.


\section{Decapsulation}
\label{sec:kyber_dec}
Similarly, as with the encapsulation process, the decapsulation process requires the~decryption function to be defined. The decryption process contains only a few calculations and is illustrated by a~figure, specifically \ref{img:cpapke_dec}. It begins with decoding and decompressing the parameters $u$ and $v$ from the ciphertext. Additionally, $u$ is transformed into the NTT domain. The private key is also decoded into the vector \rmhat{s}. The actual decryption begins by multiplying \rmhat{s} and \rmhat{u} and transforming the~product from the NTT domain. It is then subtracted from $v$ compressed and decoded to get the original message $m$. After the message is decrypted it can be used to generate the same key $K$ that will be used further.
\clearpage
\object[0.5]{obr}{pictures/kyber_dec.pdf}{Kyber decryption algorithm}{img:cpapke_dec}


\chapter{CRYSTALS-Dilithium}
\label{ch:dilithium}
Another algorithm from the group of lattice-based cryptography is the CRYSTALS-Dilithium signature scheme. It was also standardized during the 3rd round of the NIST standardization process on post-quantum cryptography. It is based on the Fiat-Shamir paradigm which means a prover can convince a verifier of the fact that they hold a secret key without actually revealing it. Similarly, Kyber is also based on the MLWE problem. Dilithium also has a binding property that allows a signature to be linked with a unique public key and a message. When it comes to the security of Dilithium, it is proven that a signature is unforgeable by classical and quantum computers. NIST mentioned in their report on the 3rd round of standardization that Dilithium has a strong security basis and along with Falcon is one of the most efficient signature algorithms. \cite{Alagic2022}


\section{Implementing Dilithium}
\label{sec:implementing_dil}
As with Kyber, Dilithium is implemented using only the standard go libraries and one external library \cite{00fV2cvg7Z6H2tS3} that contains implementations for SHAKE-128 and SHAKE-256 hash functions. Dilithium can be implemented in two ways, the first one is by using a bigger public key. This implementation of Dilithium is also simpler overall. The other option is implementing a more complex algorithm that has a~smaller public key by a factor of more than half. For this thesis, a more complex implementation was chosen. How this alternative differs from the simpler one will be explained in subsection \ref{subsec:dil_reducing_pub_key}. The algorithms as a whole is described in figure \ref{img:dil_all}. Analogous to the Kyber algorithm figure each square represents a mathematical structure or a program variable, where the structures are mostly represent a vector of polynomials. The small letters at the beginning of arrows denote the number of polynomials that the resulting structure consists of.

The process of signing in Dilithium follows a well defined order as with many other digital signatures. Firstly the public/private keys are generated and the private key is used in~the signing process. This is key is not shared and kept secret by the signer. The result of a signing process is a signature which can be verified by anyone who owns the~related public key. Since the public key is shared, it is not kept secret by the signer. The following sections will explain all of these steps in~more detail. For an even more detailed description of Dilithium, check the algorithm implementation in practical part of this thesis.

\object{tab}{tables/dil_security_levels}{Dilithium security levels \cite{y0VQZiTmHEg2xvPn}}{tab:dil_sec_levels}

Table \ref{tab:dil_sec_levels} displays the individual parameters for each of the Dilithium parameter sets. The implementation in this thesis contains all Dilithium security modes. When a parameter is relevant to the process being explained it will be mentioned and explained in that scenario instead of all the parameters explained in this section.


\section{Bit manipulation}
\label{sec:dil_bit_man}
Dilithium employs some helper functions which are used in both the simple and more complex versions of Dilithium. The first one is \texttt{Decompose} and can be well explained using an example
\begin{equation}
  \label{eq:decompose}
  \mathrm{\texttt{Decompose}}(5687946,\,1735)=3278*1735+616.
\end{equation}
\noindent As can be see in equation \ref{eq:decompose} the \texttt{Decompose} function splits a number into two smaller numbers $r_1=3278$ and $r_0=616$. The number $r_1$ is the closest multiple of the second input parameter $\alpha=1735$ to the input number. The second returned number $r_0$ is what remains after the division of $\alpha$. This function is wrapped by two additional functions \texttt{HighBits} and \texttt{LowBits}. \texttt{LowBits} returns only $r_0$ and \texttt{HighBits} returns only $r_1$.

A similar function \texttt{Power2Round} does basically the same but instead of taking any $\alpha$ as the divisor it takes a parameter $d$ which is then used for calculating a~power of 2 that is used as the divisor. Booth function output the same number for parameters $d=13$ and $\alpha=8192$ as seen bellow

\begin{align}
  \mathrm{\texttt{Power2Round}}(5687946,\,13\mathrm)&=694*8192+2698, \\
  \mathrm{\texttt{Decompose}}(5687946,\,8192\mathrm)&=694*8192+2698.
\end{align}

Functions \texttt{MakeHint} and \texttt{UseHint} make use of the aforementioned functions to create and consume hints. The implementation for \texttt{MakeHint} is relatively simple and can be seen in listing \ref{lst:make_hint}. It returns true if the number $z+r_0$ is bigger than $\alpha$ because $r_1$ would either increase or decrease at least by one if that was the case. If the number $z$ doesn't change the high bits of $r$ it returns false which means that $z$ doesn't affect the high bits of $r$.
\listing{text/code/dilithium.go}{\texttt{MakeHint} implementation}{lst:make_hint}{131}{131}
\noindent\texttt{UseHint} takes $h$, $r$ and $\alpha$ as parameters. It can be used to calculate the high bits of $r+z$ without the knowledge of $z$ and using only the hint $h$ as seen in equation \ref{eq:use_hint}. The function first decomposes $r$ and if the hint is true it will either add or subtract 1 from $r_1$ (high bits of $r$) depending on the sign of $r_0$.

\begin{equation}
  \label{eq:use_hint}
  \mathrm{\texttt{UseHint}}(\mathrm{\texttt{MakeHint}}(z,\,r,\,\alpha),\,r,\,\alpha)=\mathrm{\texttt{HighBits}}(r+z,\,\alpha)
\end{equation}


\section{Theoretical basics and bit packing}
\label{sec:dil_bit_pack}
Dilithium uses the same theoretical background as Kyber (see section \ref{sec:kyber_theroteical}) which includes rings, NTT transformation, polynomials and even uses the same $n$ as can be seen in table \ref{tab:dil_sec_levels}. However, the parameter $q$ is different.

\object[0.8]{obr}{pictures/bit_packing.pdf}{Bit packing for vectors $s_1$ and $s_2$}{img:bit_packing}

Since Dilithium needs to transfer polynomials over the network a very efficient bit-packing method can be used. For example the polynomial vectors $s_1$ and $s_2$ consist of values that are from the interval $\{-2,\,-1\,,0,\,1\,,2\}$ while using $\eta=2$. This means only 3 bits are required to pack a single coefficient into bit form (illustrated in figure \ref{img:bit_packing}). However, the coefficients firstly need to be mapped into an interval $\{0,\,1\,,2,\,3\,,4\}$ while packing and moved back to $\{-2,\,-1\,,0,\,1\,,2\}$ in the unpacking process. As a result one polynomial of the mentioned vectors only takes up 96\,B and the whole polynomial vector only takes up 384\,B. This is a big difference compared to simple packing where one byte contains one value. A very similar process is used for packing other polynomial vectors in Dilithium, the only difference being the size of the coefficient interval. Using a~slightly different packing method for each kind of coefficient interval is what makes the bit packing/unpacking a very efficient method for encoding/decoding data that has to be sent over a network. This method is also used when a vector needs to be consumed by a hash function since it can only accept a byte array as its input.


\section{Key generation}
\label{sec:dil_keygen}
The dilithium key generation process starts by generating random seeds $\rho$ and $\rho'$. $\rho$ is used for generating the matrix \rmhat{A} where its dimension are $k\times l$. $\rho'$ for generating error vectors $s_1$, $s_2$. The range of values in these vectors depends on the parameter $\eta$. Similarly as with Kyber only $\rho$ is sent over the network since the function to generated \rmhat{A} is deterministic. The product of \rmhat{A} and $s_1$ to which $s_2$ is added is passed to the aforementioned function \texttt{Power2Round} together with $d$ which is a parameter defined \ref{tab:dil_sec_levels}. This functions splits $t$ into $t_1$ and $t_0$. This is the splitting which is talked out about in subsection \ref{subsec:dil_reducing_pub_key}.

Variable $t_1$ is used as the public key together with the randomly generated $K$. The private key consists of a hash (a block with H) of the public key $tr$, private parameters $s_1$, $s_2$, $t_0$ and the random seed $\rho$. Polynomial vectors are additionally packed into bytes for easy transfer over the network. This process is also illustrated in figure \ref{img:dil_keygen}.

\object[0.5]{obr}{pictures/dil_key_gen.pdf}{Dilithium key generation}{img:dil_keygen}


\section{Signature creation}
\label{sec:dil_sign}
At the beginning of the Dilithium signing process, the private key has to be parsed into variables that it consists of. This is done by unpacking the bytes into useful data (see section \ref{sec:dil_bit_pack}), more specifically the vectors $s_1$, $s_2$, $tr$ and $t_0$. The vectors $s_1$, $s_2$ need to be converted to the NTT form but this can be precomputed ahead of time to increase the speed of singing. The parameter \rmhat{A} is generated from the shared seed $\rho$. The message to be signed is hashed together with $tr$ which is used for generating the vector $y$. However, for the sake of simplicity and clarity, this generation process is not described in \ref{img:dil_sign} and the parameter $y$ is just shown as one of the inputs. Next the product of \rmhat{A} and \rmhat{y} which is denoted $w$ is calculated. High bits of $w$ (\texttt{hb(k)} in figure \ref{img:dil_sign}) are hashed together with the hash used for generating $y$, to create the polynomial $c$. It is first used for multiplying vectors $s_1$ and $s_2$. After that is used as a part of the signature. The vector $y$ is added to $cs_1$ ($cs_1$ is $s_1$ scaled by $c$) and creates another part of the signature, the vector $z$. The subtraction of the vectors $w$ and $cs_2$ (vector $r$) is added together with the scaled vector $ct_0$. The result of this process is used as the second input for the function \texttt{MakeHint}. The first input is $ct_0$ but negated. \texttt{MakeHint} functions return the final value for the signature. These hints will then be used to calculate the missing part of the public key as described in \ref{subsec:dil_reducing_pub_key}.

However, before the signature creation is finalized, a few conditions have to be met. If these conditions are not met, most of the signing process is repeated. Some of the conditions mention parameters defined in table \ref{tab:dil_sec_levels}. These conditions are
\begin{itemize}
  \item the polynomial coefficients in $z$ can't be bigger than $\gamma_1-\beta$,
  \item the polynomial coefficients in $r_0$ (\texttt{LowBits} of $r$) can't be bigger then $\gamma_2-\beta$,
  \item the polynomial coefficients in $ct_0$ can't be bigger then $\gamma_2$,
  \item number of created hints can't be more than $\omega$.
\end{itemize}


\section{Signature verification}
\label{sec:dil_verify}
The verification process for Dilithium starts off by unpacking required variables from the public key and the signature. \rmhat{A} is generated the same way as in \ref{lst:dil_keygen_1} line 9, it is then multiplied by the vector $z$. The public parameter $t_1$ is then at first scaled by $2^d$ to make up for the lower bits taken out by the \texttt{power2Round} function. It is then again scaled by $c$ which is parsed from the signature.
\listing{text/code/dilithium.go}{Variable preperation}{lst:dil_verify_1}{117}{121}
\object[0.5]{obr}{pictures/dil_dec.pdf}{Dilithium signature verification}{img:dil_verify}

The variable $r$ is the result of subtracting $ct_1$ from $Az$. As mentioned in \ref{sec:dil_sign} the goal is to calculate the high bits of $w$. To achieve this the created hints are used on the vector $r$ to create the exact copy of $w_1$. The hash of $w_1$ together with the message and some other shared parameters are then compared to $c$. If they are equal the verification process succeeded, if they don't equal the verification failed. See figure \ref{img:dil_verify} for the simplified summary of the process.
\newpage
\listing{text/code/dilithium.go}{Signature verification}{lst:dil_verify_2}{123}{126}



\chapter{Application capabilities}
\label{ch:app_capab}
Many CLI applications programed in Go use an external library to parse input arguments as options or commands to alter the usage of an application. This external library is called Cobra (TODO: citation). Go also has a built in library for parsing input arguments but it can't compare to Cobra, which contains a lot more useful tools. The following section in this chapter will explain the differences between Cobra and the built in argument parser and how the Go application in this thesis utilizes it.

The application is mainly split into four groups of capabilities where each group gets a section characterizing it. The parts are
\begin{itemize}
  \item communication,
  \item benchmarking,
  \item configuration,
  \item other commands.
\end{itemize}
\noindent Additionally this chapter describes the process adding of new algorithms and extending it. The implementation results of Kyber and Dilithium are also present in a separate section of this chapter. Commands usage examples can be found in TODO: appendix.



\section{Commands and flags}
\label{sec:options}
The Cobra library is able to split an application into logical parts and execution paths, also called commands. A group of commands form a hierarchical structure, which implies that a command can only be used if its parent commands were used beforehand. An example of a group of commands is shown in Figure~\ref{img:example_tree}.
\begin{figure}[h]
  \begin{center}
    \begin{minipage}[t]{0.2\linewidth}
      \dirtree{%.
        .1 app.
        .2 foo.
        .3 baz.
        .2 qux.
      }
    \end{minipage}
  \end{center}
  \caption{Example command tree}
  \label{img:example_tree}
\end{figure}

\noindent In this case the command \texttt{baz} can only be used if \texttt{foo} has been used before
\begin{center}
  \texttt{app foo baz}.
\end{center}
On the other hand \texttt{baz} can't be used if it doesn't have the required parent commands present, so the Cobra parser would throw an error if given this set of commands
\begin{center}
  \texttt{app {\color{red} qux} baz}.
\end{center}

Additionally, Cobra allows the programmer to add a flag to a command that alters the command in some way. Flags can be inherited by other commands so that they don't have to be defined in every command separately. A good example of an inherited flag is the \texttt{-\--log} flag which enables a level of logging for the application. As this flag is created in the root of the command hierarchy of this application, all of the subcommands will share it. An example of using a flag other with commands using the previous command structure can look like this
\begin{center}
  \texttt{app foo baz -\--log info}.
\end{center}

The full list of commands for this application is presented in Figure \ref{img:command_pqcom_tree}. Each command has more flags that alter its execution path. Individual commands and flags can be explored in more detail by using the application. The commands that are greyed out don't alter the execution path by themselves but are needed for creating a parent for its subcommands so that flags can be shared amongst the command children. Another use for them is just having a parent command to logically group the command children. That means if they are run by themselves, the application just prints the output of the \texttt{help} command.

\begin{figure}[h!]
  \begin{center}
    \begin{minipage}[t]{0.2\linewidth}
      \dirtree{%.
        .1 pqcom.
        .2 {\color{gray} app}.
        .3 chat.
        .3 receive.
        .3 send.
        .2 completion.
        .2 {\color{gray} config}.
        .3 gen.
        .3 list.
        .2 help.
      }
    \end{minipage}
  \end{center}
  \caption{Command tree}
  \label{img:command_pqcom_tree}
\end{figure}


\section{Communication}
\label{sec:cmd_app}
The main purpose of this application is to create a secure post-quantum communication channel, through which users can send data. The communication mode can be used by invoking the command \texttt{app} together with the 3 sub-commands which are described in subsections. Additionally, every subcommand contains 4 shared flags where 3 of the 4 flags are used for altering the addressing, so ports and addresses.

One of the flags is used to alter the configuration of the application (see Section \ref{sec:cmd_config} for configuration options). A configuration file path can be specified in 3 ways

\begin{itemize}
  \item environmental variable \texttt{PQCOM\_CONFIG}\,--\,this variable can be set to point to the configuration file,
  \item the \texttt{-\--config} flag\,--\,a relative or absolute path to the configuration file,
  \item default path\,--\,if the above two are not specified a default configuration path will be used, it is listed as the config directory in Appendix \ref{ch:directories_app}.
\end{itemize}


\subsection{Chat command}
\label{subsec:cmd_app_chat}
An interactive mode where users can send text messages asynchronously. Text messages are printed out in the terminal which the application was ran in. The underlying application protocol works differently depending whether the user is a client or a server. The role of the user can be chosen by using the either the \texttt{-c} flag to act as a user or the \texttt{-l} flag to act as the server.

\subsection{Receive command}
\label{sec:cmd_app_recv}
In this mode the application is ran in read-only mode and can only display or save sent data. As the data is read it can either be
\begin{itemize}
  \item redirected trough the stdout (standard output) of the terminal to a different command via the pipe (|) operator or redirected to a file via the redirect operator (>),
  \item saved directly to a new file, by supplying the destination directory where the new files will be created.
\end{itemize}
To choose of one these options flags are used, if no flag is supplied the data is sent to stdout, if the flag \texttt{-\--dir [directory]} can be used. The supplied directory can be either an absolute or relative path.


\subsection{Send command}
\label{sec:cmd_app_send}
This mode is the opposite of the \texttt{receive} command since it is a write-only mode, where the user can send data to another user. Similarly the user can choose to send data in two ways
\begin{itemize}
  \item using the output of another command as the input via the pipe (|) operator,
  \item reading the contents of a file by supping the path to it.
\end{itemize}
Again the default approach when flags are not supplied is utilizing the first option. The flag \texttt{-\--file [path]} is used for supplying the file path to the input file.


\section{Benchmarking}
\label{sec:cmd_benchmark}
A speed performance benchmark can be ran on all of the post-quantum algorithms\,--\,digital signatures and key encapsulation methods\,--\,that are added to the application using the modularity system. The benchmark can be ran with the \texttt{benchmark} command. The benchmark first generates the public/private key pair and then depending on the algorithm either encrypts/decrypts a shared key or signs/verifies some data. The flag \texttt{-i} can be used to specify the amount of iterations.


\section{Configuration}
\label{sec:cmd_config}
In order to choose what post-quantum algorithms for key encapsulation and digital signature the program will use, a configuration file is used. It is in a json format and there are 4 keys that can be configured
\begin{itemize}
  \itemtt{kem\_alg}key encapsulation mechanism used for key exchange between clients,
  \itemtt{sign\_alg}digital signature algorithm for creating/verifing signatures during the initial communication,
  \itemtt{public\_key}base64 encoded string of the public key,
  \itemtt{private\_key}also a base64 encoded string of the private key.
\end{itemize}
Similarly as with the \texttt{app} command the \texttt{config} command servers for the purpose of logically separating subcommands. In this case its the \texttt{list} and \texttt{gen} commands. The latter is used for generating a new configuration file with all of the keys filled out and the latter for listing available algorithm names.
\object[0.605]{obr}{pictures/config.pdf}{Configuartion file keys}{img:config_keys}

While generating a configuration file the first two keys for the choice of algorithms can be configured using flags \texttt{-\--kem} and \texttt{-\--sign}. By supplying a string parameters to these flags the algorithm name is chosen. The selection of algorithms is generated from the source code which defines the algorithms and their functions. To get a better understanding of how these algorithms are defined in code, refer to section \ref{sec:algs_modularity}. The other two keys for the public/private key are generated depending on the choice of algorithms. If no algorithms are selected, default algorithms are used. Finally two configuration files are generated one for the client and one for the server. The clients configuration file contains the public key of the server and his own private key. The similar can be said for the server where he has the clients public key and his own private key (illustrated by figure \ref{img:config_keys}). This is necessary for the initial communication establishment phase of the underlying protocol. See chapter \ref{ch:network_com_sec} for more information.


\section{Completion and help}
\label{sec:cmd_help}
These two commands are present in any Cobra application by default. The command \texttt{help} is self-explanatory and provides information about a given command, like its description and flags. Another default command is the \texttt{completion} command which provides scripts that add the autocomplete feature to the application's commands. Autocomplete provides the user with the completion of available commands when pressing the tab key. The installation of the autocomplete functionality depends on the environment. For example if on Linux the running shell is bash, the output of the completion command needs to be copied to the \texttt{.bashrc} file. After reloading the terminal autocomplete should now work on the compiled binary.


\section{Algorithm modularity}
\label{sec:algs_modularity}
Any key encapsulation method or digital signature can be added to this application. Modularity in this application works by implementing methods of a Go interface. An interface serves a definition of methods, their parameters and return types without actually giving them an implementation. All of the interface methods need to be implemented for an algorithm to be a valid choice. Defined methods for the KEM interface are listed in \ref{lst:kem_int}.

\listing{text/code/kem.go}{KEM interface}{lst:kem_int}{1}{8}
The first methods are self explanatory. Method \texttt{EkLen} needs to return the size of the public key, \texttt{CLen} returns the size of the ciphertext. \texttt{Id} needs to return a random number, which is not already returned by any other algorithms ranging from 0 to 255. If an ID is already taken by other KEM, the application will throw an error asking the user to change the ID.

The interface for signatures is shown in listing \ref{lst:sign_int}. As with the KEM interface, the signature interface apart from the first three methods also needs a method that returns an ID, private/public key length and the signature length.
\listing{text/code/sign.go}{Signature interface}{lst:sign_int}{1}{9}

After implementing these interfaces they need to be added to a shared map contained in either \texttt{crypto/kem.go} or \texttt{crypto/sign.go} files. The key for the map entry is the algorithm name that will be used in the configuration file and the value is a pointer to the implemented interface (a Go \texttt{struct}). See the listing bellow for an example.
\listing{text/code/kem.go}{KEM algorithms map}{lst:kem_map}{10}{13}


\section{Terminal User Interface}
\label{sec:tui}
When the applications is ran using the \texttt{chat} command, it is always launched together with a TUI (\acl{TUI}) when a connection is created. A TUI is very similar to a GUI where the only difference is that it doesn't require any desktop environment and only requires a terminal interface to work. The created TUI is very easy to control, the user types in a message into the text field and can send that message to the other peer by pressing enter. The sent message will then appear in the window above together with any messages that were sent to him. To quit the application the user can press either escape or ctrl+c. The TUI is also responsive to any window size changes since it is implemented via the bubbletea\footnote{\url{https://github.com/charmbracelet/bubbletea}} TUI framework. Examples for light and dark terminal themes can be found in appendix \ref{ch:TUI_example}.


\section{Implementation results}
\label{sec:lattice_performance_measuring}
% TODO: rework
% TODO: Add mention of implementing all modes
% TODO: Fix order of bibliography
% TODO: Read trough everything 2 times
% TODO: Do more spell checking

% The benchmarked values available in tables \ref{tab:kyber_perf} and \ref{tab:dil_perf} were measured using a go program available in the practical part of this thesis, more specifically the code inside the \texttt{benchmark.go} file. One iteration of the benchmarking process for Kyber consists of key generation, key encapsulation and decapuslation. Similarly an iteration for Dilithium consists of key generation, signing process and a verification process. All benchmarks were done on AMD Ryzen 3600 cpu with a base core clock of 3.6\,GHz. Implementations used to compare the performance to the thesis are the following:
% \begin{itemize}
%   \benchmarklink{kyber-k2so}{symbolicsoft}{https://github.com/SymbolicSoft/kyber-k2so}
%   \benchmarklink{circl}{cloudflare}{https://github.com/cloudflare/circl}
% \end{itemize}
% \noindent As is clear from the results of benchmarking, this implementation of the post quantum algorithms is substantially worse then other implementations. This is because the main goal behind the implementations was to get an understanding of how these algorithms work and so they are not optimized in any way other then the NTT transformation used for polynomial multiplication. One of the main goals for the second part of this diploma thesis ill be optimizing these algorithms.
% \object{tab}{tables/kyber_performance}{Kyber performace comparison}{tab:kyber_perf}
% \object{tab}{tables/dil_performance}{Dilithium performace comparison}{tab:dil_perf}

% Along with a guide on how to run/build the benchmarking application in appendix \ref{ch:go_instructions} it also contains a guide on how to test the correctness of Kyber and Dilithium implementations. In the case of Kyber the test only checks if the shared key is the same before and after decryption. The test for Dilithium checks if the~signature is verified. These tests can be found in \texttt{main\_test.go}.


\chapter{Network communication and security}
\label{ch:network_com_sec}
The underlying communication protocol that was created for this application is described in this chapter. More specifically the header structure, types of messages and the connection initialization. Second part of this chapter will describe possible approaches of attacking this protocol and techniques that prevent such attacks.


\section{Protocol definition}
\label{sec:protocol_def}
As with any modern L7 protocol, the messages in this protocol consist of a header the actual data that is being sent also referred to as a payload. The data can either be used for initializing the connection or just for sending arbitrary data. The header is very simple and consists of only two fields the length of the payload and the type of the payload. As can be seen in Figure \ref{img:header}, the type is an 8\,bit integer and the length is a 16\,bit integer.

\object[0.6]{obr}{pictures/header.pdf}{Protocol header}{img:header}
\noindent From the possible 255 types only 4 types of payloads are implemented:
\begin{itemize}
  \itemtt{ClientInitT}initialization message for the client side,
  \itemtt{ServerInitT}initialization message for the server side,
  \itemtt{ContentT}generic data payload type,
  \itemtt{ErrorT}error messages.
\end{itemize}
The initialization together with the client and server init types is detailed in the following Subsection \ref{subsec:init_phase}. The rest of the types are mentioned in Subsection \ref{subsec:other_phase}. Apart from the headers of the message types all of the data is encrypted using a quantum-resistant symmetric cipher, more specifically AES-GCM with a keys size of 256\,b. The GCM block mode also computes a MAC which is used for validating the authentication and integrity of the encrypted message \cite{Paar2010}.


\subsection{Initialization}
\label{subsec:init_phase}
Before the initialization process can start client and server need to share each others public keys, this can easily be done by firstly generating a pair config files with the command \texttt{pqcom config gen}. Then one of the config files is moved to a peer that wants to connect to the other peer. Another approach would be to share each others public key. However it is very important to share either the public key or the config file out of band or by using another authenticated and secure communication channel.

After public keys have been exchanged, the process starts with client sending the \texttt{ClientInitT} message which is illustrated by figure \ref{img:clientinit_pdf}. This is the biggest message out of the four defined messages. It contains 7 fields in total.
\begin{itemize}
  \item \textbf{Header}\,--\,It servers the same purpose as in any other messages, to provide a message type and to delicate the payload length.
  \item \textbf{KEM and Sign Type}\,--\,These two fields exists for checking whether the two peers have the same algorithm ids configured in their configuration files. \textit{KEM Type} stores the ide for the key encapsulation method and the \textit{Sign Type} stores the digital signature algorithm. The ids that are used in these fields are defined while implementing algorithms via the modularity system (see section \ref{sec:algs_modularity}). Booth of the fields are 1 byte long so for each algorithm type there are 255 possible algorithms.
  \item \textbf{Timestamp}\,--\,In order to prevent repeat attacks, a timestamp is always sent by the client. It contains the Epoch time in microseconds and is defined with the size of 8 bytes. How exactly this timestamp is used to prevent repeat attacks is described in subsection \ref{subsec:repeat attack}.
  \item \textbf{Public Encryption Key}\,--\,This field enables the server to encrypt a randomly generated symmetric key when he receives the client init message. The client can then decrypt the randomly generated encrypted symmetric key in order to establish a shared key for the symmetric cipher. The length of public key is defined by the KEM Type field.
  \item \textbf{Signature}\,--\,To secure the above mentioned fields a digital signature is created to protect the authenticity of the whole message. Only the client can create a signature since he holds the private key in his configuration file. The public was shared with the server before hand so he can easily check whether a client init message was created by the client. The size of the signature is defined by the Sign Type field.
\end{itemize}
\object[0.6]{obr}{pictures/clientinitt.pdf}{Client inicialization message}{img:clientinit_pdf}

Once the server receives the client init message it first verifies its signature. Then it checks the timestamp and saved to a predefined location if nessescery, again to see how exactly this helps prevent the replay attack see subsection \ref{subsec:repeat attack}. Next it checks if the algorithm types are the same the ones in its configuration file. If all these checks are positive a random symmetric key is generated and encrypted using the public encryption key the client sent. Nonce is saved in memory to be used later during encryption/decryption. The \texttt{ServerInitT} message contains 3 fields as can be seen in figure \ref{img:serverinit_pdf}.
\begin{itemize}
  \item \textbf{Header}\,--\,Utilized for message type and length.
  \item \textbf{Key Ciphertext}\,--\,Encrypted symmetric key generated by the server. Only the client can decrypt it since, the keys were generated by him. There is no fixed length fot the ciphertext since its size depends on the KEM type field.
  \item \textbf{Signature}\,--\,In order to provide two way authentication the server has to digitally sign the server init message with the pre-configured private key. The signature can then be verified by the client who has the corresponding public key configured. As with the client init message signature, this signature size also depends on the received and configured Sign Type field.
\end{itemize}
\object[0.6]{obr}{pictures/serverinitt.pdf}{Server inicialization message}{img:serverinit_pdf}

Upon receive the server init message, the client first verifies its signature and decrypts the symmetric key with his private encryption key. Then the encrypted communication can start using a symmetric cipher and the shared symmetric key.


\subsection{Other communication}
\label{subsec:other_phase}
The other two message types are relatively simple. The first one is \texttt{ErrorT} message which is used for sending error messages. For example in a situation where the client's configured algorithms are not the same as the server's algorithms, the server sends an error message to the client, stating that there has been a misconfiguration. This message type can be seen in figure \ref{img:errort_pdf}.
\object[0.6]{obr}{pictures/errort.pdf}{Error message}{img:errort_pdf}

The last defined message type is the plain data message \texttt{ContentT} (see figure \ref{img:contentt_pdf}). This message is used for sending any data that the users want to exchange be it a file or just plain text messages. It can be deduced from the header fields that the maximum payload size is 65 439 or $2^{16}-1-12*8$. The header length is 16 bits long then the length of the nonce is deducted. This means that in the chat mode, users can exchange a message up to 65 439 bytes long. While in the file-sending mode, the file that is being read is read by chunks, these chunks are then sent, so the maximum file size is theoretically infinite. As mentioned beforehand the nonce is also a part of this type of message and is randomly generated for every new message. It is used as the initialization vector for the GCM operation mode while using AES.

\object[0.6]{obr}{pictures/contentt.pdf}{Content message}{img:contentt_pdf}


\section{Protection against attacks}
\label{sec:attacks_protection}
The most obvious approach to attacking this application would be to eavesdrop on the communication channel and just capture and read the traffic. This is not possible since as mention beforehand all traffic apart from the header and initialization messages is encrypted using symmetric cipher AES-GCM-256.

However more approaches could be used for breaking the security of this application. One of those approaches is the impersonation of an communicating entity, in this case one of the peers. The use of public key cryptography, more specifically digital signatures prevent the use of this approach. An attacker impersonating the client can't create a valid client init message since he can't create a digital signature that would be verifiable by the server. Only the real client can crete that signature since he has the private key. Similarly if the attacker would impersonate the server, he can't create a valid server init message since only the client holds the valid private key to the shared public key.

Another type of attack is the MitM (\acl{MITM}) attack. It happens when at attacker manages to create two simultaneous connections with the user and the server. The peers think they are communicating with each other, in truth the attacker is just forwarding their messages back and forth while being able to read them. In order to prevent this type of attack, the digital signature is again used together with the preconfigured keys. If the attack wants to create a separate connection with the server after receiving the client init message from the client, the only thing he can do is forward it trough to the server, which is harmless. He can't edit it or change the value for example of the public encryption key since he can't create a new signature. He is also unable to impersonate the server since he doesn't know his private key. Also even if the attack theoretically managed to create a connection during the same microsecond (in this instance the Epoch time is measured in microseconds) as the legitimate client he would also be rejected since the timestamp has to always be newer and not the same.


\subsection{Repeat attack}
\label{subsec:repeat attack}
If the attacker would be able to eavesdrop on the communication between two peers let's say, Alice and Bob, he could save the client init message and resend it again at a later time to initialize the connection again with being either client. Now Bob would think that he is accepting a connection from Alice, while in truth he's making a connection with the attacker.

To neutralize this type of attack a prevention mechanism is used in the form of timestamp cookies. As mentioned in \ref{subsec:init_phase} the client init message contains a 64\,b timestamp. In order to prevent time zone synchronization errors the Epoch time is used. When a server receives a client init, he first creates a name associated with the client's public key hash. If there is no cookie with this name, the server saves the timestamp in local storage together with the received timestamp. The cookie is always saved to a cookie directory listed in appendix \ref{ch:directories_app}. If a new client init message comes in from the same client containing a timestamp that is newer\,--\,meaning a higher number, meaning ahead of the current timestamp\,--\,the cookie is no longer created just updated with the new timestamp. On the other hand, if the server receives a client init with the same or older timestamp, he drops the connection since the timestamp was not updated with a newer one. This prevents the attack from repeating a client init message. Of course, this works only at the assumption that the first-ever connection from that client is a legitimate one.



%%% Vložení souboru 'text/conclusion' se závěrem
\chapter*{Conclusion}
\phantomsection
\addcontentsline{toc}{chapter}{Conclusion}

The aim of this this was to introduce the reader with the possibility of a quantum supremacy where powerful enough quantum computers are capable of breaking modern cryptography. As was shown this might be a real possibility in the near future since IBM and other companies have started researching quantum computers and even building them. However the best quantum computer at the time of writing this is a 127 qubit one. Although that doesn't change the fact that post quantum algorithms which are resistent to quantum algorithm attack for public key cryptography are needed.

So far this thesis has introduced Kyber and Dilithium which are lattice-based post quantum algorithms standardized by NIST (\acl{NIST}). The continuation of this thesis will include implementations for additional post quantum algorithms such as SPHINCS+ which is based on hashes and McEliece based on codes.

The implemented algorithms in this thesis are working correctly but are much less performant then other implementations such as the cloudflares circl library. One of the goals for the continuation of this thesis will be improving the performance of Kyber and Dilithium. Another goal will be creating a client-server communication protocol for exchanging data. The protocol will use the implementation of post quantum public key algorithms described in this thesis.


%%% Vložení souboru 'text/bibliography' se seznamem zdrojů
% For the list of references, use one of the two options below

%%%%%%%%%%%%%%%%%%%%%%%%%%%%%%%%%%%%%%%%%%%%%%%%%%%%%%%%%%%%%%%%%%%%%%%%%
%1) References created directly, by hand, using the 'thebibliography' environment

\begin{thebibliography}{99}
  \bibitem{Bernstein149}
  BERNSTEIN, Daniel J. a Tanja LANGE. Post-quantum cryptography. \textit{Nature} [online]. 2017, 14.9, \textbf{2017}(549), 188-194 [cit. 2022-10-09]. Dostupné z: doi:\url{https://doi.org/10.1038/nature23461}
  \bibitem{Smart2004}
  SMART, Nigel. \textit{Cryptography: An Introduction} [online]. 3rd. ed. McGraw-Hill College, 2004 [cit. 2020-10-18]. ISBN 978-0077099879. Dostupné z: \url{https://www.cs.umd.edu/~waa/414-F11/IntroToCrypto.pdf}
  \bibitem{Ristic2014}
  RISTIĆ, Ivan. \textit{Bulletproof SSL and TLS: Understanding and Deploying SSL/TLS and PKI to Secure Servers and Web Applications Ivan Ristic}. 6 Acantha Court, Montpelier Road, London W5 2QP, United Kingdom: Feisty Duck, 2014. ISBN 978-1-907117-04-6.
  \bibitem{Paar2010}
  PAAR, Christof a Jan PELZL. \textit{Understanding Cryptography: A Textbook for Students and Practitioners}. 2nd edition. London New York: Springer Heidelberg Dordrecht, 2010, 382~s. ISBN 978-3-642-44649-8.
  \bibitem{Shannon1949}
  SHANNON, Claude E. Communication Theory of Secrecy Systems. \textit{Bell System Technical Journal}. 1949, \textbf{4}(28), 656-715.
  \bibitem{Barker2017}
  BARKER, Elaine a Nicky MOUHA. \textit{Recommendation for the Triple Data Encryption Algorithm (TDEA) Block Cipher}. 2nd ed. NIST Pubs, 2017, 32~s. Dostupné také z: \url{https://nvlpubs.nist.gov/nistpubs/SpecialPublications/NIST.SP.800-67r2.pdf}
  \bibitem{Chen2016}
  CHEN, Lily, Stephen JORDAN, Yi-Kai LIU, Dustin MOODY, Rene PERALTA, Ray PERLNER a Daniel SMITH-TONE. NISTIR 8105. \textit{Report on Post-Quantum Cryptography}. NIST, 2016, 15~s. Dostupné také z: \url{http://dx.doi.org/10.6028/NIST.IR.8105}
  % \bibitem{Nir2015}
  % NIR, Y. a A. LANGLEY. \textit{ChaCha20 and Poly1305 for IETF Protocols}. Internet Engineering Task Force, 2015, 45~s. Dostupné také z: \url{https://tools.ietf.org/html/rfc7539}
  \bibitem{rd1wUlxEgliEynii}
  FIPS PUB 180-4. \textit{Secure Hash Standard}. Gaithersburg, USA: NIST, 2015, 36~s. Dostupné také z: \url{http://dx.doi.org/10.6028/NIST.FIPS.180-4}
  \bibitem{1Od8f4TuMxetfmHu}
  FIPS PUB 202. \textit{SHA-3 standard: permutation-based hash and extendable output functions}. Gaithersburg, USA: NIST, 2015, 37~s. Dostupné také z: \url{http://dx.doi.org/10.6028/NIST.FIPS.202}
  \bibitem{Bernstein2009}
  BERNSTEIN, Daniel J., Johannes BUCHMANN a Erik DAHMEN. \textit{Post-Quantum Cryptography}. Berlin: Springer-Verlag, 2009, 248~s. ISBN 978-3-540-88701-0.
  \bibitem{Yanofsky2008}
  YANOFSKY, Noson S. a Mirco A. MANNUCCI. \textit{Qunatum computing for cumputer scientists}. New York: Cambridge university press, 2008, 402~s. ISBN 978-0-521-87996-5.
  \bibitem{McMahon2008}
  MCMAHON, David. \textit{Quantum computing explained}. New Jersey: John Wiley \& Sons, 2008, 351~s. ISBN 978-0-470-09699-4. 
  \bibitem{Pittenger2000}
  PITTENGER, Arthur O. \textit{An Introduction to Quantum Computing Algorithms}. Boston: Birkhäuser, 2000, 150~s. ISBN ISBN 0-8176-4127-0.
  \bibitem{Pretson2022}
  PRETSON, Richard. \textit{Applying Grover-s Algorithm to Hash Functions: A Software Perspective}. Bedford: The MITRE Corporation, 2022. Dostupné také z: \url{https://arxiv.org/pdf/2202.10982.pdf}
  \bibitem{Mosca2015}
  MOSCA, Michele. \textit{Cybersecurity in an era with quantum computers: will we be ready?}. Ontario: Cryptology ePrint Archive, 2015, 4~s. Dostupné také z: \url{https://eprint.iacr.org/2015/1075}
  \bibitem{0MBNdFRCTLK35MFY}
  IBM Unveils Breakthrough 127-Qubit Quantum Processor. IBM. \textit{IBM Newsroom} [online]. 2021 [cit. 2022-10-26]. Dostupné z: \url{https://newsroom.ibm.com/2021-11-16-IBM-Unveils-Breakthrough-127-Qubit-Quantum-Processor}
  \bibitem{Gambetta2021}
  GAMBETTA, Jay. Expanding the IBM Quantum roadmap to anticipate the future of quantum-centric supercomputing. IBM. \textit{IBM research} [online]. 2021 [cit. 2022-10-26]. Dostupné z: \url{https://research.ibm.com/blog/ibm-quantum-roadmap-2025}
  

\end{thebibliography}

%%%%%%%%%%%%%%%%%%%%%%%%%%%%%%%%%%%%%%%%%%%%%%%%%%%%%%%%%%%%%%%%%%%%%%%%%
%%2) References generated using BibTeX (automatically from a database of sources)
%% Selection of the citation 'style'
% \bibliographystyle{unsrt}
%% Selection of the database file containing the sources
% \bibliography{text/literatura}
%
%% The following command is only to show a list of references using BibTeX.
%% It makes listed all the items from literatura.bib, although they are not cited in the text.
% \nocite{*}

%%% Vložení souboru 'text/abbreviation' se seznam použitých symbolů, veličin a zkratek
\cleardoublepage
\chapter*{\listofabbrevname}
\phantomsection
\addcontentsline{toc}{chapter}{\listofabbrevname}

\begin{acronym}[mmmmmmm]
	\acro{AES}[AES]{Advanced Encryption Standard}
	\acro{CBD}[CBD]{Centrail Bionimal Distribution}
	\acro{CFB}[CFB]{Cipher Feedback}
	\acro{CRYSTALS}[CRYSTALS]{Cryptographic Suite for Algebraic Lattices}
	\acro{CVP}[CVP]{Closest Vector Problem}
	\acro{DES}[DES]{Data Encryption Standard}
	\acro{DH}[DH]{Diffie-Hellman}
	\acro{DLP}[DLP]{Discrete Logarithm Problem}
	\acro{DSA}[DSA]{Digital Signature Algorithm}
	\acro{ECB}[ECB]{Electronic Code Book}
	\acro{ECC}[ECC]{Elitic Cruve Cryptography}
	\acro{ECDH}[ECDH]{Elitic Curve Diffie-Hellman}
	\acro{ECDSA}[ECDSA]{Elitpic curve Digital Signature Algorithm}
	\acro{GCD}[GCD]{Greatest Common Divisor}
	\acro{GCM}[GCM]{Galois/Counter Mode}
	\acro{IFP}[IFP]{Integer Factorization Problem}
	\acro{KEM}[KEM]{Key Encapsulating Mechanism}
	\acro{LWE}[LWE]{Learning With Errors}
	\acro{MAC}[MAC]{Message Authentication Code}
	\acro{MLWE}[MLWE]{Module Learning with Errors}
	\acro{NIST}[NIST]{National Insititue of Standards and Technology}
	\acro{NTRU}[NTRU]{N-th degree Truncated Polynomial Ring}
	\acro{OFB}[OFB]{Output Feedback}
	\acro{PRNG}[PRNG]{Pseudo random number generator}
	\acro{QFT}[QFT]{Quantum Fourier Transform}
	\acro{RSA}[RSA]{Rivest Shamir Adleman}
	\acro{SVP}[SVP]{Shortest Vector Problem}
	\acro{TLS}[TLS]{Transport Layer Security}
  \acro{KEP}[KEP]{Key Exchange Protocol}
  \acro{PDU}[PDU]{Protocol Data Unit}
  \acro{TCP}[TCP]{Transmission Control Protocol}
  \acro{UDP}[UDP]{User Datagram Protocol}
	\acro{NTT}[NTT]{Number Theoretic Transform}
\end{acronym}


%%% Začátek příloh
\appendix

%%% Vysázení seznamu příloh
% (vynechejte, pokud máte dvě nebo méně příloh)
\listofappendices

%%% Vložení souboru 'text/appendix' s přílohami
% Obvykle je přítomen alespoň popis co najdeme na přiloženém médiu
\chapter{Appendix 1}


\end{document}

% TODO: Citacie na kyber/dilithium?
% TODO: Reduce go version
% TODO: Check benchstat install on a fresh system
