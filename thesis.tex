% Files should be in encoding corresponding to the setting of \usepackage[...]{inputenc}

\documentclass[%      Basic settings of the document
%  draft,    				  % Turn on draft compilation
  12pt,       				% 12 points default font size
  a4paper,    				% Paper format A4
  %oneside,      			% One-sided print
	twoside,      			% Two-sided print (chapters and other important blocks begin at odd-numbered pages)
	unicode,						% Bookmarks and metainformation in the compiled PDF will be unicode
]{report}				    	% Report class, suitable for typesetting theses

\usepackage[utf8]		  %	The input files are in UTF-8 (unicode)
	{inputenc}					% Package for setting input encoding

\usepackage[				    % Geometry of the pages
	bindingoffset=10mm,		% Binding offset
	hmargin={25mm,25mm},	% Inner and outer margins
	vmargin={25mm,34mm},	% Upper and lower margins
	footskip=17mm,			  % Footskip size
	nohead,					      % No header
	marginparsep=2mm,		  % Margin separation amount
	marginparwidth=18mm,	% Width of the text margins
]{geometry}

\usepackage{sectsty}
	%sets all (sub)section headers to sans-serif type,
	%except for \chapter, for which the setting is done separately in thesis.sty
	\allsectionsfont{\sffamily}

\usepackage{graphicx} % Package for inclusion of external graphics
						

\usepackage[
	nohyperlinks				% No hyperlinks to the list of acronyms will be created
]{acronym}						% Package 'acronym' for automated handling of abbreviations and symbols
											% Neccessary for the 'acronym' environment of 'thesis' to work

\usepackage[
	breaklinks=true,		% Hypertext links can be line-broken
	hypertexnames=false % Names of hypertextlinks will be
											% independent of TeX labels
]{hyperref}						% Package 'hyperref' for the support of hypertext link
											% Neccessary for the 'pdfsettings' command of 'thesis' package

\usepackage{pdfpages} % Package allowing including pages from external PDFs
                      % Neccessary for appending the PDF of the title page etc.,
											% which are generated by the BUT information system

\usepackage{enumitem} % Package for handling spacing in itemization environments
  \setlist{topsep=0pt,partopsep=0pt,noitemsep}

\usepackage{cmap} 		% This package allows the final PDF
											% is fully "searchable" and "copyable"

%\usepackage{upgreek}	% Package for typesetting roman-type greek letters
											%% e.g. upright pi: \uppi
											%% e.g. upright mu: \upmu (usable in micrometers, for instance)
											%% be aware of the graphical incompatibility with the default Computer Modern fonts!

\usepackage{dirtree}	% Printing the directory tree
                      % useful for detailing the stuff that is included on the additional medium like CD

\usepackage[formats]{listings}	% Package enabling inclusion of computer codes
\lstset{
%	Definition of the programming language
% language=[LaTeX]{TeX},	% LaTeX
%	language={Matlab},		  % Matlab
	language={C},           % C
    basicstyle=\ttfamily,	% basic font type
    tabsize=2,			      % with of the tab
    inputencoding=utf8,   % character encoding of the files that will be included
		columns=fixed,        % choose fixed or flexible,
		fontadjust=true       % adjusting the columns
    extendedchars=true,
    literate=%              definition of symbols with diacritics
    {á}{{\'a}}1
    {č}{{\v{c}}}1
    {ď}{{\v{d}}}1
    {é}{{\'e}}1
    {ě}{{\v{e}}}1
    {í}{{\'i}}1
    {ň}{{\v{n}}}1
    {ó}{{\'o}}1
    {ř}{{\v{r}}}1
    {š}{{\v{s}}}1
    {ť}{{\v{t}}}1
    {ú}{{\'u}}1
    {ů}{{\r{u}}}1
    {ý}{{\'y}}1
    {ž}{{\v{z}}}1
    {Á}{{\'A}}1
    {Č}{{\v{C}}}1
    {Ď}{{\v{D}}}1
    {É}{{\'E}}1
    {Ě}{{\v{E}}}1
    {Í}{{\'I}}1
    {Ň}{{\v{N}}}1
    {Ó}{{\'O}}1
    {Ř}{{\v{R}}}1
    {Š}{{\v{S}}}1
    {Ť}{{\v{T}}}1
    {Ú}{{\'U}}1
    {Ů}{{\r{U}}}1
    {Ý}{{\'Y}}1
    {Ž}{{\v{Z}}}1
}

%%%%%%%%%%%%%%%%%%%%%%%%%%%%%%%%%%%%%%%%%%%%%%%%%%%%%%%%%%%%%%%%%
%%%%%%%%                   Definitions                 %%%%%%%%%%
%%%%%%%%%%%%%%%%%%%%%%%%%%%%%%%%%%%%%%%%%%%%%%%%%%%%%%%%%%%%%%%%%

% Variable fields such as your name, title of the thesis atc. are set in this file.
% The file is SHARED between the thesis text and the presentation --- no need to set anything twice.

\usepackage[
  english,			% English study program
%%% Choose one of the thesis types
	semestral,		%	semestral project (abstracts, declaration and acknowledgements are excluded) (default)
  %bachelor,			%	Bachelor's thesis
  %master,			% Master's thesis
  %treatise,			% Treatise on doctoral thesis
  %doctoral,				% Doctoral thesis
%%% Choose one of the options:
%  left,				% Equations and captions will be aligned to the left
  center,			% Equations and captions will be centered (default)
]{thesis}   % Package for typesetting theses


%%% First and last name of the thesis author, following the scheme
% [titles in front]{FirstName}{LastName}[titles after]
% If the author does not have a title in front/after, just delete the entire string '[...]'
\author[Bc.]{Samuel}{Kopecký}

%%% Brno University of Technology identification number of author (BUT ID)
\butid{211799}

%%% First and last name of the advisor
% [titles in front]{FirstName}{LastName}[titles after]
% If the author does not have a title in front/after, just delete the entire string '[...]'
% [titles in front]{FirstName}{LastName}[titles after]
\advisor[Ing.]{David}{Smékal}[]

%%% First and last name of the opponent
% [titles in front]{FirstName}{LastName}[titles after]
% Makes use only in the presentation for defense;
% in case you do not want the opponent to be shown on the title slide, just comment out the command
% The opponent is not shown in case of the semestral thesis (no opponent exists actually at that time)
% In the case of a doctoral thesis, two to three oponents are typically involved. In such a case, if you want to have them on the title slide, please go to the definition of "VUT title page" in the thesis.sty file, uncomment and adapt their names.
\opponent[TODO]{TODO}{TODO}[TODO]

%%% Thesis title
%  In the case of a very long thesis title, it may happen that it does not fit
%  into the slot in the footbar of the slides. You may use the command
%  \def\insertshorttitle{Shortened th.\ title}
%  where the shortened version of the title appears as the parameter.
%  If you do not want to shorten the title, you will have to redefine how the slide footbar
%  is generated, see: https://bit.ly/3EJTp5A
\title{Modular communication based on post-quantum cryptohraphy}

%%% Study program/specialization
\specialization{Information Security} %Teleinformatika

%%% Department
%\department{Department of Control and Instrumentation}  %Ústav automatizace a měřicí techniky
%\department{Department of Biomedical Engineering}       %Ústav biomedicínského inženýrství
%\department{Department of Electrical Power Engineering} %Ústav elektroenergetiky
%\department{Department of Electrical and Electronic Technology}   %Ústav elektrotechnologie
%\department{Department of Physics}                      %Ústav fyziky
%\department{Department of Foreign Languages}            %Ústav jazyků
%\department{Department of Mathematics}                  %Ústav matematiky
%\department{Department of Microelectronics}             %Ústav mikroelektroniky
%\department{Department of Radio Electronics}            %Ústav radioelektroniky
%\department{Department of Theoretical and Experimental Electrical Engineering}  %Ústav teoretické a experimentální elektrotechniky
\department{Department of Telecommunications}           %Ústav telekomunikací
%\department{Department of Power Electrical and Electronic Engineering}   %Ústav výkonové elektrotechniky a elektroniky

%%% Faculty
%\faculty{Faculty of Architecture}   %Fakulta architektury
\faculty{Faculty of Electrical Engineering and~Communication}   %Fakulta elektrotechniky a~komunikačních technologií
%\faculty{Faculty of Chemistry}   %Fakulta chemická
%\faculty{Faculty of Information Technology}   %Fakulta informačních technologií
%\faculty{Faculty of Business and Management}   $Fakulta podnikatelská
%\faculty{Faculty of Civil Engineering}   %Fakulta stavební
%\faculty{Faculty of Mechanical Engineering}   %Fakulta strojního inženýrství
%\faculty{Faculty of Fine Arts}   %Fakulta výtvarných umění
%
%Logotype selection (in square brackets short logo, in curly brackets full logo):
\facultylogo[logo/FEEC_abbreviation_color_PANTONE_EN]{logo/UTKO_color_PANTONE_EN}


%%% Graduate year (typically the calendar year of the defense)
\graduateyear{2023}
%%% Academic year (typically the year of solution of the thesis in the format n/n+1)
\academicyear{2022/23}
% Date of the defense (makes use only in the presentation slides)
\date{TODO} 

%%% Place of the defense
\city{Brno}

%%% Abstract
\abstract{%
Abstract in English.
}

%%% Keywords
\keywrds{%
Keywords in English
}

%%% Thanks and acknowledgement
\acknowledgement{%
TODO
}%  % please go into that file and fill information about you, your topic etc.



%%%%%%%%%%%%%%%%%%%%%%%%%%%%%%%%%%%%%%%%%%%%%%%%%%%%%%%%%%%%%%%%%%%%%%%%
%%%%%   Setting values that appear in the Properties of the PDF  %%%%%%%
%%%%%%%%%%%%%%%%%%%%%%%%%%%%%%%%%%%%%%%%%%%%%%%%%%%%%%%%%%%%%%%%%%%%%%%%
%% If 'hyperref' package is loaded then command '\pdfsettings' can be simply used
\pdfsettings
%  However, you can set the individual fields by hand:
%\hypersetup{
%  pdftitle={Title of the thesis},    	% 'Document Title' field
%  pdfauthor={Author name},   	        % 'Author' field
%  pdfsubject={Type of the document},  	% 'Subject' field
%  pdfkeywords={Keywords}           	  % 'Keywords' field
%}
%%%%%%%%%%%%%%%%%%%%%%%%%%%%%%%%%%%%%%%%%%%%%%%%%%%%%%%%%%%%%%%%%%%%%%%

\pdfmapfile{=vafle.map}

%%%%%%%%%%%%%%%%%%%%%%%%%%%%%%%%%%%%%%%%%%%%%%%%%%%%%%%%%%%%%%%%%%%%%%%
%%%%%%%%%%%     Actual start of the document      %%%%%%%%%%%%%%%%%%%%%
%%%%%%%%%%%%%%%%%%%%%%%%%%%%%%%%%%%%%%%%%%%%%%%%%%%%%%%%%%%%%%%%%%%%%%%
\begin{document}
\pagestyle{empty}      %avoiding page numbering for the moment

%%%% Include the cover  --- Sept. 2021: the faculty wants to disable generating the cover
%\includepdf[pages=1]%  either by using PDF generated by the information system
  %{pdf/cover-example}% no spaces allowed in the filename!
%%%% OR the cover can be generated using
%%\makecover
%%%%
%\oddpage%  this command adds an empty page when two-side printing is selected
%% but in any case:
%\setcounter{page}{1} %reset the pagecounter --- the cover should not affect numbering

%%% Include the titlepage
\includepdf[pages=1]%    either by using PDF generated by the information system
  {pdf/titlepage-example}% no spaces allowed in the filename!
%%% OR the titlepage can be generated using
%\maketitle
%%%
\oddpage%    this command adds an empty page when two-side printing is selected
   
%%% Include the thesis assignment (goals)
\includepdf[pages=1]%   either by using PDF generated by the information system
  {pdf/assignment-example}% no spaces allowed in the filename!
%%% OR the empty page reserving the space for assignment can be generated using
%\patternpage{}%
%	{\sffamily\Huge\centering IN PLACE OF HERE\\ INSERT THE SHEET WITH THE THESIS ASSIGNMENT}%
%	{\sffamily\centering This page is here in order to keep page numbering correct}
%%%
\oddpage%   this command adds an empty page when two-side printing is selected

%%% Generate abstract
\makeabstract

%%% Generate declaration
\makedeclaration

%%% Generate page with thanks and acknowledgements
\makeacknowledgement

%%% Generate table of contents
\tableofcontents

%%% Generate list of figures
% (exclude such a list if it contains two or less items)
\listoffigures

%%% Generate list of tables
% (exclude such a list if it contains two or less items)
\listoftables

%%% Generate list of listings
% (exclude such a list if it contains two or less items)
\lstlistoflistings

\cleardoublepage\pagestyle{plain}   % turn page-numbering on

%Including chapters and appendices should be preferrably done with \include instead of \input
%%% Include the file with Introduction
\chapter*{Introduction}
\phantomsection
\addcontentsline{toc}{chapter}{Introduction}

Here comes the introduction of the thesis, for example\,\dots

This thesis is devoted to \acs{DSP} (\acl{DSP}),
especially it analyses the effect happening when the Nyquist condition for \ac{symfs} is not satisfied.%
\footnote{This sentence is only to demonstrate how abbreviations can be used and typeset.}

The template is set to twoside printing by default.
Do not be surprised that you find empty pages in your PDF.
They are there to make the chapters and other important stuff begin on the right side when the document is printed.
Having a~serious reason to print one-sided, please switch the option \texttt{twoside} to \texttt{oneside}!

test

daw
da
w
test


%%% Include the file with Introduction
\chapter*{Aim of the thesis}
\phantomsection
\addcontentsline{toc}{chapter}{Aim of the thesis}

Specification of the objectives to be solved in the thesis.
If your study program does not insist on having such a~separate chapter with the aims,
please specify them as a~part of the Introduction.

%%% Include the file with the project solution
\chapter{Theory}

Theoretical background of the thesis comes now, suitably split into chapters and sections.

(The structure suggested in this template is the coarsest one.
Please discuss your particular structure with your adviser.)


%%% Include the file with the results
\chapter{Thesis Results}

Practical part and results of the student, suitably split into chapters and sections.
\section{Selection of Programming Language}
Lorem ipsum dolor sit amet, consectetuer adipiscing elit. Nulla pulvinar eleifend sem. Integer in sapien. Etiam sapien elit, consequat eget, tristique non, venenatis quis, ante. In laoreet, magna id viverra tincidunt, sem odio bibendum justo, vel imperdiet sapien wisi sed libero. Phasellus enim erat, vestibulum vel, aliquam a, posuere eu, velit. Aliquam erat volutpat. Nullam faucibus mi quis velit \cite{sr02/2009}.

\section{Implementation}
Fusce tellus odio, dapibus id fermentum quis, suscipit id erat. Fusce tellus. Morbi scelerisque luctus velit. In laoreet, magna id viverra tincidunt, sem odio bibendum justo, vel imperdiet sapien wisi sed libero. Quisque porta. Fusce suscipit libero eget elit. Nulla non lectus sed nisl molestie malesuada. Phasellus faucibus molestie nisl. Integer vulputate sem a nibh rutrum consequat. Proin mattis lacinia justo. Phasellus et lorem id felis nonummy placerat. Etiam ligula pede, sagittis quis, interdum ultricies, scelerisque eu. Cras elementum. Aenean placerat. Donec ipsum massa, ullamcorper in, auctor et, scelerisque sed, est. Aliquam ante. Integer imperdiet lectus quis justo. Vivamus ac leo pretium faucibus. Nullam faucibus mi quis velit.

\subsection{Tests and Evaluation}
Neque porro quisquam est, qui dolorem ipsum quia dolor sit amet, consectetur, adipisci velit, sed quia non numquam eius modi tempora incidunt ut labore et dolore magnam aliquam quaerat voluptatem. Aliquam erat volutpat. Lorem ipsum dolor sit amet, consectetuer adipiscing elit \cite{sr02/2009,pravidla}. Nunc auctor. Neque porro quisquam est, qui dolorem ipsum quia dolor sit amet, consectetur, adipisci velit, sed quia non numquam eius modi tempora incidunt ut labore et dolore magnam aliquam quaerat voluptatem. Maecenas lorem. Maecenas libero. In laoreet, magna id viverra tincidunt, sem odio bibendum justo, vel imperdiet sapien wisi sed libero. Nullam rhoncus aliquam metus.

\subsubsection{Integer rutrum orci vestibulum}
Integer rutrum, orci vestibulum ullamcorper ultricies, lacus quam ultricies odio, vitae placerat pede sem sit amet enim. Ut enim ad minim veniam, quis nostrud exercitation ullamco laboris nisi ut aliquip ex ea commodo consequat. Fusce tellus odio, dapibus id fermentum quis, suscipit id erat. Nullam eget nisl. Nunc auctor. Etiam dui sem, fermentum vitae, sagittis id, malesuada in, quam. Fusce dui leo, imperdiet in, aliquam sit amet, feugiat eu, orci. Curabitur vitae diam non enim vestibulum interdum. Aliquam erat volutpat. Pellentesque sapien. Phasellus enim erat, vestibulum vel, aliquam a, posuere eu, velit.

\subsubsection{Eger rutrum orci westibulum}
Fusce dui leo, imperdiet in, aliquam sit amet, feugiat eu, orci. Maecenas aliquet accumsan leo. Aliquam ornare wisi eu metus. Cum sociis natoque penatibus et magnis dis parturient montes, nascetur ridiculus mus. Aliquam erat volutpat. Donec iaculis gravida nulla. Sed elit dui, pellentesque a, faucibus vel, interdum nec, diam. Temporibus autem quibusdam et aut officiis debitis aut rerum necessitatibus saepe eveniet ut et voluptates repudiandae sint et molestiae non recusandae. Nulla non arcu lacinia neque faucibus fringilla. Phasellus enim erat, vestibulum vel, aliquam a, posuere eu, velit. Praesent vitae arcu tempor neque lacinia pretium
\cite{Walter1999,Svacina1999IEEE,RajmicSysel2002}.
a
Aliquam erat volutpat. Quisque porta. Integer imperdiet lectus quis justo. Nullam justo enim, consectetuer nec, ullamcorper ac, vestibulum in, elit. Nullam faucibus mi quis velit. Fusce tellus. Fusce consectetuer risus a nunc. Cras pede libero, dapibus nec, pretium sit amet, tempor quis. Morbi imperdiet, mauris ac auctor dictum, nisl ligula egestas nulla, et sollicitudin sem purus in lacus
\cite{CSN_ISO_690-2011,CSN_ISO_7144-1997,CSN_ISO_31-11}.
Mauris elementum mauris vitae tortor. Neque porro quisquam est, qui dolorem ipsum quia dolor sit amet, consectetur, adipisci velit, sed quia non numquam eius modi tempora incidunt ut labore et dolore magnam aliquam quaerat voluptatem. Quisque porta. Integer vulputate sem a nibh rutrum consequat. Nulla pulvinar eleifend sem. Praesent id justo in neque elementum ultrices \cite{BiernatovaSkupa2011:CSNISO690komentar}.



%%% Include the file with Conclusion
\chapter*{Conclusion}
\phantomsection
\addcontentsline{toc}{chapter}{Conclusion}

Thesis conclusion. Do this when you are finished with writing the main chapters.


%%% Include the file with references
% For the list of references, use one of the two options below

%%%%%%%%%%%%%%%%%%%%%%%%%%%%%%%%%%%%%%%%%%%%%%%%%%%%%%%%%%%%%%%%%%%%%%%%%
%1) References created directly, by hand, using the 'thebibliography' environment

\begin{thebibliography}{99}
  \bibitem{Bernstein149}
  BERNSTEIN, Daniel J. a Tanja LANGE. Post-quantum cryptography. \textit{Nature} [online]. 2017, 14.9, \textbf{2017}(549), 188-194 [cit. 2022-10-09]. Dostupné z: doi:\url{https://doi.org/10.1038/nature23461}
  \bibitem{Smart2004}
  SMART, Nigel. \textit{Cryptography: An Introduction} [online]. 3rd. ed. McGraw-Hill College, 2004 [cit. 2020-10-18]. ISBN 978-0077099879. Dostupné z: \url{https://www.cs.umd.edu/~waa/414-F11/IntroToCrypto.pdf}
  \bibitem{Ristic2014}
  RISTIĆ, Ivan. \textit{Bulletproof SSL and TLS: Understanding and Deploying SSL/TLS and PKI to Secure Servers and Web Applications Ivan Ristic}. 6 Acantha Court, Montpelier Road, London W5 2QP, United Kingdom: Feisty Duck, 2014. ISBN 978-1-907117-04-6.
  \bibitem{Paar2010}
  PAAR, Christof a Jan PELZL. \textit{Understanding Cryptography: A Textbook for Students and Practitioners}. 2nd edition. London New York: Springer Heidelberg Dordrecht, 2010, 382~s. ISBN 978-3-642-44649-8.
  \bibitem{Shannon1949}
  SHANNON, Claude E. Communication Theory of Secrecy Systems. \textit{Bell System Technical Journal}. 1949, \textbf{4}(28), 656-715.
  \bibitem{Barker2017}
  BARKER, Elaine a Nicky MOUHA. \textit{Recommendation for the Triple Data Encryption Algorithm (TDEA) Block Cipher}. 2nd ed. NIST Pubs, 2017, 32~s. Dostupné také z: \url{https://nvlpubs.nist.gov/nistpubs/SpecialPublications/NIST.SP.800-67r2.pdf}
  \bibitem{Chen2016}
  CHEN, Lily, Stephen JORDAN, Yi-Kai LIU, Dustin MOODY, Rene PERALTA, Ray PERLNER a Daniel SMITH-TONE. NISTIR 8105. \textit{Report on Post-Quantum Cryptography}. NIST, 2016, 15~s. Dostupné také z: \url{http://dx.doi.org/10.6028/NIST.IR.8105}
  % \bibitem{Nir2015}
  % NIR, Y. a A. LANGLEY. \textit{ChaCha20 and Poly1305 for IETF Protocols}. Internet Engineering Task Force, 2015, 45~s. Dostupné také z: \url{https://tools.ietf.org/html/rfc7539}
  \bibitem{rd1wUlxEgliEynii}
  FIPS PUB 180-4. \textit{Secure Hash Standard}. Gaithersburg, USA: NIST, 2015, 36~s. Dostupné také z: \url{http://dx.doi.org/10.6028/NIST.FIPS.180-4}
  \bibitem{1Od8f4TuMxetfmHu}
  FIPS PUB 202. \textit{SHA-3 standard: permutation-based hash and extendable output functions}. Gaithersburg, USA: NIST, 2015, 37~s. Dostupné také z: \url{http://dx.doi.org/10.6028/NIST.FIPS.202}
  \bibitem{Bernstein2009}
  BERNSTEIN, Daniel J., Johannes BUCHMANN a Erik DAHMEN. \textit{Post-Quantum Cryptography}. Berlin: Springer-Verlag, 2009, 248~s. ISBN 978-3-540-88701-0.
  \bibitem{Yanofsky2008}
  YANOFSKY, Noson S. a Mirco A. MANNUCCI. \textit{Qunatum computing for cumputer scientists}. New York: Cambridge university press, 2008, 402~s. ISBN 978-0-521-87996-5.
  \bibitem{McMahon2008}
  MCMAHON, David. \textit{Quantum computing explained}. New Jersey: John Wiley \& Sons, 2008, 351~s. ISBN 978-0-470-09699-4. 
  \bibitem{Pittenger2000}
  PITTENGER, Arthur O. \textit{An Introduction to Quantum Computing Algorithms}. Boston: Birkhäuser, 2000, 150~s. ISBN ISBN 0-8176-4127-0.
  \bibitem{Pretson2022}
  PRETSON, Richard. \textit{Applying Grover-s Algorithm to Hash Functions: A Software Perspective}. Bedford: The MITRE Corporation, 2022. Dostupné také z: \url{https://arxiv.org/pdf/2202.10982.pdf}
  \bibitem{Mosca2015}
  MOSCA, Michele. \textit{Cybersecurity in an era with quantum computers: will we be ready?}. Ontario: Cryptology ePrint Archive, 2015, 4~s. Dostupné také z: \url{https://eprint.iacr.org/2015/1075}
  \bibitem{0MBNdFRCTLK35MFY}
  IBM Unveils Breakthrough 127-Qubit Quantum Processor. IBM. \textit{IBM Newsroom} [online]. 2021 [cit. 2022-10-26]. Dostupné z: \url{https://newsroom.ibm.com/2021-11-16-IBM-Unveils-Breakthrough-127-Qubit-Quantum-Processor}
  \bibitem{Gambetta2021}
  GAMBETTA, Jay. Expanding the IBM Quantum roadmap to anticipate the future of quantum-centric supercomputing. IBM. \textit{IBM research} [online]. 2021 [cit. 2022-10-26]. Dostupné z: \url{https://research.ibm.com/blog/ibm-quantum-roadmap-2025}
  

\end{thebibliography}

%%%%%%%%%%%%%%%%%%%%%%%%%%%%%%%%%%%%%%%%%%%%%%%%%%%%%%%%%%%%%%%%%%%%%%%%%
%%2) References generated using BibTeX (automatically from a database of sources)
%% Selection of the citation 'style'
% \bibliographystyle{unsrt}
%% Selection of the database file containing the sources
% \bibliography{text/literatura}
%
%% The following command is only to show a list of references using BibTeX.
%% It makes listed all the items from literatura.bib, although they are not cited in the text.
% \nocite{*}

%%% Include the file with the list of symbols, quantities and abbreviations
\cleardoublepage
\chapter*{\listofabbrevname}
\phantomsection
\addcontentsline{toc}{chapter}{\listofabbrevname}

\begin{acronym}[mmmmmmm]

	\acro{test}		% label of the abbrev.
		[PQ]								% symbol
		{Post Quantum}
											% full text
\end{acronym}


%%% Apendices begin here
\appendix

%%% Generate the list of appendices 
% (exclude such a list if it contains two or less items)
\listofappendices

%%% Include the file with appendices
% Usually, at least the description of the attached medium is present
\chapter{Lattice based algorithms}

\begin{figure}[h!]
  \centering
  \includegraphics[width=0.484\textwidth]{pictures/kyber_all.pdf}
  \caption{Kyber block scheme}
  \label{img:kyber_all}
\end{figure}

\begin{figure}[h!]
  \centering
  \includegraphics[width=0.484\textwidth]{pictures/dil_all.pdf}
  \caption{Dilithium block scheme}
  \label{img:dil_all}
\end{figure}

\end{document}